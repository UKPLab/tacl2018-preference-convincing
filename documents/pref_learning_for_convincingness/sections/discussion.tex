\section{Conclusions and Future Work}

We presented a novel, scalable approach to predicting argument convincingness using Bayesian preference learning,
and demonstrated how our method can outperform SVM and neural network methods with sparse and noisy 
training data.
Several related NLP and argumentation problems could benefit from a similar methodology, particularly in an 
interactive setting where large amounts of clean training data are unavailable. 
An example is the argument reasoning comprehension task\cite{habernal2017arg}, 
where annotators select preferred components to complete an argument. 
Future work will therefore evaluate this preference learning approach on 
other NLP tasks where training data for standard classification and regression is unavailable.

The GP approach could also be used to combine information of different types: classifications and 
absolute scores as well as pairwise labels. The idea of combining scores and pairwise labels 
using an active learning approach for image quality assessment with crowds was demonstrated by
\cite{ye2013combining}, albeit using the Laplace approximation for the Gaussian process. 
In future we plan to investigate the effectiveness of this idea using our 
scalable Gaussian process approach for interactive learning settings with implicit feedback.
