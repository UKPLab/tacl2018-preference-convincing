\documentclass[11pt,letterpaper]{article}
\usepackage[letterpaper]{geometry}
\usepackage{acl2012}
\usepackage{times}
\usepackage{latexsym}
\usepackage[fleqn]{amsmath}
\usepackage{multirow}
\usepackage{url}
\makeatletter
\newcommand{\@BIBLABEL}{\@emptybiblabel}
\newcommand{\@emptybiblabel}[1]{}
\makeatother
\usepackage[hidelinks]{hyperref}
\DeclareMathOperator*{\argmax}{arg\,max}
\setlength\titlebox{6.5cm}    % Expanding the titlebox

\usepackage[draft]{todonotes}

% \documentclass[11pt]{article}
% \usepackage{acl2016}
% \usepackage{times}
% \usepackage{url}
% \usepackage{latexsym}
% 
% \usepackage[fleqn]{amsmath}
% \usepackage{amssymb}
% \usepackage{amstext}
% \usepackage{amsthm}
% 
% \usepackage{cite}

\usepackage{graphicx}
\usepackage{amsfonts}
\usepackage{algorithm2e}
\usepackage{array}
\usepackage[caption=false,font=footnotesize]{subfig}
\usepackage{url}
\usepackage{tabularx}
\usepackage{numprint}
\usepackage{multirow}

\newcommand{\bs}{\boldsymbol}  
\newcommand{\wrtd}{\mathrm{d}}

\makeatletter
\makeatother %some sort of hack related to the symbol @

%%%%%%%%%%%%%%%%%%%%%%%%%%%%%%%%%%%%%%%%%%%%%%%%%%%%%%%%%%%%%%%%%

\title{ 
Finding Convincing Arguments using Scalable Bayesian Preference Learning
}

\author{First Author \\
  Affiliation / Address line 1 \\
  Affiliation / Address line 2 \\
  Affiliation / Address line 3 \\
  {\tt email@domain} \\\And
  Second Author \\
  Affiliation / Address line 1 \\
  Affiliation / Address line 2 \\
  Affiliation / Address line 3 \\
  {\tt email@domain} \\}

\date{}
\begin{document}

\maketitle

% Points to address:
% 1. Is scalability needed when data is sparse? Yes -- for a single topic we quickly hit the 
% limits with > 500 comparisons; these pairs may be noisy rather than just small; the 
% large feature space means data is still sparse; scalabe method means that model can be
% continuously updated as new data arrives.

\begin{abstract}
%Automated methods for identifying convincing arguments can be hampered by 
%a lack of gold-standard ratings or rankings in new topics or domains.
%We address this problem by introducing a scalable Bayesian preference learning method
%for argument convincingness.
We introduce a scalable Bayesian preference learning method for identifying
convincing arguments in the absence of gold-standard ratings or rankings.
% so far, classifiers and graph ranking methods have been adapted to PL -- our approach addresses the problem without a pipeline and without 
In contrast to previous work, we avoid the need for separate approaches or pipelines
to produce training data, predict rankings and perform pairwise classification.
Although Bayesian methods are known to be effective when faced with sparse or noisy training data, 
they have not previously been used to identify convincing arguments.
One deterrent is their perceived lack of scalability, which we address by developing a 
stochastic variational inference method for Gaussian process (GP) preference learning.
We show how our method can be applied to predict argument convincingness from crowdsourced data, 
outperforming state-of-the-art methods, particularly when the data is sparse or noisy.  
We demonstrate how our Bayesian approach enables more effective active learning,
thereby reducing the amount of data required to identify convincing arguments for new users and domains.
While word embeddings are principally used with neural networks, our results show that word embeddings in combination with linguistic features also benefit GPs when predicting
argument convincingness.

\todo{Two options for the paper's main claim (at the moment it is more toward number 1): (1) We provide a scalable Bayesian method that improves preference learning with sparse/noisy data (2) We show state of the art performance for argumentation by developing a scalable Bayesian preference learning method. 
Notes from other papers in source comments:
%From reece 2011 -- data fusion task is put first; main claim is not more general than the application domain of data fusion:
%"The complex target of interest in this paper is the ‘pattern of life’ associated with a counter-insurgency (COIN) scenario involving vehicle movements between specific buildings in a fictitious town... The focus of this paper is the formulation and proof-of- principle testing of a mathematical framework for representing and fusing heterogeneous data sources to provide inference in
%support of intelligence analysts."
% Look at structure of intro to "A Bayesian Approach to Unsupervised Semantic Role Induction". The application domain is front-and-centre here. 
Concerns about the first claim -- currently it is phrased in a way that is too general, e.g. "We demonstrate how our Bayesian approach enables more effective active learning"; it would have to be more specific to the scalable method presented, e.g. "...from sparse and noisy preferences with large real-world NLP datasets". To emphasise claim 2: "...of argument convincingess from small numbers of judgements provided by human annotators". 
Options: (1) claim 1 should be played down completely so it can be made into a separate ML paper; (2) claim 2 becomes focus but claim 1 is rephrased to specifically back it up; (3) claim 1 is the main focus but need a lot more background on the use cases in NLP and prior work on Bayesian methods in NLP + sell as new tool for NLP.
}
\end{abstract}

% For peer review papers, you can put extra information on the cover
% page as needed:
% \ifCLASSOPTIONpeerreview
% \begin{center} \bfseries EDICS Category: 3-BBND \end{center}
% \fi
%
% For peerreview papers, this IEEEtran command inserts a page break and
% creates the second title. It will be ignored for other modes.
%\IEEEpeerreviewmaketitle

%%%%%%%%%%%%%%%%%%%%%%%%%%%%%%%%%%%%%%%%%%%%%%%%%%%%%%%%%%%%%%%%%

\section{Introduction}\label{sec:intro}

% Possibly useful phrasing: https://arxiv.org/abs/1707.08349
% The goal of this paper is to demonstrate that our shallow and simple approach based on ... (with minor 
% improvements) can pass the test of time and reach state-of-the-art performance in ...

% Finding well-written and convincing arguments from large bodies of text
%   could enable better decision-making and analysis of controversial topics.
% However, sources of information may vary from social media posts to
%   articles and books, and the language used in each new topic can vary substantially.
%   This presents a challenge when training machine learning algorithms to identify convincing
%   arguments, since annotated data is in short supply. 
% Pairwise preference judgements provide a convenient way for multiple people to communicate the relative convincingness of arguments.
% Implicit preferences can be elicited from user actions, such as selecting a document from a list given its summary to read in more detail. 

Argumentation is intended to persuade the reader of a particular point of view and 
is an important way for humans to reason about controversial topics \cite{mercier2011humans}. 
The amount of argumentative text on any given topic can, however, overwhelm a reader, particularly
considering the scale of historical text archives 
and the prevalence of social media platforms with millions of authors.
To gain an understanding of a topic it is therefore useful to identify high-quality, 
persuasive arguments from different sides of a debate. 
Whether an argument is persuasive or not is subjective \cite{lukin2017argument},
hence analysing which arguments a particular person or group of people finds convincing can tell us
about their opinions and influences.
% Motivations for modelling argument convincingness:
% \begin{itemize}
%   \item Learning about a controversial topic often requires reading large amounts of text, often with much duplicate information, in order to understand different points of view
%   \item Points of view on controversial topics are often presented as arguments for or against a particular position
%   \item Finding well-written arguments could allow better understanding of why people hold particular opinions
%   \item Identifying arguments that are considered convincing to a particular group of people helps understand who holds which point of view
%   \item Tools that identify convincing arguments could therefore assist in making better decisions and analysing public opinion
% \end{itemize}

Previous work \cite{habernal2016argument} showed that it is possible to predict  the
convincingness of arguments taken from online discussion forums with reasonable accuracy,
using models trained on one topic and transferred to another.
Their experiments made use of pairwise preference labels indicating which argument in a pair the annotator thought
was more convincing. 
As a means of eliciting convincingness, pairwise preferences have a number of advantages. 
Unlike ratings or scores, they do not require calibrating, even if multiple people provide the labels,
e.g. to mitigate the fact that some annotators may avoid very high or very low ratings, or may be biased toward particular scores.
Pairwise comparisons are also more fine-grained than categorical labels 
and can lead to more reliable results with less cognitive burden on human annotators\cite{kendall1948rank,kingsley2006preference}. 
Implicit preferences can also be elicited from user actions, 
such as selecting a document from a list given its summary to read in more detail\cite{joachims2002optimizing}.
% Preference learning from pairwise preferences is effective because it removes the need for humans to provide scores or classifications and allows them to make relevance judgements,
% which have been shown to be easier for human annotators in many cases\cite{brochu_active_2007}. Pairwise comparisons also occur in implicit feedback, for example, when a user chooses to click on link from a list of several. They are therefore a useful tool for practical learning from end users. 
% However, the pairwise comparisons we observe may not be a perfect representation of their preferences as they may contain noise, leading to inconsistencies where items cannot be ranked in such a way that the ranking agrees with all the observed comparisons. 
  
In practice, however, preference data may be noisy -- particularly if obtain from crowds or implicit feedback -- 
and we may be faced with very small amounts of data when we move to new domains, topics and users for whom we wish 
to predict convincingness. Small data can present a problem to methods such as 
deep neural networks\cite{srivastava2014dropout}.
The approach used by \cite{habernal2016argument} to handle unreliable crowdsourced data involved 
first determining consensus labels using MACE\cite{hovy2013learning} and then ranking using PageRank
to obtain training data for regression.
Such pipeline approaches can be prone to error propagation\cite{chen2016joint} and require multiple 
crowdsourced labels 
for each argument pair to avoid individual errors. 

% \begin{itemize}
%   \item New types of text, new domains, and new users with different preferences means we may face situations in practice where models trained on existing corpora are less effective, but data for the new task is limited (sparse)
%   \item Use two different pipelines consisting of multiple steps: combining crowdsourced data, removing inconsistencies, classification; combining crowdsourced data, ranking using PageRank, regression
%   \item In low-data situations, these approaches may underperform, since model uncertainty is not accounted for between each stage, nor in the final predictions, and errors propagate along the pipeline
%   \item Training data may also contain errors, which would be propagated through the pipeline (this was avoided in previous work by combining labels from multiple crowdworkers; we should be able to handle the case where this is not possible).
%   \item Feature space becomes very large when working with textual features -- can we narrow it down automatically to improve scalability and improve performance?
% \end{itemize}

In contrast to previous work, we propose the use of preference learning techniques for argument convincingness
to directly model the relationship between crowdsourced preferences and textual features, including word embeddings.
We choose a Bayesian approach, since Bayesian methods have been shown to successfully handle the problem of small(for example, \cite{xiong2011bayesian,titov2012bayesian}) and unreliable datasets (e.g. \cite{simpson2015language}),
and provide a good basis for active selection to reduce labelling costs\cite{mackay1992information}.
Our method is based on the Gaussian process (GP) model of \cite{chu2005preference},
which assumes that preferences over items are described by a latent preference function.
By providing a Bayesian treatment to this latent function, the method handles uncertainty
in the function values due to noise and data sparsity in a principled manner.
GP preference learning (GPPL) has not previously been applied to text problems with large
numbers of features and the inference scheme proposed by \cite{chu2005preference}
was limited by a computational complexity of $\mathcal{O}(N^3)$, where $N$ is the number of items.
We address the problem of scalability by applying recent advances in stochastic variational
inference (SVI) \cite{hoffman2013stochastic}  to this model, and
developing an efficient optimisation technique for key hyper-parameters.
We then show how our method can be applied to argument convincingness 
with a large number of linguistic features and high-dimensional text embeddings.
%   \item Confidence estimates from Bayesian models account for sparsity and noise in data, as well as uncertainty in the model. This means they do not make overly-confident predictions when training data is small (they know when they don't know).
%  \item Bayesian preference learning methods have been proposed but scalable implementations were not developed and models have not been applied to text with large numbers of features
%  \item We address the limitations above by adapting Bayesian preference learning approach to argumentation
%  \item Introduce stochastic variational inference (SVI) to train the model on large numbers of preferences and documents
%  \item Develop gradient-based ARD to identify relevant text features
%\end{itemize}
Our evaluation compares Bayesian preference learning to established SVM and neural network approaches for predicting convincing arguments, and show that our approach can
outperform these alternatives particularly with small and noisy datasets. 
%  \item Show that Bayesian Gaussian process (GP) models are applicable to performing preference learning over text (existing evaluation of GPs for text is very limited, although they have been used extensively with great success in domains such as Physics, finance, Biology. This is possibly because GPs were seen as more difficult to implement and could not be scaled up until recent advances such as SVI)
%  \item Evaluate the ability of each method to handle noisy and sparse data, showing improved performance using our method in the presence of noise and data sparsity
 % \item Analyse the features that are most informative when determining convincingness, providing insight into what makes a convincing argument
%\end{itemize}

The rest of the paper is structured as follows.
First, we review related work in more detail: on argumentation; Bayesian methods for preference learning; and scalable approximate inference.
We then explain the preference learning approach in detail and develop our SVI inference  and hyper-parameter optimisation methods.
The following section details a number of experiments: a comparison with the
state-of-the art on predicting preference in online debates; 
noisy dataset; active learning; and feature relevance determination.
Finally, we present some conclusions and avenues for future work.

\section{Related Work}\label{sec:related}

Recent work on argumentation by \cite{habernal2016argument} has established datasets and methods for
predicting which argument is most convincing. Our experiments make extensive use of this data to establish
a different methodology. This work was also extended to evaluate the reasons why one argument
is more convincing than another\cite{habernal2016makes}, however our paper focusses on prediction when reasons are not given. 
Investigations by \cite{lukin2017argument} demonstrated the effect of personality and prior stance 
of the audience on the persuasiveness of arguments,
although their work does not extend to modelling this persuasiveness using preference learning.
The sequence of arguments in a dialogue is another important factor in their ability to change
the audience's opinions \cite{tan2016winning}.
This idea is used by \cite{rosenfeld2016providing,monteserin2013reinforcement}, 
who address the problem of choosing the best argument in a dialogue between a human user and an agent. 
However, these works focus on applying  reinforcement learning to predict the best argument 
to present in a sequence rather than learning user preferences for arguments with certain qualities.
 
% GP-based approach versus Bradley/Terry/Luce/MPM method or Mallows models?
The goal of preference learning is to predict a ranking over items in terms of preference,
or to predict which single item $x_i$ in a pair or small set would be chosen by the user. 
A preference for item $x_i$ over $x_j$ is written as $x_i \succ x_j$.
Given a ranking over items, it is possible to determine the pairwise preferences,
but pairwise labels can also be predicted using a generic classifier without the need to learn a total ordering.
During training and prediction, pairs of items are transformed either by concatenating the feature vectors of two items as in \cite{habernal2016argument}, 
or computing the difference of the two feature vectors as in SVM-Rank\cite{joachims2002optimizing}. 
The classifier is then trained as normal with preference labels treated as binary class labels.

However, the ranking of items is useful for producing ordered lists in response to a query -- 
consider a sorted list of the most convincing arguments in favour of topic X.
Another approach is to learn this ordering directly using Mallows models\cite{mallows1957non},
which define distributions over permutations of a list. 
Mallows models have been extended to provide a generative model\cite{qin2010new} and 
to be trained from pairwise preferences rather than by observing rankings\cite{lu2011learning}. 
A disadvantage of Mallows models is that inference is typically costly, 
since the number of possible permutations to be considered is $\mathcal{O}(N^2)$, 
where $N$ is the number of items to be ranked. 
Modelling only the order of items means we are unable to quantify 
how closely rated items at similar ranks are to one another: how much better is the top ranked item 
from the second-rated?

% Bradley-Terry: MPM\cite{volkovs_new_2014}
To avoid the problems of classifier-based and permutation-based methods, 
another approach is to learn a set of underlying real-valued scores from pairwise labels.
These scores can then be used to predict rankings, pairwise labels, or ratings for individual items.
To do this, a model is required to map the real-valued scores to discrete pairwise labels.
Two established approaches for this are based on the Bradley-Terry-Plackett-Luce model \cite{bradley1952rank,luce1959possible,plackett1975analysis}
and the Thurstone-Mosteller model\cite{thurstone1927law,mosteller2006remarks}.
In more recent work, Bayesian extensions of the Bradley-Terry-Plackett-Luce model
were proposed by \cite{guiver2009bayesian,volkovs_new_2014}, 
while the Thurstone-Mosteller model was used by \cite{chu2005preference}.
This latter piece of work assumes a Gaussian process (GP) prior over the scores,
which enables us to predict scores for previously unseen items given their features 
using a Bayesian nonparametric approach.
Gaussian processes have been well established as effective and versatile models that
extrapolate from training data in a principled manner, taking into account model uncertainty
\cite{rasmussen_gaussian_2006}.
Their nonparametric nature means that the function complexity can grow with the amount of data observed.
These characteristics make them suitable for the task of modelling argument convincingness
where data for new topics, domains and users is limited. 
% The Gaussian process (GP) preference learning approach of \cite{chu2005preference} resolves inconsistencies between preferences and provides a way to predict rankings or preferences for 
% items for which we have not observed any pairwise comparisons based on the item's features. 
% This model assumes that preferences are noisy, i.e. contain some erroneous labels.
% particularly as the modular nature of inference algorithms such as Gibb's sampling and variational approximation is suited to extending the model to handle different types of feedback that give indications of some underlying preferences. 

The method developed by \cite{chu2005preference} used the Laplace approximation to perform approximate
inference over the model. Unfortunately the memory and computational costs scale with $\mathcal{O}(N^3)$
due to matrix inversion. If this limitation is overcome, there is still a computational and memory cost 
during training of $\mathcal{O}(N^2)$ due to the number of pairs in the training dataset.
Such problems are common when performing inference over Gaussian process models but have been addressed
by \cite{hensman2013gaussian,hensman_scalable_2015} for regression and classification tasks using
the stochastic variational inference (SVI) algorithm proposed by \cite{hoffman2013stochastic}. 
SVI has, however, not previously been adapted for preference learning with GPs.
% The modular nature of VB allows us to take advantage of models for feedback of different types
% where the input values for each type of feedback do not directly correspond (e.g. explicit user ratings and number of clicks may have different values).
% By using SVI, we provide a formal way to deal with scalability that comes with guarantees\cite{hoffman2013stochastic}.
The next section explains our preference learning method for argument convincingness.

%GPs for NLP in other task areas?

\section{Identifying Common Patterns of Convincingness}\label{sec:model}

\subsection{Baseline methods}

\begin{itemize}
  \item Random: select a label at random
  \item Most common (MC): select the most common preference label from across the dataset
  \item No differentiation (ND): we do not model differences between workers. Labels are estimated by taking the average of other people's labels for the same preference pair. When there are no previous pairs available, select the most common preference label
  \item Gaussian process preference learning with no differentiation (GP-ND): learn a latent ranking function for the objects from pairwise preferences, ignoring differences between workers and features of the arguments. This provides a probabilistic variant of ND  
\end{itemize}

\subsection{Modelling Correlations Between Individuals}

Two main types of approach:
\begin{itemize}
  \item Factor analysis -- map the set of pairwise preferences to a low-dimensional embedding
  \item Clustering -- assumes that people fall into distinct preference clusters, or can be modelled as a mixture of several archetypes
\end{itemize}

Specific methods to test can be split into several types. First,
we can run different clustering methods on the training data, 
then predict a worker's label by taking the mean of the other cluster members. 
When the no members of the cluster have labelled the pair, we predict using the most common label.
This method is applied to several clustering algorithms:
\begin{itemize}
   \item Affinity propagation (AP-mean)
   \item Gaussian mixture model, using most probable cluster assignment (GMM-mean)
   \item Gaussian mixture model, using cluster assignments weighted by probability (GMM-WM)
\end{itemize}

A similar approach can be taken with dimensionality reduction techniques, where we can use K-nearest neightbours (in this case, few workers label each pair, so we choose k=1 and use MC when no workers have labelled the current instance?):
\begin{itemize}
   \item Factor analysis with K-nearest neighbours (FA-KNN)
\end{itemize}
Alternatively, we can take a weighted average of the other labels for a pair, where the weights are based on inverse distance from the worker in question in the embedded space:
\begin{itemize}
   \item Factor analysis with an inverse distance-weighted mean (FA-weighted)
\end{itemize}
The distance function can be optimised, which leads to proposing more sophisticated methods...

\section{Bayesian Preference Learning Model}

The model introduced in \cite{houlsby2012collaborative} combines preference learning with matrix factorisation 
to identify latent features of items and users that affect their preferences. This allows for a collaborative filtering effect, whereby users with similar preferences on a set of observed items are assumed to have similar 
preferences for other items with similar features. This allows us to make better predictions about the unobserved preferences of a given user when we have seen preferences of a similar user.

The method presented in \cite{houlsby2012collaborative} uses a combination of expectation propagation (EP) and variational Bayes (VB). Since the inference steps require inverting a covariance matrix, this method scales with 
$\mathcal{O}(N^3)$ and is therefore impractical for large datasets. For our modified version of this method, we improve scalability by using stochastic variational inference to infer the complete model. 
The variational approximation to the posterior is given by...

The variational inference algorithm maximises a lower bound on the log marginal likelihood:
\begin{flalign*}
  \mathcal{L} = \sum_{i=1}^N \mathbb{E}[ \log p(t_i | x_{i,1}, x_{i,2}, \bs f) ] + \nonumber\\
  \sum_{u=1}^U  \mathbb{E}\left[ \log \frac{p(\bs f_u | \bs w \bs y_u, \bs K_{f,u} / s_{f,u} )}{q(\bs f_u)}\right] + \nonumber\\
  \sum_{c=1}^C \mathbb{E}\left[\log\frac{p(\bs w_c | \bs 0, \bs K_w / s_{w,c} ) }{q(\bs w_c) } \right] + \nonumber\\
  \sum_{c=1}^C \mathbb{E}\left[\log\frac{p(\bs y_c | \bs 0, \bs K_y / s_{y,c} ) }{q(\bs y_c) } \right] + \nonumber\\
  \mathbb{E}\left[\log\frac{p(\bs t | \bs \mu, \bs K_{t} / s_t ) }{q(\bs t)} \right] + \nonumber\\
  \sum_{u=1}^U \mathbb{E}\left[\log\frac{p(s_{f,u} | a_{f,u}, b_{f,u})}{q(s_{f,u})}\right] + \nonumber\\
  \sum_{d=1}^D \mathbb{E}\left[\log\frac{p(s_{w,d} | a_{w,d}, b_{w,d})}{q(s_{w,d})}\right] +\nonumber\\
  \sum_{d=1}^D \mathbb{E}\left[\log\frac{p(s_{y,d} | a_{y,d}, b_{y,d})}{q(s_{y,d})}\right] 
\end{flalign*}
where $t_i$ is the preference label for the $i$th pair, 

To perform feature selection with large numbers of features, we introduce 
an automatic relevance determination
(ARD) approach that uses the gradient of the lower bound on the log marginal likelihood to optimise the kernel length-scales using the conjugate gradient method. The gradient is given by:
\begin{flalign*}
\nabla\mathcal{L} = \left[ \frac{\partial \mathcal{L}}{\partial l_{w,1}}, ...,  \frac{\partial \mathcal{L}}{\partial l_{w,D_w}},  \frac{\partial \mathcal{L}}{\partial l_{y,1}}, ...,  \frac{\partial \mathcal{L}}{\partial l_{y,D_y}} \right], &&\\
\frac{\partial \mathcal{L}}{\partial l_{w,d}} = \frac{\partial}{\partial l_{w,d}} 
\sum_{u=1}^U  \mathbb{E}\left[\log\frac{p(\bs f_u | \bs w \bs y_u, \bs K_{f,u} / s_{f,u} )}{q(\bs f_u)}\right] + \nonumber && \\
\sum_{c=1}^C \mathbb{E}\left[\log\frac{p(\bs w_c | \bs 0, \bs K_w / s_{w,c} ) }{q(\bs w_c) } \right] - \nonumber&&\\
\sum_{u=1}^U \mathbb{E}\left[\log q(s_{f,u})\right] - \sum_{d=1}^D \mathbb{E}\left[\log q(s_{w,d})\right] +\nonumber&&\\
= 0.5 (\hat{f}_u - wy_u)^T \bs K_{f,u}^{-1} \frac{\partial \bs K}{\partial \log l_{w,d}} \hat{s}_{f,u} \bs K_{f,u}^{-1} (\hat{f}_u - wy_u) \nonumber\\
- 0.5\mathrm{tr}\left( (\bs K_{f,u}^{-1} - \frac{\bs C^{-1}}{\hat{s}_{f,u}} ) \frac{\partial \bs K_{f,u}}{\partial \log l_{w,d}} \right)\nonumber&&\\
\frac{\partial \mathcal{L}}{\partial l_{y,d}} = &&\\
\end{flalign*}
where $l_{w,d}$ is a length-scale used for all the GPs over item features. The implicit terms are zero when the VB algorithm has converged.

\pagebreak
No
\pagebreak

\section{Experiments}\label{sec:expts}

The first dataset contains a number of pairwise convincingness preference labels for a set of arguments from a crowd of workers. Each label is associated with two arguments and expresses whether a worker in the crowd found the first argument most convincing, the second argument, or had no preference.  

The task is to train the models then predict the preference labels for held-out data. Each method can be assessed in terms of classification accuracy, since the labels have three possible values. In the first set of experiments we evaluate the baselines and the different methods for modelling correlations between workers' preferences. In the second set of experiments, we assess the value of different language features. Finally, the third experiment evaluates approaches that integrate both argument features and models of preference correlations.

\subsection{Hypotheses -- needs to be reconciled with the above paragraph} 

Prior work on convincingness:
\begin{itemize}
 \item \cite{habernal2016argument} shows how to predict convincingness of arguments by training a NN 
 from crowdsourced annotations. 
 \item \cite{lukin2017argument} shows that persuasion is correlated with personality traits.
\end{itemize}

We build on this to show...
\begin{itemize}
 \item How we can predict convincingness for a specific user given only previous preferences and 
 preferences of others (collaborative filtering)
 \item How we can rank arguments in terms of convincingness using preference learning, and resolve
 conflicts.
 \item How a combination of text and personality features improves predictions of convincingness
 \item Which input features/side information features are informative and how we can learn this using a Bayesian approach
 \item That we can extract human-interpretable latent features in people and items,
 which improve performance over just using the input features.
 \item Uncertainty is modelled correctly in the Bayesian approach so that (a) we can filter
 out the uncertain decisions to improve accuracy; (b) the brier score/cross entropy error is lower;
 \item Uncertainty also means that simple active learning works better
 \item Bayesian approach means accuracy is better with sparse data, e.g. in the cold-start situation,
  esp. if we can use (a) and (c) to avoid acting on uncertain labels.
\end{itemize}
This is all useful because we can use the approach to determine which features are worth 
obtaining, make predictions when data is sparse, and obtain data from users efficiently.

The steps to show this are:
\begin{enumerate}
  \item Show a table comparing the baselines, alternative collaborative filtering methods, 
  results from \cite{habernal2016argument}, and unsupervised method
  \item Add in results when using the input information with out method
  \item Show a table comparing the baselines, alternative collaborative filtering methods, 
  results from \cite{lukin2017argument}, and unsupervised method
  \item Add in results using item information, person information and both
  \item Visualise latent features?
  \item Table showing importance of input features
  \item Add results with lower confidence items excluded to the tables in 1-4. We can also plot the
  effect of confidence threshold on our results and on the rival methods.
  \item Add in Bier/cross entropy -- may need to rerun the original code from the previous papers?
  \item Run \cite{habernal2016argument} and my complete method with reduced data -- check accuracy as it increases. Use confidence cut-off from previous results.
  \item Simple active learning approach selecting the most uncertain data point (this will be due to 
  uncertainty about a person, an item with too little data, or disagreement/stochasticity in the likelihood). The plot can be added to the previous results and should be run with rival methods.
\end{enumerate}

\begin{table*}
  \begin{tabularx}{\textwidth}{ l  X  X  X }
  Dataset & Dataset properties & Hypothesis & Methods \\
  \hline\hline\\
  1. UKPConvArgStrict & 
  MACE output with confidence $\ge 95\%$; \newline
  Discard arguments marked as equally convincing; \newline
  Discard conflicting preferences. & 
  Bayesian method is competitive with previous methods at predicting clean preference pairs from a clean dataset. &
  GP Preference learning + linguistic features + embeddings. \\\hline\\
  2. UKPConvArgMACE & 
  MACE output with confidence $\ge 95\%$;\newline
  No further filtering. & 
  Bayesian method is competitive with previous methods if filtering step is removed. &
  GP Preference learning + linguistic features + embeddings. \\\hline\\      
  3. UKPConvArgRank & 
  MACE output with confidence $\ge 95\%$;\newline
  Equal arguments included; \newline
  PageRank used to rank arguments. & 
  Bayesian method is competitive with previous methods at ranking arguments and 
   can perform ranking given pairs rather than rank scores. &
  GP regression with preference learning output + linguistic features + embeddings (trained on rank scores); \newline
  GP Preference learning + linguistic features + embeddings (trained on pairs). \\\hline\\  
  4. UKPConvArgAll & 
  No filtering, all pairs from original workers are provided. & 
  Bayesian method can predict argument pairs for individual annotators with competitive performance to rival methods on clean, combined data; \newline
  There are patterns of common agreement/disagreement among workers. &
  Bayesian Preference Components + linguistic features + embeddings (pair prediction). \\\hline\\
  5. UKPConvArgAllR & 
  No filtering, all pairs from original workers are provided; \newline 
  PageRank used to produce gold-standard ranking. & 
  Bayesian model can predict individual argument rankings;\newline
  Significant differences between individual rankings and gold-standard ranking. &
  Bayesian Preference Components + linguistic features + embeddings (ranking output). \\\hline\\ 
  %UKPConvArgAll (train), UKPConvArgStrict (test) &
  %Use the original pairs for training and try to predict the cleaned output &
  %(Optional -- more about preference learning from noisy data, so probably deserves a separate paper/student project?)
  %Show that the model can also be used to predict a gold standard -- end-to-end preference learning &
  %Bayesian Preference Components + linguistic features + embeddings + gold-standard pairs on the training data (model is not really designed for aggregation, but for sharing information between similar people and items; without gold standard training labels, we need to know the y-feature mixture of the gold-standard); 
  %HeatmapBCC with preference learning forward model (deals directly with noisy labels, rather than different opinions about convincingness; useful for learning from implicit feedback)
  \end{tabularx}
  \caption{\label{tab:expt_data} The datasets and hypotheses in each experiment.}
\end{table*}

\subsection{Alternative Application -- Notes}

The core contribution is to do preference learning with sparse observations with text data. 
There may be several other problems related to NLP and argumentation, more specifically, that
could benefit from this approach. 
Argument cloze task? 
The model can be adapted to classification problems, regression, or mixed observation types by applying a different likelihood. The core of the method is the abstraction of a latent function over items and people, dependent on latent features of items and people, with the ability to include side information.
The paper could therefore apply the model to multiple NLP tasks, and be a methodology paper. 
This needs us to discuss and distinguish the class of problems and novelty of the method. 
What are the alternative methods, e.g. if we were to use this for classification? One could supply all 
item and person data to a neural network and train it in a semi-supervised manner?
Sticking to the idea of preference learning or ranking, what other tasks could be handled in this way?

\section{Conclusions}\label{sec:conclusion}

We proposed a novel Bayesian preference learning approach 
for modelling both the preferences of individuals 
and the overall consensus of a crowd. 
Our model learns the latent utilities of items from pairwise comparisons 
using a combination of Gaussian processes and Bayesian matrix factorisation 
to capture differences in  opinion.
We introduce a stochastic variational inference (SVI) method, that, 
unlike previous work, can scale to arbitrarily large datasets,
since its time and memory complexity do not grow with the dataset size.
Our experiments confirm the method's scalability and
 show that jointly modelling the consensus and personal
preferences can improve predictions of both.
Our approach performs competitively
against less scalable alternatives
%despite the approximations required for SVI.
and improves on 
the previous state of the art
for predicting argument convincingness from crowdsourced data~\citep{simpson2018finding}.
%far better than previous
%Gaussian process preference learning methods 
%without harming  
%performance on individual utility prediction and significantly improving performance
%on consensus learning.

Future work will investigate learning inducing point locations and
optimising length-scale hyperparameters by maximising $\mathcal L$ as part of the variational inference method.
Another important direction will be to generalise the likelihood from pairwise comparisons
to comparisons involving $k$ items~\citep{pan2018stagewise}
or best--worst scaling~\citep{kiritchenko2017best}
to provide scalable Bayesian methods for other forms of comparative preference data.
%and investigate the possibility of integrating deep generative models 
%for learning feature representations from input data.
%the models can readily be adapted to regression or classification tasks by swapping out the preference likelihood, resulting in 
%different values for $\bs G$ and $\bs H$.
% Preference elicitation using information theoretic methods.


%%%%%%%%%%%%%%%%%%%%%%%%%%%%%%%%%%%%%%%%%%%%%%%%%%%%%%%%%%%%%%%%%%%%%%%%%%%%%%%%

% use section* for acknowledgment
\section*{Acknowledgments}

\cleardoublepage

% \addcontentsline{toc}{chapter}{Bibliography}
%\bibliographystyle{apalike}
\bibliographystyle{acl2012}
\bibliography{simpson_pref_learning_for_convincingness}

%%%%%%%%%%%%%%%%%%%%%%%%%%%%%%%%%%%%%%%%%%%%%%%%%%%%%%%%%%%%%%%%%%%%%%%%%%%%%%%%%

\end{document}
