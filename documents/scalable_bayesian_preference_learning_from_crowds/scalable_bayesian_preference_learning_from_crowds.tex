\RequirePackage[fleqn]{amsmath}
\RequirePackage{fix-cm}
%
\documentclass[smallcondensed,natbib]{svjour3}     % onecolumn (ditto)
%
\smartqed  % flush right qed marks, e.g. at end of proof
%
\usepackage{graphicx}

%\usepackage{mathptmx} % this causes problems with bold font

\usepackage{url}
\makeatletter
\makeatother
%\usepackage[hidelinks]{hyperref}
\DeclareMathOperator*{\argmax}{arg\,max}

\usepackage[fleqn]{amsmath}
%\usepackage{amssymb}
% \usepackage{amstext}
% \usepackage{amsthm}
\usepackage{amsfonts}
\usepackage{algorithm2e}
\usepackage{array}
\usepackage[caption=false,font=footnotesize]{subfig}
\usepackage{url}
\usepackage{tabularx}
\usepackage{booktabs}
\usepackage{numprint}
\usepackage{multirow}
\usepackage{xcolor}
\usepackage{todonotes}

\newcommand{\bs}{\boldsymbol}  
\newcommand{\wrtd}{\mathrm{d}}

\makeatletter
\makeatother %some sort of hack related to the symbol @

%%%%%%%%%%%%%%%%%%%%%%%%%%%%%%%%%%%%%%%%%%%%%%%%%%%%%%%%%%%%%%%%%

\title{ 
Predicting Preferences for Crowds of Users: a Scalable, Bayesian Approach
%Scalable Bayesian Preference Learning for Crowds of Users
% title change to emphasise that this is a personalised model, not all about aggregation
%Scalable Bayesian Methods for Personalised Preference Learning
%One source of confusion seems to be the inclusion of features, which is a big distinction.
% predictive feature-based models.
% ... with item and user features
% predictive modelling with input features
%Learning predictive preference models from crowds of users: a scalable Bayesian method
% Predicting preferences of crowds of users: a scalable, Bayesian approach
}

\author{Edwin Simpson 
\and Iryna Gurevych \\
Ubiquitous Knowledge Processing Lab, Dept. of Computer Science, Technische Universit\"at Darmstadt, Germany\\
              \email{\{simpson,gurevych\}@ukp.informatik.tu-darmstadt.de}
}
\date{Received: date}
\begin{document}

\titlerunning{Scalable Bayesian Preference Learning}
\authorrunning{Simpson, E and Gurevych, I}

% do not exceed 20 pages including references

\maketitle

\begin{abstract}
We propose a scalable Bayesian preference learning method 
for jointly predicting the preferences of individuals as well as the consensus of a crowd
 from pairwise labels.
Peoples' opinions often differ greatly,
making it difficult to predict their preferences if the amount of data for each user is small.
These personal biases also make it harder to infer the consensus of the whole crowd
when there are few labels per item.
%In applications such as recommendation it is also necessary to predict the preferences of individual annotators or users. 
%and identify common preferences between users. % use of the latent factors 
%We address these challenges by combining matrix factorization to model individual preferences with 
%Gaussian processes to integrate user and item features. By taking a Bayesian approach, our model
We address these challenges by combining matrix factorization with 
Gaussian processes,
using a Bayesian approach to account for uncertainty arising from sparse data and annotation noise.
Our method exploiting input features, such as text embeddings, image features or metadata,
to predict preferences for new items and users with limited or no training data.
As previous methods for Gaussian process preference learning do not scale to 
large numbers of users, items or pairwise labels, 
we propose a stochastic variational inference approach that limits computational and memory costs.
Our experiments on a benchmark recommendation task show that
our method is competitive with previous approaches despite our scalable inference approximation.
We demonstrate the method's scalability empirically
on a natural language processing task with thousands of users and items.
On this task, our method improves consensus prediction over the state of the art
by modelling the preferences of individual members of the crowd.
%We also show how to  %that robustness to %able to learn the effective number of components required to model the data and
%choosing more latent components than required,
%apply gradient-based optimization to length-scale hyper-parameters to improve performance.
We make our software publicly available for future 
work~\footnote{\url{https://github.com/UKPLab/tacl2018-preference-convincing/tree/crowdGPPL}}.
%We show how to make collaborative preference learning work at scale and how it can be used to learn
%a target preference function from crowdsourced data or other noisy preference labels. 
%The collaborative model captures the reliability of each worker or data source and models their biases and error rates. 
%It uses latent factors to share information between similar workers and a target preference function.
%We devise an SVI inference schema to enable the model to scale to real-world datasets.
%Experiments compare results using standard variational inference, laplace approximation and SVI.
%On real-world data we show the benefit of the personalised model over a GP preference learning approach 
%that treats all labels as coming from the same source,
%as well as established alternative methods and classifier baselines.
%We show that the model is able to identify a number of latent features for the workers and for textual arguments.
\end{abstract}

% For peer review papers, you can put extra information on the cover
% page as needed:
% \ifCLASSOPTIONpeerreview
% \begin{center} \bfseries EDICS Category: 3-BBND \end{center}
% \fi
%
% For peerreview papers, this IEEEtran command inserts a page break and
% creates the second title. It will be ignored for other modes.
%\IEEEpeerreviewmaketitle

%%%%%%%%%%%%%%%%%%%%%%%%%%%%%%%%%%%%%%%%%%%%%%%%%%%%%%%%%%%%%%%%%

\section{Introduction}\label{sec:intro}

% Possibly useful phrasing: https://arxiv.org/abs/1707.08349
% The goal of this paper is to demonstrate that our shallow and simple approach based on ... (with minor 
% improvements) can pass the test of time and reach state-of-the-art performance in ...

% Finding well-written and convincing arguments from large bodies of text
%   could enable better decision-making and analysis of controversial topics.
% However, sources of information may vary from social media posts to
%   articles and books, and the language used in each new topic can vary substantially.
%   This presents a challenge when training machine learning algorithms to identify convincing
%   arguments, since annotated data is in short supply. 
% Pairwise preference judgements provide a convenient way for multiple people to communicate the relative convincingness of arguments.
% Implicit preferences can be elicited from user actions, such as selecting a document from a list given its summary to read in more detail. 

Argumentation is intended to persuade the reader of a particular point of view and 
is an important way for humans to reason about controversial topics \cite{mercier2011humans}. 
The amount of argumentative text on any given topic can, however, overwhelm a reader, particularly
considering the scale of historical text archives 
and the prevalence of social media platforms with millions of authors.
To gain an understanding of a topic it is therefore useful to identify high-quality, 
persuasive arguments from different sides of a debate. 
Whether an argument is persuasive or not is subjective \cite{lukin2017argument},
hence analysing which arguments a particular person or group of people finds convincing can tell us
about their opinions and influences.
% Motivations for modelling argument convincingness:
% \begin{itemize}
%   \item Learning about a controversial topic often requires reading large amounts of text, often with much duplicate information, in order to understand different points of view
%   \item Points of view on controversial topics are often presented as arguments for or against a particular position
%   \item Finding well-written arguments could allow better understanding of why people hold particular opinions
%   \item Identifying arguments that are considered convincing to a particular group of people helps understand who holds which point of view
%   \item Tools that identify convincing arguments could therefore assist in making better decisions and analysing public opinion
% \end{itemize}

Previous work \cite{habernal2016argument} showed that it is possible to predict  the
convincingness of arguments taken from online discussion forums with reasonable accuracy,
using models trained on one topic and transferred to another.
Their experiments made use of pairwise preference labels indicating which argument in a pair the annotator thought
was more convincing. 
As a means of eliciting convincingness, pairwise preferences have a number of advantages. 
Unlike ratings or scores, they do not require calibrating, even if multiple people provide the labels,
e.g. to mitigate the fact that some annotators may avoid very high or very low ratings, or may be biased toward particular scores.
Pairwise comparisons are also more fine-grained than categorical labels 
and can lead to more reliable results with less cognitive burden on human annotators\cite{kendall1948rank,kingsley2006preference}. 
Implicit preferences can also be elicited from user actions, 
such as selecting a document from a list given its summary to read in more detail\cite{joachims2002optimizing}.
% Preference learning from pairwise preferences is effective because it removes the need for humans to provide scores or classifications and allows them to make relevance judgements,
% which have been shown to be easier for human annotators in many cases\cite{brochu_active_2007}. Pairwise comparisons also occur in implicit feedback, for example, when a user chooses to click on link from a list of several. They are therefore a useful tool for practical learning from end users. 
% However, the pairwise comparisons we observe may not be a perfect representation of their preferences as they may contain noise, leading to inconsistencies where items cannot be ranked in such a way that the ranking agrees with all the observed comparisons. 
  
In practice, however, preference data may be noisy -- particularly if obtain from crowds or implicit feedback -- 
and we may be faced with very small amounts of data when we move to new domains, topics and users for whom we wish 
to predict convincingness. Small data can present a problem to methods such as 
deep neural networks\cite{srivastava2014dropout}.
The approach used by \cite{habernal2016argument} to handle unreliable crowdsourced data involved 
first determining consensus labels using MACE\cite{hovy2013learning} and then ranking using PageRank
to obtain training data for regression.
Such pipeline approaches can be prone to error propagation\cite{chen2016joint} and require multiple 
crowdsourced labels 
for each argument pair to avoid individual errors. 

% \begin{itemize}
%   \item New types of text, new domains, and new users with different preferences means we may face situations in practice where models trained on existing corpora are less effective, but data for the new task is limited (sparse)
%   \item Use two different pipelines consisting of multiple steps: combining crowdsourced data, removing inconsistencies, classification; combining crowdsourced data, ranking using PageRank, regression
%   \item In low-data situations, these approaches may underperform, since model uncertainty is not accounted for between each stage, nor in the final predictions, and errors propagate along the pipeline
%   \item Training data may also contain errors, which would be propagated through the pipeline (this was avoided in previous work by combining labels from multiple crowdworkers; we should be able to handle the case where this is not possible).
%   \item Feature space becomes very large when working with textual features -- can we narrow it down automatically to improve scalability and improve performance?
% \end{itemize}

In contrast to previous work, we propose the use of preference learning techniques for argument convincingness
to directly model the relationship between crowdsourced preferences and textual features, including word embeddings.
We choose a Bayesian approach, since Bayesian methods have been shown to successfully handle the problem of small(for example, \cite{xiong2011bayesian,titov2012bayesian}) and unreliable datasets (e.g. \cite{simpson2015language}),
and provide a good basis for active selection to reduce labelling costs\cite{mackay1992information}.
Our method is based on the Gaussian process (GP) model of \cite{chu2005preference},
which assumes that preferences over items are described by a latent preference function.
By providing a Bayesian treatment to this latent function, the method handles uncertainty
in the function values due to noise and data sparsity in a principled manner.
GP preference learning (GPPL) has not previously been applied to text problems with large
numbers of features and the inference scheme proposed by \cite{chu2005preference}
was limited by a computational complexity of $\mathcal{O}(N^3)$, where $N$ is the number of items.
We address the problem of scalability by applying recent advances in stochastic variational
inference (SVI) \cite{hoffman2013stochastic}  to this model, and
developing an efficient optimisation technique for key hyper-parameters.
We then show how our method can be applied to argument convincingness 
with a large number of linguistic features and high-dimensional text embeddings.
%   \item Confidence estimates from Bayesian models account for sparsity and noise in data, as well as uncertainty in the model. This means they do not make overly-confident predictions when training data is small (they know when they don't know).
%  \item Bayesian preference learning methods have been proposed but scalable implementations were not developed and models have not been applied to text with large numbers of features
%  \item We address the limitations above by adapting Bayesian preference learning approach to argumentation
%  \item Introduce stochastic variational inference (SVI) to train the model on large numbers of preferences and documents
%  \item Develop gradient-based ARD to identify relevant text features
%\end{itemize}
Our evaluation compares Bayesian preference learning to established SVM and neural network approaches for predicting convincing arguments, and show that our approach can
outperform these alternatives particularly with small and noisy datasets. 
%  \item Show that Bayesian Gaussian process (GP) models are applicable to performing preference learning over text (existing evaluation of GPs for text is very limited, although they have been used extensively with great success in domains such as Physics, finance, Biology. This is possibly because GPs were seen as more difficult to implement and could not be scaled up until recent advances such as SVI)
%  \item Evaluate the ability of each method to handle noisy and sparse data, showing improved performance using our method in the presence of noise and data sparsity
 % \item Analyse the features that are most informative when determining convincingness, providing insight into what makes a convincing argument
%\end{itemize}

The rest of the paper is structured as follows.
First, we review related work in more detail: on argumentation; Bayesian methods for preference learning; and scalable approximate inference.
We then explain the preference learning approach in detail and develop our SVI inference  and hyper-parameter optimisation methods.
The following section details a number of experiments: a comparison with the
state-of-the art on predicting preference in online debates; 
noisy dataset; active learning; and feature relevance determination.
Finally, we present some conclusions and avenues for future work.

\section{Related Work}\label{sec:related}

Recent work on argumentation by \cite{habernal2016argument} has established datasets and methods for
predicting which argument is most convincing. Our experiments make extensive use of this data to establish
a different methodology. This work was also extended to evaluate the reasons why one argument
is more convincing than another\cite{habernal2016makes}, however our paper focusses on prediction when reasons are not given. 
Investigations by \cite{lukin2017argument} demonstrated the effect of personality and prior stance 
of the audience on the persuasiveness of arguments,
although their work does not extend to modelling this persuasiveness using preference learning.
The sequence of arguments in a dialogue is another important factor in their ability to change
the audience's opinions \cite{tan2016winning}.
This idea is used by \cite{rosenfeld2016providing,monteserin2013reinforcement}, 
who address the problem of choosing the best argument in a dialogue between a human user and an agent. 
However, these works focus on applying  reinforcement learning to predict the best argument 
to present in a sequence rather than learning user preferences for arguments with certain qualities.
 
% GP-based approach versus Bradley/Terry/Luce/MPM method or Mallows models?
The goal of preference learning is to predict a ranking over items in terms of preference,
or to predict which single item $x_i$ in a pair or small set would be chosen by the user. 
A preference for item $x_i$ over $x_j$ is written as $x_i \succ x_j$.
Given a ranking over items, it is possible to determine the pairwise preferences,
but pairwise labels can also be predicted using a generic classifier without the need to learn a total ordering.
During training and prediction, pairs of items are transformed either by concatenating the feature vectors of two items as in \cite{habernal2016argument}, 
or computing the difference of the two feature vectors as in SVM-Rank\cite{joachims2002optimizing}. 
The classifier is then trained as normal with preference labels treated as binary class labels.

However, the ranking of items is useful for producing ordered lists in response to a query -- 
consider a sorted list of the most convincing arguments in favour of topic X.
Another approach is to learn this ordering directly using Mallows models\cite{mallows1957non},
which define distributions over permutations of a list. 
Mallows models have been extended to provide a generative model\cite{qin2010new} and 
to be trained from pairwise preferences rather than by observing rankings\cite{lu2011learning}. 
A disadvantage of Mallows models is that inference is typically costly, 
since the number of possible permutations to be considered is $\mathcal{O}(N^2)$, 
where $N$ is the number of items to be ranked. 
Modelling only the order of items means we are unable to quantify 
how closely rated items at similar ranks are to one another: how much better is the top ranked item 
from the second-rated?

% Bradley-Terry: MPM\cite{volkovs_new_2014}
To avoid the problems of classifier-based and permutation-based methods, 
another approach is to learn a set of underlying real-valued scores from pairwise labels.
These scores can then be used to predict rankings, pairwise labels, or ratings for individual items.
To do this, a model is required to map the real-valued scores to discrete pairwise labels.
Two established approaches for this are based on the Bradley-Terry-Plackett-Luce model \cite{bradley1952rank,luce1959possible,plackett1975analysis}
and the Thurstone-Mosteller model\cite{thurstone1927law,mosteller2006remarks}.
In more recent work, Bayesian extensions of the Bradley-Terry-Plackett-Luce model
were proposed by \cite{guiver2009bayesian,volkovs_new_2014}, 
while the Thurstone-Mosteller model was used by \cite{chu2005preference}.
This latter piece of work assumes a Gaussian process (GP) prior over the scores,
which enables us to predict scores for previously unseen items given their features 
using a Bayesian nonparametric approach.
Gaussian processes have been well established as effective and versatile models that
extrapolate from training data in a principled manner, taking into account model uncertainty
\cite{rasmussen_gaussian_2006}.
Their nonparametric nature means that the function complexity can grow with the amount of data observed.
These characteristics make them suitable for the task of modelling argument convincingness
where data for new topics, domains and users is limited. 
% The Gaussian process (GP) preference learning approach of \cite{chu2005preference} resolves inconsistencies between preferences and provides a way to predict rankings or preferences for 
% items for which we have not observed any pairwise comparisons based on the item's features. 
% This model assumes that preferences are noisy, i.e. contain some erroneous labels.
% particularly as the modular nature of inference algorithms such as Gibb's sampling and variational approximation is suited to extending the model to handle different types of feedback that give indications of some underlying preferences. 

The method developed by \cite{chu2005preference} used the Laplace approximation to perform approximate
inference over the model. Unfortunately the memory and computational costs scale with $\mathcal{O}(N^3)$
due to matrix inversion. If this limitation is overcome, there is still a computational and memory cost 
during training of $\mathcal{O}(N^2)$ due to the number of pairs in the training dataset.
Such problems are common when performing inference over Gaussian process models but have been addressed
by \cite{hensman2013gaussian,hensman_scalable_2015} for regression and classification tasks using
the stochastic variational inference (SVI) algorithm proposed by \cite{hoffman2013stochastic}. 
SVI has, however, not previously been adapted for preference learning with GPs.
% The modular nature of VB allows us to take advantage of models for feedback of different types
% where the input values for each type of feedback do not directly correspond (e.g. explicit user ratings and number of clicks may have different values).
% By using SVI, we provide a formal way to deal with scalability that comes with guarantees\cite{hoffman2013stochastic}.
The next section explains our preference learning method for argument convincingness.

%GPs for NLP in other task areas?

\section{Related Work}
\label{sec:rw}

% Don't think the references really show this, or should necessarily be added: Pairwise comparisons have low cognitive load:
% doesn't show this -- [1] Urszula Chajewska, Daphne Koller, and Ronald Parr. Making rational decisions using adaptive
% utility elicitation. In Proceedings of the Seventeenth National Conference on Artificial Intel-
% ligence and Twelfth Conference on Innovative Applications of Artificial Intelligence, pages
% 363–369. AAAI Press / The MIT Press, 2000.
% [2] Vincent Conitzer. Eliciting single-peaked preferences using comparison queries. Journal of
% Artificial Intelligence Research, 35:161–191, 2009. --> provides algorithms for aggregating rankings but does not consider uncertainty,
%user features...

% Done -- citation is in the intro. work from Fürnkranz should be cited (he may be a reviewer).
% \cite{hullermeier2008label} -- cite this in the intro. Preference learning through
% pairwise comparison but used to rank labels in a multiclass situation, rather than ranking items.
% \cite{furnkranz2009binary},\cite{furnkranz2010preference} -- a simple function for combining preferences generated by comparing pairs of labels assigned to items. E.g. if I can give movies scores of 1 to 5, this method combines the scores of pairs of items.
% For object ranking approaches, this idea has first been formalized by Tesauro [58] under the name comparison training.
% He proposed a symmetric neural-network architecture that can be trained with representations of two states and a training
% signal that indicates which of the two states is preferable. The elegance of this approach comes from the property that
% one can replace the two symmetric components of the network with a single network, which can subsequently provide a
% real-valued evaluation of single states.

% TODO: [18] T. Salimans, U. Paquet, and T. Graepel. Collabo-
% rative learning of preference rankings. In Proceed-
% ings of the 6th ACM Conference on Recommender
% Systems, 2012.

% Note that Khan's method does not need factorization assumptions in the approximate posterior.
% Instead, they have no prior over v (item features).
% They have a diagonal covariance for the user features -- cheap. 
% They need a separate GP per user but it seems like this is not a problem in practice -- I guess
% the method scales linearly with no. users. In our case, we model covariance between users, so
% scaling is poor unless you can use inducing points or diagonal covariance.

\subsection{Aggregating Pairwise Labels from a Crowd} % Learning a Consensus by ...

% TODO: do we really care about multiple rankings in this paragraph?

To obtain a ranking from pairwise labels, many preference learning methods model
the user's choices as a random function of the latent utility of the items~\citep{thurstone1927law}.
An example is the method of ~\citet{herbrich2007trueskill}, which learns the skill of chess players from 
match outcomes by treating them as noisy pairwise labels.
Recent work on this type of approach has analyzed bounds on error rates ~\citep{chen2015spectral}, 
sample complexity~\citep{shah2015estimation}, and joint models for ranking and clustering from pairwise comparisons~\citep{li2018simultaneous}.
% When all items have a sufficient number of labels, it is possible to estimate the underlying utility using
% simple counting procedures~\citep{kiritchenko2016capturing}.
%, but do not propose methods
%for learning multiple rankings from crowds of users.
% Most approaches use Bradley-Terry or Thurstone. However some also try Mallows models ~\cite
% {busa2014preference,raman2015bayesian} to get the uncertainty over the ranking instead of over a 
% latent score.
% Other approaches use graph-based ranking measures, e.g. based on Kleinberg's HITS ~\citep{sunahase2017pairwise} or PageRank.
% extension of chenc2013 to k-ary preferences. han2018robust
% wang2016blind/Thurstonian Pairwise Preference (TPP): Chen et al. lacks the mechanism to model multiple query domains, (what does this mean? --> better at labeling certain types of items)
%thus incapable to characterize workers’ domain-dependent expertise and truthfulness. 
% CROWD BT does not take query difficulty into account either. (do they have a feature-dependent model?)
% Furthermore, unlike TPP, CROWD BT does not model the generation of rankings (it models generation of pairwise labels directly)
% Therefore, it simplifies the generation of inconsistent annotations as solely a result from worker accuracy
% (it treats differences between workers as pure noise).

The problem of disagreement between annotators in a crowd was addressed by \citet{chen2013pairwise}
\todo{
How do we make it clear that this is not the case? --
This paper extended the pairwise ranking model in Xi Chen's work, Pairwise Ranking Aggregation in a Crowdsourced Setting with Gaussian Process.
The preference learning has well-studied. Particularly, the pairwise label using BT model has been published by Xi Chen's work. Recently, the model has been extended to more general PL model, such as:
Han et al., Robust Plackett-Luce model for k-ary crowdsourced preferences. Machine Learning 107(4): 675-702 (2018);
Pan et al., Stagewise learning for noisy k-ary preferences. Machine Learning 107(8-10): 1333-1361 (2018).
[these two seem to extend the crowd-BT model, so are mainly about de-noising.]
Hybrid-MST: A Hybrid Active Sampling Strategy forPairwise Preference Aggregation --
doesn't consider different workers at all, uses only ML learning [make better case
for Bayesian?]. The active learning strategy they devise based on minimum spanning trees
could also be used with crowdGPPL or crowdBT. Using a Bayesian method gives
us a distribution over utilities, which means we can do BALD to estimate EIG much
quicker. Their method aims at reducing the complexity of selecting pairs for active learning, it doesn't address complexity of learning a feature-based model from pairs.
Use sushi-A-small to show that crowd-BT and others don't deal with small data well
if they don't use feature information.
We should also discuss the learning strategies used in these methods. They are not aimed
at GPs and matrix factorisation, unlike our model, but could they still be applied here?
If so, this should be discussed in future work.
Finally, there may be a nice dataset from one of these that we could use? 
Important distinction is that we look for subjective tasks, not just denoising the
pairwise labels. This might need to be justified better by describing tasks
such as argument mining. Consensus in a subjective task is helpful to learn the personal 
preferences of new users -- do we show this or describe it? Consensus can also be useful
if we need to extract a gold standard or objectively correct rating, but annotators
have different biases, rather than just noise levels. What about settlement rating for
planet hunters? -- turn into pairs. Some users may ignore certain types of building,
so images with those features would be biased to low ratings for those people. 
Biases like this could harm performance if there aren't enough preferences from 
different users to cancel them out.
Can we demonstrate this, or just argue it in the intro?
Yes, find three argument pairs where the crowdGPPL and ground truth agree, but the 
annotators do not; show the annotator bias according to crowdGPPL, i.e. weights for the 
latent features.  
]
Those models are more general and robust to the noisy preferences in crowdsourcing settings.
In terms of robustness of noise modeling, I do not find the proposed model can be significantly better than the previous works.  There is no experiments comparing with any baselines in crowdsourcing, even Chen's work. 
}
and \citet{wang2016blind},
using Bayesian approaches that learn the individual accuracy of each worker.
%These methods do not learn the relationship between the workers' individual preferences
%and the ground truth. 
%\citet{wang2016blind} improved performance 
%by modeling the level of noise in the latent utility function for each annotator in a given domain, rather than
%in the pairwise labels. 
However, they do not %\citet{chen2013pairwise} nor \citet{wang2016blind}
exploit item features to mitigate data sparsity.
In contrast, Gaussian processes preference learning (\emph{GPPL})
uses item features to make predictions for unseen items and
share information between similar items~\citep{chu2005preference}.
GPPL was used by \citet{simpson2018finding} to aggregate crowdsourced pairwise labels,
but assumes the same level of noise for all annotators and ignores the effect of divergent opinions.
\todo{
Moreover, please note that the call for papers for ECMLPKDD journal
track explicitly states the following: "Consequently, journal versions
of previously published conference papers, or survey papers will not
be considered for the special issue." This is in contrast to "regular"
journal submissions, which can be extended versions of previous
conference papers. The earlier publication you have on this topic was
presented at ACL 2018. We would thus like you to make sure that the
revision puts an even stronger emphasises on the new results, to make
sure that the manuscript solidly goes beyond an "extended conference
paper".
}
To crowdsource sentiment annotations, 
\citet{kiritchenko2016capturing} propose to use an extension of pairwise comparison
known as \emph{best-worst scaling}, in which the annotator selects best and worst items from a set.
They apply a simple counting technique to infer a ranking over the items, which requires 
each item to have a sufficient number of comparisons.
% account for the varying quality of pairwise labels obtained from a crowd
%by learning an individual model of agreement with the true pairwise labels for each worker.  

%  consider the case where different rankings correspond to lists of items
% provided in response to search queries. 
% While they model the dependence of annotator accuracy on the domain of a query,
% their approach was not applied to personal or subjective rankings.

%Say we want to learn a 'ground truth' preference function that may be one user's preference function.
% One set of pairwise labels may be informative for one subset of items.
% E.g. music recommendation, two users may have similar jazz preferences but all other genres are different
% E.g. learning user preferences from webpage clicks, selecting items from a list may be informative in one context and meaningless in another
A popular method for predicting pairwise labels of new items given their features is 
\emph{SVM-rank}~\citep{joachims2002optimizing}.
%predicts only pairwise labels rather than the underlying utilities.
For crowdsourced data, \citet{fu2016robust} show that performance is improved by identifying outliers in crowdsourced data
that correspond to probable errors.
\citet{uchida2017entity} extend SVM-rank to account for different levels of confidence in each pairwise annotation expressed
by the annotators.
%However, these approaches do not model divergence of opinion between annotators
%and do not provide a Bayesian solution.
%Related works have also investigated budget constraints for crowdsourcing pairwise labels~\citep{cai2017pairwise}.
% Also consider mentioning relevant work on active learning
% -- finding the most preferred item with minimal labels
% -- learning a preference function using AL
% -- learning within a budget constraint
% Strategy: look at recent works from ML/AI/DM conferences. Must consider annotators with different preferences.
% A number of studies consider actively selecting pairs of items for
% comparison to minimize the number of pairwise labels required~\citep{radlinski2007active,qian2015learning,maystre2017just,cai2017pairwise}.
However, besides modeling labeling noise, 
the previous work on crowdsourced preference learning does not 
account for the effect of divergence of opinion on the inferred preferences.
% does not 
% provide a Bayesian approach for aggregating pairwise labels from crowds
% that can make predictions for new items
% and model the divergence of opinions between annotators.

\subsection{Bayesian Methods for Inferring Individual Preferences}

% \citet{tian2012learning} consider crowdsourcing tasks where there may be more than one correct answer.
% They use a nonparametric Dirichlet process model to infer a variable number of clusters of answers for each task,
% and also infer annotator reliability. 
% However, they do not apply the approach to ranking using pairwise labels.
% actively selecting pairs for annotation as a multi-armed bandit problem~\citep{busa2018preference}.
%  do not study the process of learning from an oracle or user that we can query.
% Rather, we develop a model for aggregating pairwise labels from multiple sources, 
% which can be used as the basis of active learning methods that exploit the model uncertainty 
% estimates provided by this Bayesian approach.
%
%queries belong to 'domains'. Annotators have different accuracy on each domain. 
% we don't do that because we assume each annotator has their own ranking and so different noise levels are introduced through the personalized preference function having larger values where the annotator is confident. 
\todo{ Section 2.2: Define or give an example of 'input features' when introduced.  }
As well as aggregating preferences from a crowd to identify a consensus,
we also wish to predict the preferences of individual users.
\citet{yi_inferring_2013} and \citet{kim2014latent} address this task by learning
 multiple latent rankings and inferring
the preference of each user toward those rankings, while 
\citet{salimans2012collaborative} use Bayesian matrix factorization to identify multiple
latent rankings.
However, none of these approaches exploit item features to remedy labeling errors or generalize to new test items.
%crowdranking \citet{yi_inferring_2013} uses the crowd to make up for the fact that a target user has small data -- it's a form of collaborative filtering with pairwise labels using a non-Bayesian inference algorithm.
% Several other works learn multiple rankings from crowdsourced pairwise labels
% rather than a single gold-standard ranking, 
% but do not consider the item or user features so cannot extrapolate to new users or 
% items~\citep{yi_inferring_2013,kim2014latent}. 
% Both \citet{yi_inferring_2013} and learn a small number of
% latent ranking functions that can be combined to construct personalized preferences, 
% although neither provide a Bayesian treatment to handle data sparsity.
%include the work on collaborative GPPL
%Several extensions of BMF use Gaussian process priors over latent factors 
%to model correlations between 
%items given side information or observed item features~\citep{adams2010incorporating,zhou2012kernelized,houlsby2012collaborative,bolgar2016bayesian}. 
%However, these techniques are not directly applicable to 
%learning from pairwise comparisons 
%as they assume that the observations are Gaussian-distributed numerical ratings~\citep{shi2017survey}. 
In contrast, \citet{guo2010gaussian} propose a joint Gaussian process over the
space of users and features. Since this scales cubically
in the number of users, \citet{abbasnejad2013learning} 
propose to cluster the users into behavioural groups.
However, distinct clusters do not
allow for collaborative learning between users with partially overlapping preferences, e.g. two users may both like one genre of music, 
while having different preferences over other genres. 
\citet{khan2014scalable} instead learn a GP for each user,
and combine them with matrix factorization to perform collaborative filtering.
However, this approach does not model the relationship between
 input features and the latent factors, does not place a prior over item factors,
 and does not scale to very large sets of users.
An alternative is \emph{Collaborative GPPL}~\citep{houlsby2012collaborative},
which uses a latent factor model, where each latent factor has a Gaussian process prior. 
This allows the model to take advantage of the input features of
users and items when learning the latent factors. 
Each individual's preferences are then represented 
by a mixture of latent functions.
%Pairwise labels from users with common interests help to predict each other's 
%preference function, hence 
%this can be seen as a collaborative learning method, as used in recommender systems.
Using matrix factorization in combination with GP priors is therefore an effective
way to model the individual preferences of users while
making use of input features to generalize to test cases. However,
none of these approaches considers a consensus preference function, and
more scalable inference is needed for datasets containing thousands of items or users.
%other Bayesian recommender systems that deal with noisy preferences
%To combine Bayesian matrix factorization with a pairwise likelihood,
%\citet{houlsby2012collaborative} propose
%However, their proposed method does not scale sufficiently to the numbers of 
%items, users or pairwise labels found in many important application domains. 
% previous work on how to make BMF more scalable with larger datasets or Latent factor analysis etc.
% who has used GPs for BMF?

 % PCA: Gaussian noise. "The classical PCA converts a set of samples with possibly correlated variables into another   
 % set of samples with linearly uncorrelated variables via an orthogonal transformation [1]. Based on this, PCA
 % is an effective technique widely used in performing dimensionality reduction and extracting features." -- Shi et al 2017. shi2017survey
 % SVD: like PCA with the mean vector set to zeroes.
 % variations of PCA: for handling outliers or large sparse errors
 % most matrix factorizations are special cases of PCA and in practice do not consider the mean vector.
 % probabilistic PCA: latent variables are unit isotropic Gaussians --> all have 0 covariance and 1 variance.
 % Bayesian PCA: places priors on all latent variables.
 % Probabilistic factor analysis: assumes different variances on each of the latent factors.
 % Probabilistic matrix factorisation: ignores the mean. --> I.e. can be done with SVD
 % I think this means our method is a form of PFA? But extended to consider correlations in the weights.
 % NMF: as matrix factorisation but the low-rank matrices are non-negative.

% of their hybrid inference method expectation propagation and variational Bayes limits its application in many domains.
% They use FITC to provide a sparse approximation. This is still not as scalable as SVI and doesn't work as well --
% see the Hensman papers?
%with Gaussian processes to improve performance with sparse data. 
%Their model can be learned using pairwise labels, numerical ratings, or other likelihoods.

%user features nor model dependencies between item features 
%and the low-dimensional latent features, so it cannot exploit the latent features to predict preference scores for new items and relies instead on a user-specific GP.
%In contrast, our approach does not require learning a separate GP per user, but instead
%places GPs on both the latent factors. This means that the item and user
%features assist in learning the latent factors as we can exploit their similarities and
%correlations.
%relying instead on a purely individual GP with no shared information,
%(this would be a problem if the user fits the latent features exactly as the GP will end up modelling the user's individual devaiation from the common preferences modeled by the latent features) 
%To achieve scalability using a variational EM algorithm, \citet{khan2014scalable}
%sub-sample the pairwise labels meaning that 
%some training data must be discarded. In this paper, we
%applies stochastic variational inference to learn from all data 
%while limiting memory and computational requirements.
% how do we compare to them or do we get out of it? --> compare results on the same datasets
%Our approach is similar to Khan et al. 2014. "Scalable Collaborative Bayesian Preference Learning" but differs in that
%we also place priors over the weights and model correlations between different items and between different people.
%Our use of priors also encourage sparseness in the features. 
%TODOs:
% what is meant by 'factorization assumptions' exactly and do we make them? I think we do but don't fully
% understand why they're so bad. See [18,11] from Khan for examples of bad factorization. 

\subsection{Scalable Approximate Bayesian Inference}

\todo{
Another claim from the authors is the SVI for large-scale crowdGPPL, which is due to limitation of GP. Indeed, the online strategy has been well-studied in crowdBT,  crowdPL,  the previous works and the following work. 
Li et al,. Hybrid-MST: A Hybrid Active Sampling Strategy for Pairwise Preference Aggregation, NIPS 2018
Therefore, I don't find any new insight of SVI here.
}
Recent work on scalable Bayesian matrix factorization focuses on distributing and parallelizing 
 inference %rather than reducing total costs, 
 but is not directly applicable to Gaussian processes~\citep{ahn2015large,vander2017distributed,chen2018large}. 
%This paper focuses instead on reducing computational and memory costs, 
%although the method we propose is amenable to parallelization.
Models that combine Gaussian processes with non-Gaussian likelihoods 
require approximate inference methods that often scale poorly with 
the amount of training data available. 
Established methods such as the Laplace approximation 
and expectation propagation~\citep{rasmussen_gaussian_2006} have
computational complexity $\mathcal{O}(N^3)$ with $N$ data points
 and memory complexity $\mathcal{O}(N^2)$. 
For collaborative GPPL, \citet{houlsby2012collaborative}
propose a  kernel for pairwise 
preference learning and use a sparse
\emph{generalized fully independent training conditional} (GFITC) 
approximation~\citep{snelson2006sparse} to reduce the computational complexity to $\mathcal{O}(PM^2 + UM^2)$ and 
memory complexity to $\mathcal{O}(PM + UM)$,
where $P$ is the number of pairwise labels, $M \ll P$ is a fixed number of inducing points, and $U$ is the number of users.
However, this is not sufficiently scalable
 for very large numbers of items, users or pairs, as there is no clear way to parallelize it or to limit memory consumption.
%which assumes that
%the observations at each training data point are independent of one another given the 
%latent values at the inducing points.
%Since $P$ is typically larger than $N$, the number of inducing points, $M$ may also need to be increased.
The GP over pairs also makes it difficult to extract posteriors for latent function values for individual items,
and prevents mixing pairwise training labels
with observed ratings in future data aggregation settings.

To handle large numbers of pairwise labels, \citet{khan2014scalable}
develop a variational EM algorithm and sub-sample pairwise data rather than learning from the complete training set.
An alternative is \emph{Stochastic variational inference (SVI)}~\citep{hoffman2013stochastic}, which updates an approximation using 
 a different random sample at each iteration. 
 This allows the approximation to make use of all training data over a number of 
 iterations, while limiting training costs per iteration.
SVI has been successfully applied to Gaussian process regression~\citep{hensman2013gaussian} and classification~\citep{hensman2015scalable},
and provides a convenient framework for sparse approximation.
 An SVI method was also developed for preference learning,
 which places a GP over items rather than pairs, but does not model multiple users' preferences~\citep{simpson2018finding}.
This paper provides the first full derivation of this approach, %including showing how to learn the observation noise
%as part of the variational approach. 
as well as providing the first application of SVI to Bayesian matrix factorization
as part of a new model that accounts for individual user preferences.
%developing the first application of SVI to a matrix factorization approach.
%As our method includes Bayesian matrix factorization (BMF) as part of the model, 
%we believe this paper is the first to apply SVI to BMF.

\section{Bayesian Preference Learning for Crowds}\label{sec:model}

%%%% Notes

% Title or name of the model: 
% -- cannot decide this until we get most of the paper complete: will emphasis be on crowds? distilling ground truth
% from noisy sources (Bayesian preference learning for fusing unreliable sources)? user preferences/collaborative filtering?
% -- need a new name to differentiate from Houlsby et al. and Khan et al.?
% -- what are the differences in the model? Let's get the model written up.
% -- should also be some buzzword or word to generate interest: 
%    -- 'variational' is on the up, could be used in paper title
%    -- 'stochastic variational' is also on the up
%    -- 'crowdsourcing' on the way down as at 2010 level
%    -- 'gaussian process' on the way up
%    -- 'matrix factorization' kind of on the way up
%    -- 'scalable' on the way up
%    -- 'interactive learning' on the way down
%    -- 'preference learning' flattish, may be on way up slowly
% -- aimed at crowdsourcing problems (uses a common mean function as consensus?)
% -- other parameters for importance of features?
% -- combined preference learning? Preference aggregation? Collaborative crowdsourced preference learning? Bayesian preference learning for crowds? another word for 'multi-user' or 'many users and many items' vs. crowds?

% TO ADD: Why does the variance in f cancel out when predicting the probability of a pairwise label?
% TO ADD: Why does \sigma disappear if we learn the output scale.

% We also estimate the output scale of the GPs, the latent components, and item bias as part of the 
% variational approximation allowing us to estimate these parameters in a Bayesian manner without 
% resorting to maximum likelihood approaches.

% mention how the noise model deals with inconsistencies in preferences

% \begin{enumerate}
% \item What are the benefits of Bayesian methods and Gaussian processes in particular?
% \item The proposal by \citep{chu2005preference} shows how the advantages of a Bayesian
% approach can be exploited for preference learning by modifying the observation model

% Extensions:
% -- how do we replace the GP with a NN?
% -- would this move us from a Bayesian to an ML solution?
% -- maybe save this for future work? Or add a few lines if we can make it fit with the theme of the paper.
% -- another is to replace the fixed number of clusters with a CRP, then the whole thing can be nonparameteric preference learning with crowds.

We assume that a pair of items, $a$ and $b$, have utilities
$f(\bs x_a)$ and $f(\bs x_b)$, which represent their value to a user,
and that $f: \mathbb{R}^D \mapsto \mathbb{R}$ 
is a function of item features, where $\bs x_a$ and $\bs x_b$ are vectors 
of length $D$ containing the features of items $a$ and $b$, respectively.
If $f(\bs x_a) > f(\bs x_b)$, then $a$ is preferred to $b$ (written $a \succ b$).
The outcome of a comparison between $a$ and $b$ is 
a pairwise label, $y(a, b)$.
Assuming that pairwise labels never contain errors,
then $y(a, b)=1$ if $a \succ b$ and $0$ otherwise.
Given knowledge of $f$, we can compute the utilities 
of items in a test set given their features, and the outcomes of pairwise comparisons.

\citet{thurstone1927law} proposed the random utility model,
which relaxes the assumption that pairwise labels, $y(a, b)$,
are always consistent with the ordering of $f(\bs x_a)$ and $f(\bs x_b)$.
Under the random utility model, the likelihood $p(y(a,b)=1)$ 
increases as $f_a - f_b$ increases, i.e.,
as the utility of item $a$ increases
relative to the utility of item $b$.
This reflects the greater consistency in a user's choices
when their preferences are stronger,
while accommodating
%However, since $0 < p(y(a,b)=1) < 1$, the model 
%is uncertain about the value of $y(a,b)$,
labelling errors or variations in a user's choices over time.
%The uncertainty is lower if the values $f_a$ and $f_b$ are further apart, 
%which 
%The random utility model is defined by a likelihood function that
%maps the utilities to $p(y(a,b))$.
%or the Thurstone-Mosteller model.
%\begin{align}
%p(y(a, b) | f) & = \frac{1}{1 + \exp( f(\bs x_a) - f(\bs x_b) ) }
%\end{align}
In the Thurstone-Mosteller model, % case V model, 
noise in the observations is explained by a Gaussian-distributed noise term, $\delta \sim \mathcal{N}(0, \sigma^2)$:
\begin{flalign}
 p(y(a, b) | f(\bs x_a) + \delta_{a}, f(\bs x_b) + \delta_{b} )  
 \hspace{0.9cm} & = \begin{cases}
 1 & \text{if }f(\bs x_a) + \delta_{a} \geq f(b) + \delta_{b} \\
 0 & \text{otherwise,}
 \end{cases} &
 \label{eq:thurstone}
\end{flalign}
Integrating out the unknown values of $\delta_a$ and $\delta_b$ gives:
\begin{flalign}
& p( y(a, b) | f(\bs x_a), f(\bs x_b) )  & \label{eq:plphi}\\
& = \!\! \int\!\!\!\! \int \!\! p( y(a, b) | f(\bs x_a) + \delta_{a}, f(\bs x_b) + \delta_{b} ) \mathcal{N}(\delta_{a}; 0, \sigma^2)\mathcal{N}(\delta_{b}; 0, \sigma^2) d\delta_{a} d\delta_{b} 
%= \Phi\left(\frac{f(\bs x_a) - f(\bs x_b)}{\sqrt{2\sigma^2}}\right) 
= \Phi\left( z \right), & \nonumber
\end{flalign}
where $z = \frac{f(\bs x_a) - f(\bs x_b)}{\sqrt{2\sigma^2}}$,
and $\Phi$ is the cumulative distribution function of the standard normal distribution,
meaning that $\Phi(z)$ is a 
probit likelihood.\footnote{Please note that a full list of symbols is provided for reference in Appendix $\ref{sec:not}$}

In practice, $f(\bs x_a)$ and $f(\bs x_b)$ must be inferred from
pairwise training labels, $\bs y$,
to obtain a posterior distribution over their values.
If this posterior is a multivariate Gaussian distribution,
then the probit likelihood allows us to analytically marginalise 
$f(\bs x_a)$ and $f(\bs x_b)$
to obtain the probability of a pairwise label:
\begin{align}
p(y(a,b)| \bs y) 
= \Phi(\hat{z}),& & \hat{z} = \frac{\hat{f}_a - \hat{f}_b}{\sqrt{2\sigma^2 + C_{a,a} + C_{b,b} 
- 2C_{a,b}} }, \label{eq:predict_z}
\end{align}
where $\hat{f}_a$ and $\hat{f}_b$ are the means and
$\bs C$ is the posterior covariance matrix of the multivariate Gaussian over
$f(\bs x_a)$ and $f(\bs x_b)$.
Unlike other choices for the likelihood, such as a sigmoid,
the probit allows us to compute the posterior over a pairwise label
without further approximation, %numerical integration
hence we assume this pairwise label likelihood for our proposed preference learning model.
This likelihood is also used by
~\citet{chu2005preference} for Gaussian process preference learning (GPPL), but here 
we simplify the formulation by assuming that $\sigma^2 = 0.5$,
which leads to $z$ having a denominator of $\sqrt{2 \times 0.5}=1$.
Instead, we model varying degrees of noise in the pairwise labels
by scaling $f$ itself, as we describe in the next section.
%Obtaining the posterior over $f$ is itself challenging, however, 
%and therefore in Section $\ref{sec:inf}$ we propose 
%an approximate inference method to address this problem.


\subsection{GPPL for Single User Preference Learning}

We can model the preferences of a single user by assuming
a Gaussian process prior over the user's utility function, 
%is a function of item features and 
$f \sim \mathcal{GP}(0, k_{\theta}/s)$, where $k_{\theta}$ is a kernel function with hyperparameters $\theta$
and $s$ is an inverse scale parameter.
The kernel function takes numerical item features as inputs and determines the covariance between values of $f$ for different items. 
The choice of kernel function and its hyperparameters controls the shape and smoothness of the function 
across the feature space and is often treated as a model selection problem.
Kernel functions suitable for a wide range of tasks include the \emph{squared exponential} 
and the \emph{Mat\'ern}~\citep{rasmussen_gaussian_2006},
which both make minimal assumptions but 
assign higher covariance to items with similar feature values.
We use $k_{\theta}$ to compute a covariance matrix $\bs K_{\theta}$,
between a set of $N$ observed items with features $\bs X = \{ \bs x_1, ..., \bs x_N \}$.

Here we extend the original definition of GPPL~\citep{chu2005preference},
by introducing the inverse scale, $s$,
which is drawn from a gamma prior, 
$s \sim \mathcal{G}(\alpha_0, \beta_0)$, with shape $\alpha_0$ and scale $\beta_0$.
The value of $1/s$ determines the variance of $f$,
and therefore 
the magnitude of differences between $f(\bs x_a)$ and $f(\bs x_b)$ for
items $a$ and $b$. This in turn affects the level of certainty
in the pairwise label likelihood as per Equation \ref{eq:plphi}.

Given a set of $P$ pairwise labels, %for a single user, 
$\bs y=\{y_1,...,y_P\}$,
where %the $p$th label, 
$y_p=y(a_p, b_p)$ is the preference label for items $a_p$ and $b_p$, % refers to items $\{ a_p, b_p \}$.
we can write the joint distribution over all variables as follows:
\begin{flalign}
p\left( \bs{y}, \bs f, s | k_{\theta}, \bs X, \alpha_0, \beta_0 \right) 
=  \prod_{p=1}^P p( y_p | \bs f ) 
\mathcal{N}(\bs f; \bs 0, \bs K_{\theta}/s) \mathcal{G}(s; \alpha_0, \beta_0) %\nonumber \\
%=  \prod_{p=1}^P \Phi\left( z_p \right) 
%\mathcal{N}(\bs f; \bs 0, \bs K_{\theta}/s) \mathcal{G}(s; \alpha_0, \beta_0), &
\label{eq:joint_single}
\end{flalign}
where 
$\bs f = \{f(\bs {x}_1),...,f(\bs {x}_N)\}$
is a vector containing the utilities of the $N$ items referred to by $\bs y$,
and $p( y_p | \bs f ) = \Phi\left( z_p \right)$ is the pairwise likelihood (Equation \ref{eq:plphi}). 
%We henceforth refer to this model simply as \emph{GPPL}.

\subsection{Crowd Preference Learning} \label{sec:crowd_model}

To predict the preferences of individuals in a crowd,
we could use an independent GPPL model for each user.
However, by modelling all users jointly, we can
exploit correlations between their interests
to improve predictions when preference data is sparse,
and reduce the memory cost of storing separate models.
Correlations between users 
can arise from common interests over certain subsets of items,
such as in one particular genre in a book recommendation task.
Identifying such correlations helps to predict 
 preferences from  fewer observations and is the core idea of collaborative filtering~\citep{resnick1997recommender} and matrix factorisation~\citep{koren2009matrix}.

As well as individual preferences, 
we wish to predict the consensus by aggregating
preference labels from multiple users. 
Individual biases of different users may affect consensus predictions,
particularly when data for certain items comes from a small subset of users.
The consensus could also help
predict preferences of users with little or no data
 by favouring items popular items
and avoiding generally poor items.
We therefore propose 
%address this problem by proposing
 \emph{crowdGPPL}, which jointly models 
the preferences of individual users as well as the underlying consensus of the crowd.
Unlike previous methods for inferring the consensus, 
such as \emph{CrowdBT}~\citep{chen2013pairwise}, we do not treat differences between users as simply the result of labelling errors, 
but also account for their subjective biases
towards particular items. 
 
% is there a better word than 'label sources' for the different sources of implicit feedback or other types of labeling?
%In a scenario with multiple users or label sources, 
For crowdGPPL, 
we represent utilities in a matrix, $\bs{F} \in \mathbb{R}^{N \times U}$,
with %$N$ rows corresponding to items and 
$U$ columns corresponding to users. %, and entries are preference values.
We assume that $\bs{F} = \bs{V}^T \bs{W} + \bs{t}$
 is the product of two low-rank matrices
plus a vector of consensus utilities, $\bs{t} \in \mathbb{R}^N$, 
where $\bs{W} \in \mathbb{R}^{C \times U}$ is a latent representation
of the users,
$\bs{V} \in \mathbb{R}^{C \times N}$ is a latent representation of the items,
and $C$ is the number of latent \emph{components}, i.e., the dimension
of the latent representations.
The column $\bs v_{.,a}$ of $\bs V$, and the column $\bs w_{.,j}$ of $\bs W$,
 are latent vector representations of item $a$ and user $j$,
 respectively.
%Users with similar values for a certain feature will have similar preferences for 
%the subset of items with corresponding feature values. 
Each row of $\bs V$, $\bs v_c=\{ 
v_c(\bs{x}_1),...,v_c(\bs{x}_N)\}$,  
contains evaluations of a latent function, 
$v_c\sim \mathcal{GP}(\bs 0, k_{\theta} /s^{(v)}_c)$,
of item features, $\bs x_a$,
where $k$ is a kernel function, $s^{(v)}_c$ is an inverse function scale,
and $\theta$ are kernel hyperparameters.
%Since our goal is to infer a consensus from a crowd as well as to model individual 
%users' preferences, 
The consensus utilities, $\bs t = \{t(\bs {x}_1),...,t(\bs {x}_N)\}$,
are values of a consensus utility function over item features,
$t\sim \mathcal{GP}(\bs 0, k_{\theta} /s^{(t)})$, which is shared across all users,
with inverse scale $s^{(t)}$.
Similarly, each row of $\bs W$, 
$\bs w_c=\{w_c(\bs u_1),...,w_c(\bs u_U)\}$,
 contains evaluations of a latent function,
$w_c \sim \mathcal{GP}(\bs 0, k_{\eta}/s_c^{(w)})$,
of user features, $\bs u_j$, 
with inverse scale $s_c^{(w)}$
and kernel hyperparameters $\eta$.
Therefore, each utility in $\bs F$ can be written as
a weighted sum over the latent components:
\begin{flalign}
  f(\bs x_a, \bs u_j) = \sum_{c=1}^C  v_c(\bs x_a) w_c(\bs u_j) + t(\bs x_a),
  \label{eq:vw_plus_t}
\end{flalign}
where $\bs u_j$ are the features of user $j$ and $\bs x_a$ are the features of item $a$.
Each latent component corresponds to a utility function 
for certain items, which is shared by a subset of users to differing degrees.
For example, in the case of book recommendation,
$c$ could relate to science fiction novels, 
$v_c$ to a ranking over them,
and $w_c$ to the degree of agreement of users with that ranking.
%CrowdGPPL therefore combines latent features of items and
%users -- represented by the latent components -- with the
%utilities of the items according to an underlying consensus across users.
%Given the consensus, $t$,
%utility for item $a$, $t(\bs x_a)$,
The individual preferences of each user $j$ deviate from a consensus across users, $t$, according
to $\sum_{c=1}^C  v_c(\bs x_a) w_c(\bs u_j)$. 
This allows us to subtract the effect of individual biases when inferring the consensus utilities. 
The consensus can also help 
when inferring personal preferences for %new users, 
%new items or 
new combinations of users and items that are
very different to those in the training data by
 accounting for any objective or widespread appeal that an item may have.
%We provide a Bayesian treatment to matrix factorization by placing Gaussian process priors over the latent functions.
%differently for each user and item. For example, the observed user feature 'age' may correlate with some latent interests of users, but certain users will deviate from their peer group. 
% what happens if two users have identical features (say, the feature representation
% has only simple values, such as age in years)? They have 1-1 covariance, but there 
% is variance in the GP at one location, so both can be drawn separately from the prior.
%It is not necessary to learn a separate scale for $w_c$, since $v_c$ and $w_c$ are multiplied with each other, making a single $s^{(v)}_c$ equivalent to the product of two separate scales. 
%The choice of $C$ can be treated as a hyperparameter, or modeled using a non-parametric prior, such as 
%the Indian Buffet Process, which assumes an infinite number of latent components ~\citep{ding2010nonparametric}.
%This section described a Bayesian matrix factorization model, 
%which we will subsequently extend to a preference learning model for crowds of users and label sources. 
% joint distribution
% notes about problems with inference.

Although the model assumes a fixed number of components, $C$,
the GP priors over $\bs w_c$ and $\bs v_c$ act as \emph{shrinkage}
or \emph{ARD priors} that favour values close to zero~\citep{mackay1995probable,psorakis2011overlapping}. 
Components that are not required to explain the data will have posterior
expectations and scales $1/s^{(v)}$ and $1/s^{(w)}$ approaching zero.
Therefore, %due to our choice of prior, 
it is not necessary to optimise the value of $C$, providing a sufficiently large number is chosen. 

Equation \ref{eq:vw_plus_t} is similar to
\emph{cross-task crowdsourcing}~\citep{mo2013cross}, which 
uses matrix factorisation to model annotator performance in different tasks,
where $\bs t$ corresponds to the objective difficulty of a task.
However, unlike crowdGPPL, they do not use GPs to model the factors, 
nor apply
the approach to preference learning.
For preference learning, collabGP~\citep{houlsby2012collaborative}
is a related model that 
excludes the consensus and uses values in $\bs v_c$ to represent pairs
 rather than individual items, so does not infer item ratings.
It also omits scale parameters for the GPs that 
encourage shrinkage when $C$ is larger than required.
 
We combine the matrix factorisation method with the preference likelihood of Equation \ref{eq:plphi}
to obtain the joint preference model for multiple users, \emph{crowdGPPL}:
%represent a consensus between users,
%if present, while allowing individual users' preferences to deviate from this value through $\bs V^T \bs W$. 
%Hence, $\bs t$ can model the underlying ground truth or consensus in crowdsourcing scenarios, or when using
%multiple label sources to learn preferences for one individual.
\begin{flalign}
&p\left( \bs{y}, \bs V, \bs W, \bs t, s^{(v)}_1 \!\!, .., s^{(v)}_C\!\!, s^{(w)}_1\!\!, .., s^{(w)}_C\!\!, s^{(t)} 
| k_{\theta}, \bs X, k_{\eta}, \bs U, \alpha_0^{(t)}\!\!, \beta_0^{(t)}\!\!,
\alpha_0^{(v)}\!\!, \beta_0^{(v)}\!\!, \alpha_0^{(w)}\!\!, \beta_0^{(w)} \right) 
 & \nonumber \\ 
& = \prod_{p=1}^P \Phi\left( z_p \right) 
\mathcal{N}(\bs t; \bs 0, \bs K_{\theta} /s^{(t)})
\mathcal{G}({s^{(t)}}; \alpha_0^{(t)}, \beta_0^{(t)})
\prod_{c=1}^C \left\{
\mathcal{N}(\bs v_c; \bs 0, \bs K_{\theta} /s^{(v)}_c)
\right.
 & \nonumber \\  
&\left.
\mathcal{N}(\bs w_c; \bs 0, \bs L_{\eta}/s^{(w)}_c) \mathcal{G}(s^{(v)}_c; \alpha_0^{(v)}, \beta_0^{(v)})\mathcal{G}(s^{(w)}_c; \alpha_0^{(w)}, \beta_0^{(w)}) \right\}, &
\label{eq:joint_crowd}
\end{flalign}
where 
$z_p = \bs v_{.,a_p}^T \bs{w}_{.,u_p} + t_{a_p} - \bs v_{.,b_p}^T \bs{w}_{.,u_p} - t_{b_p}$,
index $p$ refers to a tuple $\{u_p, a_p, b_p \}$ that identifies a user and a pair of items,
$\bs U$ is the set of feature vectors for all users,
and $\bs L_{\eta}$ is the prior covariance matrix for the users computed
using $k_{\eta}$.


\section{Scalable Inference}\label{sec:inf}

In the single user case, the goal is to infer the posterior distribution over the utilities of test items, $\bs f^*$, 
given a set of pairwise training labels, $\bs y$. In the multi-user case, we aim to find the posterior over the matrix
$\bs F^*=\bs V^{*T} \bs W^*$ of utilities for test items and test users.
The non-Gaussian likelihood makes exact inference intractable, hence previous work has used
 the Laplace approximation for the single user case~\citep{chu2005preference}
or a combination of expectation propagation (EP) with variational Bayes (VB) for a 
multi-user model~\citep{houlsby2012collaborative}.
The Laplace approximation is a maximum a-posteriori (MAP) solution that
takes the most probable values of parameters rather than integrating over their distributions,
and has been shown to perform poorly for classification~\citep{nickisch2008approximations}. 
EP and VB approximate the true posterior with a simpler, factorized distribution
that can be learned using an iterative algorithm.
For crowdGPPL, the true posterior is multi-modal, since the latent factors can be re-ordered arbitrarily without
affecting $\bs F$, causing a \emph{non-identifiability problem}.
EP would average these modes and produce uninformative predictions over $\bs F$, so
\citet{houlsby2012collaborative} incorporate a VB step that approximates a single mode.
A drawback of EP is that unlike VB, convergence is not guaranteed~\citep{minka2001expectation}.
\todo{Page 8, line 12: VB is only guaranteed to converge for conjugate distributions.}
%do they also linearise in the same way? -- both linearise. But EP uses a joint over y and f as its approximation to p(y|f), then optimises the parameters iteratively. It's not guaranteed to converge. Variational EGP instead approximates
% p(y|f) directly with the best fit Gaussian. It's not clear whether this could be updated iteratively but it doesn't
% seem to work if done simultaneously with the other variables we need to learn (the linearisation), 
% perhaps because the algorithm for learning the weights breaks if the variance of q(y|f), Q, keeps changing. 
% Possibly because Q does not change incrementally. So it's
% possible that an outer loop could be used.

\todo{Page 8, second paragraph: you criticise GFITC for not being decentralisable.  It is
not clear to me that crowdGPPL is decentralisable.  }
%TODO: remove redundancy with the related work section. Consider whether this should actually be in a background section. NM^2 is limiting, not just the other costs. The other costs NP etc come into play in this pairwise model only.
Exact inference for a Gaussian process has computational complexity $\mathcal{O}(N^3)$ 
and memory complexity $\mathcal{O}(N^2)$, where $N$ is the number of data points.
The cost of inference can be reduced using a \emph{sparse} approximation based on a set of 
\emph{inducing points}, which act as substitutes for the set of points in the training dataset.
By choosing a fixed number of inducing points, $M \ll N$, the computational cost is cut to $\mathcal{O}(NM^2)$,
and the memory complexity to $\mathcal{O}(NM)$.
These points must be selected so as to give a good approximation, 
using either heuristics or optimizing their positions to maximize the approximate
marginal likelihood. 
The sparse approximation used by \citet{houlsby2012collaborative} for the collaborative GP 
is the \emph{generalized fully independent training conditional} (GFITC)~\citep{snelson2006sparse}.
In practice, GFITC is unsuitable for datasets with more than a few thousands points,
as the computational and memory costs cannot be constrained
and GFITC is not amenable to distributed computation~\citep{hensman2015scalable}.  
%In contrast to crowdGPPL, \citet{houlsby2012collaborative} place GPs over the space of pairs rather than items,
%which is typically much larger, meaning that $\mathcal{O}(PM^2 + UM^2)$ computational and $\mathcal{O}(PM + UM)$
%memory costs dominate. 
We derive a more scalable approach for GPPL and crowdGPPL using
stochastic variational inference (SVI), an iterative scheme that limits the computational and memory costs at
each iteration~\citep{hoffman2013stochastic}.
% how does hensman 2015 optimize inducing points? This is one selling point of SVI. 
% Advantages of VB:  Nickisch 2008 got poorer results for VB methods they tried than EP. But our method may be different?, Seeger 2000
First, we define an approximate likelihood that enables the SVI method.
%then derive the SVI method for single user GPPL. update equations for the iterative algorithm to optimize
%the approximation. We begin with single user GPPL then extend this to crowdGPPL. 

\subsection{Approximate Pairwise Likelihood}

\todo{ got really confused by Section 4.1.  In equation (7), p(y|f) is a product of the cdfs.  Where does the expectation come from just after the first equal sign? The second equal sign in (7) should be an approximation.  In equation (7), Q is (approximately) the covariance of y conditioned on f and yet, in equation (8), Q is determined by the expectation over f.  Can you please check this section and provide some further explanation.}
\todo{There are many approximations in Sec 4.1 in order to make a tractable inference. It is not clear after so many approximations, eq (7), eq (8), eq (9), eq (11)... , the resultant objective may be very far away from the original one. The properties from GP may not be effective. There is no study to check the gap.}
To obtain a tracatable approximate posterior, we begin by approximating 
the expected preference likelihood (Equation \ref{eq:plphi}) with a Gaussian:
\begin{flalign}
p(\bs y | \bs f) = \mathbb{E}\left[\prod_{p=1}^P \Phi(z_p)\right] = \prod_{p=1}^P \Phi(\hat{z}_p) \approx \mathcal{N}(\bs y; \Phi(\hat{\bs z}), \bs Q),
\label{eq:likelihood_approx}
\end{flalign}
where $\bs Q$ is a diagonal noise covariance matrix
and $\hat{\bs z}=\{\hat{z}_1, ..., \hat{z}_P\}$, and $\hat{z}_p$ is defined by 
Equation \ref{eq:predict_z}.
Since $\Phi(\hat{z}_p)$ defines a bernoulli distribution, for which the conjugate prior is a beta distribution,
we moment match the diagonal entries of $\bs Q$ to the expected variance of a bernoulli distribution as follows:
\begin{flalign}
Q_{p,p} & = \mathbb{E}_{f}[\Phi(z_p)(1 - \Phi(z_p))] 
%= \mathbb{E}_{\bs f}[\Phi(z_p)] - \mathbb{E}_{\bs f}[\Phi(z_p)]^2 - \mathbb{V}_{\bs f}[\Phi(z_p)], 
 \approx \frac{ (y_p + \gamma)(1-y_p + \lambda) }{ (\gamma + \lambda + 1)} - \frac{ (y_p + \gamma)(1-y_p + \lambda) }{ (\gamma + \lambda + 1)^2(\gamma + \lambda + 2)}, &
\end{flalign}
where $\gamma$ and $\lambda$ are parameters of a beta distribution with the same variance as the prior,
$p(\Phi(z_p) | \bs K_{\theta}, \alpha_0, \beta_0)$, and are estimated using numerical integration.
%Setting $\bs Q$ in this way matches the moments of the true likelihood, $\Phi(z_p)$,
%to those of the Gaussian approximation.

Unfortunately, the nonlinear term, $\Phi(\bs z)$ means that the posterior is still intractable, 
so we linearize $\Phi(\bs z)$ by taking its first-order Taylor series expansion
about the expectation of $\bs f$ for GPPL (for crowdGPPL, replace $\bs f$ with $\bs F$ in the following):
\begin{flalign}
\Phi(\bs z) &\approx \tilde{\Phi}(\bs z) = \bs G (\bs f-\mathbb{E}[\bs f]) + \Phi(\hat{\bs z}), \\
G_{p,i} &= \Phi(\mathbb{E}[z_p])(1 - \Phi(\hat{z}_p)) (2y_p - 1)( [i = a_p] - [i = b_p]) 
\end{flalign}
where $\bs G$ is a matrix containing elements $G_{p,i}$, which are the
partial derivatives of the pairwise likelihood with respect to each of 
the latent function values, $\bs f$.
This creates a circular dependency between the posterior mean of $\bs f$ and $\bs G$, %the linearization terms in the likelihood,
which can be estimated iteratively using variational inference~\citep{steinberg2014extended},
as we describe below.
Linearization makes the approximate likelihood conjugate to the prior, $\mathcal{N}(\bs f; \bs 0, \bs K_{\theta}/s)$,
so that the approximate posterior over $\bs f$ is also Gaussian. 
%We now use our likelihood approximation to define an approximate
%posterior over all variables for GPPL:
%use variational inference to estimate the marginal over $\bs f$,
%by optimizing an approximate posterior over all latent variables:
%\begin{flalign}
%p(\bs f, s | \bs y, \bs x, k_{\theta}, & \alpha_0, \beta_0) \approx q(s)q(\bs f), & \nonumber \\
%\textrm{where } \ln q(s) & = \mathbb{E}_{\bs f}[\ln p(s | \bs f, \alpha_0, \beta_0)] & \nonumber \\
%& = \ln \mathcal{N}(\mathbb{E}[\bs f]; \bs 0, \bs K_{\theta}/s) + \ln \mathcal{G}(s; \alpha_0, \beta_0) + \textrm{const}, & \nonumber \\
%\ln q(\bs f) & = \mathbb{E}_{s}[\ln p(\bs f | k_{\theta}, s, \bs x, \bs y) ] & \nonumber \\
%& = \ln \mathcal{N}(\bs y; \tilde{\Phi}(\bs z), \bs Q) + \ln \mathcal{N}(\bs f; \bs 0, \bs K_{\theta}/\mathbb{E}[s]) + \textrm{const}. &
%\label{eq:vb_approx}
%\end{flalign}
%don't some of the expectations over s simplify out?
%Since the Gamma is conjugate to $s$, $q(s)$ is also a Gamma distribution. 
%Likewise, the mean of the Gaussian likelihood approximation is
%a linear function of $\bs f$, 
%which has a multivariate Gaussian prior, hence
%$q(\bs f)$ is also a multivariate Gaussian.
The Gaussian likelihood approximation and linearization
also appear in GP inference methods based on expectation propagation~\citep{rasmussen_gaussian_2006} 
and the extended Kalman filter~\citep{reece2011determining,steinberg2014extended}.
%However, neither these methods nor our approximation in Equation \ref{eq:vb_approx}
%make inference sufficiently scalable, as they all require
%inversion of an $N \times N$ matrix and further computations involving $N \times P$ and $P \times P$ matrices.
We can now use the approximate likelihood to derive equations for SVI.% stochastic variational inference (SVI).

\subsection{SVI for Single User GPPL}

%TODO can we ommit the f from the posterior here?
\todo{Page 9, Section 4.2: Please state up front that you are proposing a mean-field
variational approximation in equation (11).  This will then justify why E[s] appears in equation (13). }
We introduce a sparse approximation to the Gaussian process that allows
us to limit the size of the covariance matrices we need to work with.
% that needs to be inverted,
%and permit stochastic inference methods that consider only a subset of the $P$ observations at each iteration
%~\citep{hensman2013gaussian,hensman2015scalable}. 
To do this, we introduce a set of $M \ll N$ \emph{inducing} items with inputs 
$\bs x_m$,
utilities $\bs f_m$, covariance $\bs K_{mm}$,
and covariance between the observed and inducing items, $\bs K_{nm}$.
% The inducing points act as proxies for the observed points during inference,
% and thereby reduce the number of data points we have to perform costly operations % over.
%We modify the variational approximation in Equation \ref{eq:vb_approx} to introduce the inducing points 
For clarity, we omit $\theta$ from this point on.
We assume an approximation to the posterior over the inducing and training items
that factorises between different sets of latent variables:
\begin{flalign}
p(\bs f, \bs f_m, s | \bs y, \bs x, \bs x_m, k_{\theta}, \alpha_0, \beta_0) &\approx q(\bs f, \bs f_m, s) = q(s)q(\bs f)q(\bs f_m), \label{eq:svi_approx} &&
\end{flalign}
where are $q(.)$ are \emph{variational factors}, which we define below. 
Each factor for a subset of latent variables, $\bs z$, takes the form $\ln q(\bs z) = \mathbb{E}_{\not \bs z}[\ln p(\bs z, \bs x, \bs y)]$, that is, the expectation with respect
to all other latent variables, $\not \bs z$, of the log joint distribution
of the observations and the current subset of latent variables, $\bs z$.

We marginalize $\bs f$ to obtain the factor for $\bs f_m$:
\begin{flalign}
\ln q(\bs f_m) &= \ln \mathcal{N}\left(\bs y; \tilde{\Phi}(\bs z), \bs Q\right)
+ \ln\mathcal{N}\left(\bs f_m; \bs 0, \bs K_{mm}/\mathbb{E}\left[s\right]\right)  + \textrm{const}, & \nonumber \\
 & = \ln \mathcal{N}(\bs f_m; \hat{\bs f}_m, \bs S ),
 \label{eq:fhat_m}
\end{flalign}
where the variational parameters $\hat{\bs f}_m$ and $\bs S$ are computed using the iterative SVI procedure described below.
We choose an approximation of $q(\bs f)$ that depends only on the inducing point utilities, $\bs f_m$, and is independent of the observations:
 \begin{flalign}
\ln q(\bs f) & = \ln \mathcal{N}(\bs f; \bs A \hat{\bs f}_m, 
\bs K + \bs A (\bs S - \bs K_{mm}/\mathbb{E}[s]) \bs A^T ),
\end{flalign}
where $\bs A=\bs K_{nm} \bs K^{-1}_{mm}$.
This means we no longer need to invert an $N \times N$ covariance matrix to compute $q(\bs f)$.
The factor $q(s)$ is also modified to depend on the inducing points:
\begin{flalign}
& \ln q(s) = \mathbb{E}_{q(\bs f_m)}[\ln\mathcal{N}(\bs f_m| \bs 0, \bs K_{mm}/s)] + \ln \mathcal{G}(s; \alpha_0, \beta_0) + \mathrm{const}
= \ln \mathcal{G}(s; \alpha, \beta), & \label{eq:qs}
\end{flalign}
where $\alpha= \alpha_0 + \frac{M}{2}$ and $\beta = \beta_0 + \frac{
\textrm{tr}(\bs K^{-1}_{mm}(S + \hat{\bs f}_m \hat{\bs f}_m^T))}{2}$.
%None of the other factors depend on $\ln q(\bs f)$,
%so it need not be computed unless required for prediction.

%once the other factors have already been estimated, as explained in the next subsection.

%Since the inducing points stand in for the observed locations,
%their choice affects the quality of our approximation. 
The location of inducing points can be learned
as part of the variational inference procedure ~\citep{hensman2015scalable},
or by optimizing a bound on the log marginal likelihood.
However, the former breaks the convergence guarantees, and both approaches
may add substantial computational cost. 
We find that we are able to obtain good performance by choosing inducing points up-front using K-means++~\citep{arthur2007k} with $K=M$ to  
cluster the feature vectors, 
then taking the cluster centers as inducing points that represent the spread of observations across feature space.
%Compared to standard K-means, K-means++ introduces a new method 
%for choosing the initial cluster seeds that
%provides theoretical bounds on the error function. In practice, this 
%reduces the number of poor-quality clusterings.


%\subsection{SVI for Single-User Preference Learning}

%The approximate posterior can now be optimized using stochastic variational inference (SVI),
%which uses a series of stochastic updates involving different subsets of the observations.
%Variational inference 
We can apply variational inference to iteratively reduce the KL-divergence between our approximate posterior
%$q(s)q(\bs f)q(\bs f_m)$
and the true posterior (both stated in Equation \ref{eq:svi_approx}) %, $p(s, \bs f, \bs f_m | \bs K, \alpha_0, \beta_0, \bs y)$,
by maximizing a lower bound, $\mathcal{L}$, on the log marginal likelihood (see also the detailed equations in Appendix \ref{sec:vb_eqns}):
%(see also Equation \ref{eq:full_L_singleuser} in the Appendix):%, $\ln p(\bs y | \bs K, \alpha_0, \beta_0)$ :
\begin{flalign}
\ln p(\bs y | \bs K, \alpha_0, \beta_0) = \textrm{KL}(q(\bs f, \bs f_m, s)  || p(\bs f, \bs f_m, s | \bs y, \bs K, \alpha_0, \beta_0)) 
+ \mathcal{L} &&\label{eq:lowerbound}
\\
%\end{flalign}
%Taking expectations with respect to the variational $q$ distributions, $\mathcal{L}$ is:
%\begin{flalign}
\mathcal{L} = \mathbb{E}_{q(\bs f)}[\ln p(\bs y | \bs f)]
+ \mathbb{E}_{q(\bs f_m, s))}[\ln p(\bs f_m, s | \bs K, 
\alpha_0, \beta_0) -\ln q(\bs f_m) - \ln & q(s) ]. && \nonumber
\end{flalign}
%         invK_mm_expecFF = self.invK_mm.dot(self.uS + self.um_minus_mu0.dot(self.um_minus_mu0.T))
%         self.rate_s = self.rate_s0 + 0.5 * np.trace(invK_mm_expecFF)
We optimize $\mathcal{L}$ by initializing the $q$ factors randomly, then
updating each one in turn, taking expectations with respect to the other factors. 
The only term in $\mathcal{L}$ that refers to the observations, $\bs y$, 
is a sum of $P$ terms, each of which refers to one observation only.
This means that $\mathcal{L}$ can be maximized iteratively by considering a random subset of 
observations at each iteration~\citep{hensman2013gaussian}.
%Therefore, the SVI solution replaces Equations \ref{eq:fhat_m} and \ref{eq:S} for computing
%$\hat{\bs f}_m$ and $\bs S$ over all observations with a sequence of stochastic updates.
For the $i$th update of $q(\bs f_m)$, we randomly select observations $\bs y_i = \{ y_p \forall p \in \bs P_i \}$, where $\bs P_i$ is random subset of indexes of observations.
%Rather than using the complete matrices, 
We then perform updates using $\bs Q_i$ (rows and columns of $\bs Q$ for observations in $\bs P_i$),
$\bs K_{im}$ and $\bs A_i$, (rows of $\bs K_{nm}$ and $\bs A$ referred to by $\{y_p \forall p \in \bs P_i\}$),
$\bs G_i$ (rows of $\bs G$ in $\bs P_i$ and columns referred to by any items $\{a_p \forall p \in \bs P_i \} \cup \{ b_p \forall p \in \bs P_i\}$),
and $\hat{\bs z}_i = \{ \mathbb{E}[\bs z_p] \forall p \in P_i \}$.
%All matrices with subscript $_i$ contain only the subset of elements relating to 
%observations in $\bs P_i$.
% The linearization matrix $\bs G_i$ is the subset of elements in $\bs G$ relating to observations in $\bs P_i$, 
%  is the corresponding subset of elements in $\bs Q$,
%  is the covariance between the items referred to by pairs in $\bs P_i$ 
% and the inducing points,
% and  contains the corresponding rows of $\bs A$.
The update equations optimize the natural parameters of the Gaussian distribution by following the
natural gradient~\citep{hensman2015scalable}:
\begin{flalign}
\bs S^{-1}_i  & = (1 - \rho_i) \bs S^{-1}_{i-1} + \rho_i\left( \mathbb{E}[s]\bs K_{mm}^{-1} + \pi_i\bs A_i^T \bs G^T_{i} \bs Q^{-1}_i \bs G_{i} \bs A_{i} \right)& 
\label{eq:S_stochastic} \\
\hat{\bs f}_{m,i}  & = \bs S_i \left( (1 - \rho_i) \bs S^{-1}_{i-1} \hat{\bs f}_{m,i-1}  + 
%\right. \nonumber \\
%& \left.\hspace{1.5cm} 
\rho_i \pi_i  
\bs A_{i}^{T} \bs G_{i}^T \bs Q_i^{-1} \left( \bs y_i  - \Phi(\mathbb{E}[\bs z_i]) + \bs G_{i} \bs A_i \hat{\bs f}_{m,i} \right) \right) & 
\label{eq:fhat_stochastic}
\end{flalign}
where
$\rho_i=(i + \epsilon)^{-r}$ is a mixing coefficient that controls the update rate,
$\pi_i = \frac{P}{|P_i|}$ weights each update according to sample size,
 $\epsilon$ is a delay hyperparameter and $r$ is a forgetting rate~\citep{hoffman2013stochastic}.
For the inverse scale, %, $s$, %can also be learned as part of the SVI procedure. Its variational factor,
$q(s)$ is updated using Equation \ref{eq:qs}, then its expected value is computed as %has the following update equations:
$\mathbb{E}[s] = \frac{\alpha}{\beta}$.

\begin{algorithm}[t]
 \KwIn{ Pairwise labels, $\bs y$, training item features, $\bs x$, 
 test item features $\bs x^*$}
 \nl Compute kernel matrices $\bs K$, $\bs K_{mm}$ and $\bs K_{nm}$ given $\bs x$
 \nl Initialise $\mathbb{E}[s]$, $\mathbb{E}[\bs f]$and $\hat{\bs f}_m$ to prior means
 and $\bs S$ to prior covariance $\bs K_mm$\;
 \While{$\mathcal{L}$ not converged}
 {
 \nl Select random sample, $\bs P_i$, of $P$ observations
 \While{$\bs G_i$ not converged}
  {
  \nl Compute $\bs G_i$ given $\mathbb{E}[\bs f_i]$ \;
  \nl Compute $\hat{\bs f}_{m,i}$ and $\bs S_{i}$ \;
  \nl Compute $\mathbb{E}[\bs f_i]$ \;
  }
 \nl Update $q(s)$ and compute $\mathbb{E}[s]$ and $\mathbb{E}[\ln s]$\;
 }
\nl Compute kernel matrices for test items, $\bs K_{**}$ and $\bs K_{*m}$, given $\bs x^*$ \;
\nl Use converged values of $\mathbb{E}[\bs f]$and $\hat{\bs f}_m$ to estimate
posterior over $\bs f^*$ at test points \;
\KwOut{ Posterior mean of the test values, $\mathbb{E}[\bs f^*]$ and covariance, $\bs C^*$ }
\vspace{0.5cm}
\caption{The SVI algorithm for preference learning with a single user.}
\label{al:singleuser}
\end{algorithm}
The complete SVI algorithm is summarized in Algorithm \ref{al:singleuser}.
The use of an inner loop to learn $\bs G_i$ avoids the need to store the complete matrix, 
$\bs G$.
The inferred distribution over the inducing points can be used 
for predicting the values of test items, $f(\bs x^*)$:
\begin{flalign}
\bs f^* \! \! &= \bs K_{*m} \bs K_{mm}^{-1} \hat{\bs f}_m, &
\bs C^* \! \! = \bs K_{**} + \bs K_{*m} \bs K_{mm}^{-1} (\bs S - \bs K_{mm} / \mathbb{E}[s] ) \bs K_{*m}^T \bs K_{mm}^{-1},
\end{flalign}
where $\bs C^*$ is the posterior covariance of the test items, $\bs K_{**}$ is their prior covariance, and
$\bs K_{*m}$ is the covariance between test and inducing items.
%It is possible to recover the lower bound proposed by 
%\citet{hensman2015scalable} for classification by generalizing the
%likelihood to arbitrary nonlinear functions, and omitting terms relating to $p(s|\alpha_0,\beta_0)$ and $q(s)$.
% However, our approach avoids expensive quadrature methods by linearizing the likelihood to enable analytical updates. We also infer $s$ in a Bayesian manner, 
% rather than treating as a hyper-parameter, which is important for preference learning where $s$ controls the noise level of the observations relative to  $f$. 

\subsection{SVI for CrowdGPPL}

We now extend the SVI method to the crowd preference learning model proposed in
Section \ref{sec:crowd_model}.
To begin with, we extend the variational posterior in Equation \ref{eq:svi_approx}
to approximate the crowd model defined in Equation \ref{eq:joint_crowd}:
\begin{flalign}
& p( \bs V, \bs V_m, \bs W, \bs W_m, \bs t, \bs t_m, s_1, ..., s^{(v)}_c, s^{(t)} | \bs y, \bs x, \bs x_m, \bs u, \bs u_m, k, \alpha_0, \beta_0 ) & \\
& 
\approx 
q(\bs t) q(\bs t_m)
\prod_{c=1}^{C} q(\bs v_{c})q(\bs w_c)q(\bs v_{c,m})q(\bs w_{c,m})
q(s^{(v)}_c)q(s^{(t)})
= q(\bs F) q(s^{(t)}) \prod_{c=1}^C q(s^{(v)}_c), & \nonumber 
\end{flalign}
where $\bs u_m$ are the feature vectors of inducing users. By updating 
each of the $q$ factors in turn,
the SVI procedure optimizes the lower bound on the marginal likelihood 
(see also Equation \ref{eq:lowerbound_crowd_full} in the Appendix):
\begin{flalign}
\mathcal{L}_{cr} & = 
\mathbb{E}_{q(\bs F)}%(\bs t, \bs t_m, \bs V, \bs V_m, \bs W, \bs W_m, s_1,...,s^{(v)}_c,s^{(t)})
[\ln p(\bs y | \bs F)] 
+ \mathbb{E}_{q(\bs t_m), q(s^{(t)})}[\ln p(\bs t_m, s^{(t)} | \bs K_{mm}, \alpha_0, \beta_0)
- \ln q(\bs t_m)] & \nonumber \\
&\sum_{c=1}^C \!\! \bigg\{ 
\mathbb{E}_{q(\bs v_{m,c}), q(s^{(v)}_c)}[\ln p(\bs v_{m,c}, s^{(v)}_c | \bs K_{mm}, \alpha_0, \beta_0) - \ln q(\bs v_{m,c})]
&  \nonumber \\ 
 & +  \mathbb{E}_{q(\bs w_{m,c})}[\ln p(\bs w_{m,c} | \bs L_{mm}/s^{(w)}_c )
  - \ln q(\bs w_{m,c} ) ] \bigg\} . & 
  \label{eq:lowerbound_crowd}
\end{flalign}
The algorithm follows the same pattern as Algorithm \ref{al:singleuser}, computing means and covariances
for  $\bs V_m$, $\bs W_m$ and $\bs T_m$ instead of $\bs f_m$.
%This approximation factorizes the joint distribution between the latent item factors, $\bs V$, the latent user factors, $\bs W$, and the common means, $\bs t$. 
%but does not requiring factorization across the latent dimensions $\bs w_1,...,\bs w_C$ and $\bs v_1,...,\bs v_C$.
The variational factor for the inducing item factors is:
%TODO make the vectors all lower case for V_c and W_c
%TODO do something to make sure all the block diags are properly indiciated including A_v... but actually I think they are useless because the off-diagonal blocks only affect the collapsed posterior covariance and are never used in computing any other variables.
%TODO intialization of the factors
%TODO put definition of f in somewhere in the model section 
%TODO big Sigma is variance of W because it's different...
%TODO why is the scaling by W nice and simple without off-diagonals, but scaling by V is not? I think that when Q is diagonal, off-diagonals in any scaling factors are irrelevant.
\begin{flalign}
\ln q(\bs v_{m,c}) = \;\;& % \mathbb{E}_{q(\bs F)}\left[ doesn't need this because it's already implicit in tilde Phi z
\ln \mathcal{N}\left( \bs y; \tilde{\Phi}(\bs z), Q \right) % \right] 
 + \ln\mathcal{N}\left(\bs v_{m,c}; \bs 0, \bs K_{mm}/ \mathbb{E}[s^{(v)}_c]\right) 
+ \textrm{const} & \nonumber \\
% are the dimensions collapsed to a single MVN?
= \;&  \ln \mathcal{N}(\bs v_{m,c}; \hat{\bs v}_{m,c}, \bs S_{v,c}), &
\end{flalign}
where the posterior mean $\hat{\bs v}_{m,c}$ and covariance $\bs S_{v,c}$ are computed using 
update equations of the same form as those of the single user GPPL in 
Equations \ref{eq:S_stochastic} and \ref{eq:fhat_stochastic}.
For reasons of space, we provide the complete equations for $\hat{\bs v}_{m,c}$ and $\bs S_{v,c}$ in 
Appendix \ref{sec:post_params}, Equations \ref{eq:Sv} and \ref{eq:hatv}.
The variational component for the inducing points of the common item mean follows a similar pattern:
\begin{flalign}
\ln q(\bs t_m) = \;\;& %\mathbb{E}_{q(\bs F)}[
\ln \mathcal{N}\left( \bs y; \tilde{\Phi}(\bs z), Q \right) %] 
+ \ln\mathcal{N}(\bs t_m; \bs 0, \bs K_{mm}/\mathbb{E}[s^{(t)}])
+ \textrm{const} & \nonumber \\
= \;\;& \ln \mathcal{N}\left( \bs t_m; \hat{\bs t}_{m}, \bs s^{(t)} \right), &
\end{flalign}
where again, the posterior mean $\hat{\bs t}$ and covariance $\bs S_{t}$ use similar updates to Equation
Equations \ref{eq:S_stochastic} and \ref{eq:fhat_stochastic}, and their full definitions
are given in Appendix \ref{sec:post_params}, Equations \ref{eq:St} and \ref{eq:hatt}.
Finally, %require a different linearization matrix, $\bs J \in P \times U$, containing partial derivatives 
%of the pairwise likelihood with respect to $\hat{w}_c$. Its elements are given by:
%\begin{flalign}
%J_{p,j} = \Phi(\mathbb{E}[z_p])(1 - \Phi(\mathbb{E}[z_p]) (2y_p - 1) [u_p = j] % needs to be added or subtracted depending on a or b
%\end{flalign} 
%now multiply by V. What about covariances between v?
the variational distribution for the inducing users factors is:% then as follows:
\begin{flalign}
\ln q(\bs w_{m,c}) = \;\;& %\mathbb{E}_{q(\bs F)}\left[
\ln \mathcal{N}\left( \bs y; \tilde{\Phi}(\bs z), Q \right) %\right] 
+ \ln\mathcal{N}(\bs w_{m,c}; \bs 0, \bs L_{mm}/s^{(w)}_c)
+ \textrm{const} & \nonumber \\
= \;\;& \ln \mathcal{N}\left( \bs w_{m,c}; \hat{\bs w}_{m,c}, \bs \Sigma_c \right), & 
\end{flalign}
where $\hat{\bs w}_c$ and $\bs \Sigma_{c}$ also follow the pattern of
Equations \ref{eq:S_stochastic} and \ref{eq:fhat_stochastic}, and their full definitions
are given in Appendix \ref{sec:post_params}, Equations \ref{eq:St} and \ref{eq:hatt}.
The expectations for the inverse scales, $s_1,...,s^{(v)}_c$ and $s^{(t)}$, can be 
computed using Equation \ref{eq:qs} by
substituting in the corresponding terms for each $\bs v_c$ or $\bs t$ instead of $\bs f$. 
% The equations for the means and covariances 
% can be adapted for stochastic updating by applying weighted sums over
% the stochastic update and the previous values in the 
% same way as  Equation \ref{eq:S_stochastic} and \ref{eq:fhat_stochastic}.
% The stochastic updates for the inducing points of the latent factors depend 
% on expectations with respect to the observed points. 
% As with the single user case, the variational factors at the observed items are independent of the observations given the variational factors of the inducing points
% (likewise for the observed users):
% \begin{flalign}
% \ln q(\bs V) & = \sum_{c=1}^C \ln \mathcal{N}\left( \bs v_c; \bs A_v\hat{\bs v}_{m,c}, 
% \frac{\bs K_{v}}{\mathbb{E}[s^{(v)}_c]} + \bs A_v (\bs S_{m,c} - \frac{\bs K_{mm}}{\mathbb{E}[s^{(v)}_c]})\bs A_v \right) & \label{eq:qv} \\
% \ln q(\bs t) & = \ln \mathcal{N}\left( \bs t; \bs A_t \hat{\bs t}_m, 
% \frac{\bs K_{t}}{\mathbb{E}[s^{(t)}]} + \bs A_t (\bs s^{(t)} - \frac{\bs K_{mm}}{\mathbb{E}[s^{(t)}]})\bs A_t \right)  & \label{eq:qt}\\
% \ln q(\bs W) & = \sum_{c=1}^C \ln \mathcal{N}\left( \bs w_c; \bs A_w \hat{\bs w}_{m,c}, \bs L_{} + \bs A_w (\bs\Sigma - \bs L_{mm}/s^{(w)}_c) \bs A_w \right). &
% \label{eq:qw}
% \end{flalign}

Predictions for crowdGPPL can be made by computing the posterior mean utilities, $\bs F^*$, 
and the covariance $\bs \Lambda_u^*$ for each user, $u$, in the test set:
\begin{flalign} \label{eq:predict_crowd}
&\bs F^* = \hat{\bs t}^* + \sum_{c=1}^C \hat{\bs v}_{c}^{*T} \hat{\bs w}_{c}^*, \hspace{1cm} \bs \Lambda_u^* = \bs C_{t}^* + \sum_{c=1}^C \omega_{c,u}^* \bs C_{v,c}^* + \hat{w}_{c,u}^2  \bs C_{v,c}^*  +\omega_{c,u}^* \hat{\bs v}_{c}\hat{\bs v}_{c}^T, &
\end{flalign}
where $\hat{\bs t}^*$, $\hat{\bs v}_{c}^*$ and $\hat{\bs w}_{c}^*$ are posterior means of the predictions,
$\bs C_{t}^*$ and $\bs C_{v,c}^*$ are posterior item covariances of the predictions,
and $\omega_{c,u}^*$ is the posterior variance for the user components for the predicted users. These
terms are defined in Appendix \ref{sec:predictions}, Equations \ref{eq:tstar} to \ref{eq:omegastar}.
The mean $\bs F^*$ and covariances $\Lambda^*_u$ can be inserted into Equations \ref{eq:predict_z} and \ref{eq:plphi} to predict pairwise labels.
In practice, the full covariance terms are needed only for Equations \ref{eq:predict_z}, so need only be computed
between items for which we wish to predict pairwise labels. %Hence, the covariance for a large test set need not be computed at once.

%In this section, we proposed an SVI scheme for GPPL and crowdGPPL, 
%which can readily be adapted to regression or classification tasks by swapping out the preference likelihood, resulting in 
%different values for $\bs G$ and $\bs H$. 
% We now show how to learn the  
% length-scale parameters required to compute the prior covariances using typical kernel functions.
%then demonstrate how our method can be applied to learning user preferences or
%consensus opinion when faced with disagreement.
 

%\section{Gradient-based Length-scale Optimization}\label{sec:ls}

In the previous sections, we defined preference learning models that 
incorporate GP priors over the latent functions.
The covariances of these GPs are defined by a kernel function $k$, 
typically of the following form:
\begin{flalign}
k_{\bs \theta}(\bs x, \bs x') = \prod_{f=1}^F k_f\left(\frac{|x_f - x_f'|}{l_f}, \bs\theta_f \right)
\textrm{, where } 
\bs \theta = \{l_1,...,l_F, \bs\theta_1,...,\bs \theta_F \}
\label{eq:kernel}
\end{flalign}
where $F$ is the number of features, 
$l_f$ is a length-scale hyper-parameter,
and $\bs \theta_f$ are additional hyper-parameters for an individual 
feature kernel, $k_f$.
Each $k_f$ is a function of the distance between the $f$th feature values in 
feature vectors $\bs x$ and  and $\bs x'$.
The product over features in $k$ means that data points have 
high covariance only if the kernel functions, $k_f$, for all features are high 
(a soft AND operator). 
It is possible to replace the product with a sum, causing covariance to increase
for every $k_f$ that is similar (a soft OR operator),
or other combinations of the individual feature kernels.
The choice of combination over features is therefore an additional hyper-parameter.
% citations? 

The length-scale, $l_f$, controls the smoothness of the function, $k_f$,
across the feature space
and the contribution of each feature to the model. 
If a feature has a large length-scale,
its values, $\bs x$, have less effect on $k_{\bs\theta}(\bs x, \bs x') $
than if it has a shorter length-scale.
Hence, it is important to set $l_f$ to correctly capture feature relevance.
A computationally frugal option is the median heuristic: 
\begin{flalign}
 l_{f,MH} = \frac{1}{F} \mathrm{median}( \{ |x_{i,f} - x_{j,f}| \forall i=1,..,N, \forall j=1,...,N\} ).
\end{flalign}
Given enough samples of $x_f$, the median heuristic will capture the inherent scale 
of a feature and has been shown to work reasonably well for the task of 
comparing distributions~\citep{gretton2012optimal}. However, it is a simple heursitic
with no guarantees of optimality. 
Alternatively, we can choose $l_f$ by Bayesian model selection using 
the type II maximum likelihood method, 
which chooses the value of $l_f$ that 
maximizes the marginal likelihood, $p(\bs y | \bs \theta)$.
Since the marginal likelihoods for our models are intractable, we maximize
the variational lower bound, $\mathcal{L}$, as an approximation (
defined in Equation \ref{eq:lowerbound} for a single user, and Equation \ref{eq:lowerbound_crowd} for the crowd model). 
Optimizing a kernel length-scale in this manner is known as automatic relevance determination (ARD)~\citep{rasmussen_gaussian_2006}, since the optimal
value of $l_f$ depends on the relevance of $f$.

% Removing irrelevant features could improve performance, 
% since it reduces the dimensionality of the space of the preference function.
%A problem when using text data is that large vocabulary sizes and additional linguistic features 
%lead to a large number of dimensions, $D$. 
%The standard maximum likelihood II optimisation requires 
%$\mathcal{O}(D)$ operations to tune each length-scale.
To optimize the length-scales efficiently, we turn to gradient-based methods
 such as L-BFGS-B~\citep{zhu1997algorithm}, which allow us to optimize
 all length-scales simultaneously, rather than one-by-one.
 For the single user model, the required gradients of $\mathcal{L}(q)$
(Equation \ref{eq:lowerbound}) with respect to $l_f$ are as follows:
%Following the derivations in Appendix \ref{sec:vb_eqns}, Equation \ref{eq:gradient_ls},
\begin{flalign}
\nabla_{l_f}\mathcal{L}(q) & =  
 - \frac{1}{2}\bigg \{
 \mathbb{E}[s] \hat{\bs f}_{m}^T \bs K_{mm}^{-1} \frac{\partial \bs K_{mm}}{\partial l_f} \bs K_{mm}^{-1} \hat{\bs f}_{m} 
 \nonumber \\
& \hspace{2.5cm} + \mathrm{Tr}\left(
\mathbb{E}[s]\bs S^T\bs K_{mm}^{-1} - 1\right)
 \frac{\partial \bs K_{mm}}{\partial l_f} \bs K_{mm}^{-1}
\bigg\}.
\end{flalign}
A number of terms in the lower bound (Equation \ref{eq:lowerbound})
 contain parameters of the 
variational posteriors, i.e. $\bs f$, $\bs S$, $s$, $a$ and $b$.
Through the variational updates, these terms depend indirectly on the length-scale. 
The derivative of these terms with respect to the length-scale is:

If the variational algorithm has converged, the terms $?$ will be zero: the length-scale
is at a maximum. Hence, these terms can be eliminated in this case. This leads to the


 cancel when the updates have converged...
%*** The symbol f should be replaced with \varphi or f? due to clash with function f.

Multi-user model.

Given the kernel function defined in Equation \ref{eq:kernel}, which
contains a product over features,
the partial derivative of the covariance matrix $\bs K_{mm}$ with respect to 
$l_f$ is given by:
\begin{flalign}
\frac{\partial \bs K_{mm}}{\partial l_f} 
& = \frac{\bs K_{mm}}{k_{f}(|\bs x_{mm,f}, \bs x'_{mm,f}) }
\frac{ \bs k_{l_f}(|\bs x_{mm,f}, \bs x'_{mm,f})}{\partial l_f} \nonumber ,\\
\end{flalign}

The choice of kernel function...
In our implementation, we choose the Mat\`ern $\frac{3}{2}$ kernel function for $k$
due to its general properties of smoothness
~\citep{rasmussen_gaussian_2006}... add in citation that shows its good performance.
 
 For the Mat\`ern $\frac{3}{2}$ kernel, the 
 $\frac{\partial \bs K_{l_d}}{\partial l_d}$ is a matrix, where each 
entry, $i,j$,  is defined by:
\begin{flalign}
& \frac{\partial K_{d,ij}}{\partial l_d} = 
\frac{3\bs |\bs x_{i,d} - \bs x_{j,d}|^2}{l_d^3} \exp\left( - \frac{\sqrt{3} \bs |\bs x_{i,d} - \bs x_{j,d}|}{l_d} \right). &
\label{eq:kernel_der}
\end{flalign}
% is defined by Equation \ref{eq:kernel_der}.
% Since we cannot compute $\bs K$ in high dimensions, in practice we substitute $\bs K_{mm}$ for $\bs K$,
% $\bs S$ for $\bs C$, $\hat{\bs f}_{m}$ for $\hat{\bs f}$ and $\bs\mu_{m}$ for $\bs\mu$ so that 

We can define an optimization procedure for the length-scales...
By following the gradients of the length-scale given by 

\section{Experiments}\label{sec:expts}

The first dataset contains a number of pairwise convincingness preference labels for a set of arguments from a crowd of workers. Each label is associated with two arguments and expresses whether a worker in the crowd found the first argument most convincing, the second argument, or had no preference.  

The task is to train the models then predict the preference labels for held-out data. Each method can be assessed in terms of classification accuracy, since the labels have three possible values. In the first set of experiments we evaluate the baselines and the different methods for modelling correlations between workers' preferences. In the second set of experiments, we assess the value of different language features. Finally, the third experiment evaluates approaches that integrate both argument features and models of preference correlations.

\subsection{Hypotheses -- needs to be reconciled with the above paragraph} 

Prior work on convincingness:
\begin{itemize}
 \item \cite{habernal2016argument} shows how to predict convincingness of arguments by training a NN 
 from crowdsourced annotations. 
 \item \cite{lukin2017argument} shows that persuasion is correlated with personality traits.
\end{itemize}

We build on this to show...
\begin{itemize}
 \item How we can predict convincingness for a specific user given only previous preferences and 
 preferences of others (collaborative filtering)
 \item How we can rank arguments in terms of convincingness using preference learning, and resolve
 conflicts.
 \item How a combination of text and personality features improves predictions of convincingness
 \item Which input features/side information features are informative and how we can learn this using a Bayesian approach
 \item That we can extract human-interpretable latent features in people and items,
 which improve performance over just using the input features.
 \item Uncertainty is modelled correctly in the Bayesian approach so that (a) we can filter
 out the uncertain decisions to improve accuracy; (b) the brier score/cross entropy error is lower;
 \item Uncertainty also means that simple active learning works better
 \item Bayesian approach means accuracy is better with sparse data, e.g. in the cold-start situation,
  esp. if we can use (a) and (c) to avoid acting on uncertain labels.
\end{itemize}
This is all useful because we can use the approach to determine which features are worth 
obtaining, make predictions when data is sparse, and obtain data from users efficiently.

The steps to show this are:
\begin{enumerate}
  \item Show a table comparing the baselines, alternative collaborative filtering methods, 
  results from \cite{habernal2016argument}, and unsupervised method
  \item Add in results when using the input information with out method
  \item Show a table comparing the baselines, alternative collaborative filtering methods, 
  results from \cite{lukin2017argument}, and unsupervised method
  \item Add in results using item information, person information and both
  \item Visualise latent features?
  \item Table showing importance of input features
  \item Add results with lower confidence items excluded to the tables in 1-4. We can also plot the
  effect of confidence threshold on our results and on the rival methods.
  \item Add in Bier/cross entropy -- may need to rerun the original code from the previous papers?
  \item Run \cite{habernal2016argument} and my complete method with reduced data -- check accuracy as it increases. Use confidence cut-off from previous results.
  \item Simple active learning approach selecting the most uncertain data point (this will be due to 
  uncertainty about a person, an item with too little data, or disagreement/stochasticity in the likelihood). The plot can be added to the previous results and should be run with rival methods.
\end{enumerate}

\begin{table*}
  \begin{tabularx}{\textwidth}{ l  X  X  X }
  Dataset & Dataset properties & Hypothesis & Methods \\
  \hline\hline\\
  1. UKPConvArgStrict & 
  MACE output with confidence $\ge 95\%$; \newline
  Discard arguments marked as equally convincing; \newline
  Discard conflicting preferences. & 
  Bayesian method is competitive with previous methods at predicting clean preference pairs from a clean dataset. &
  GP Preference learning + linguistic features + embeddings. \\\hline\\
  2. UKPConvArgMACE & 
  MACE output with confidence $\ge 95\%$;\newline
  No further filtering. & 
  Bayesian method is competitive with previous methods if filtering step is removed. &
  GP Preference learning + linguistic features + embeddings. \\\hline\\      
  3. UKPConvArgRank & 
  MACE output with confidence $\ge 95\%$;\newline
  Equal arguments included; \newline
  PageRank used to rank arguments. & 
  Bayesian method is competitive with previous methods at ranking arguments and 
   can perform ranking given pairs rather than rank scores. &
  GP regression with preference learning output + linguistic features + embeddings (trained on rank scores); \newline
  GP Preference learning + linguistic features + embeddings (trained on pairs). \\\hline\\  
  4. UKPConvArgAll & 
  No filtering, all pairs from original workers are provided. & 
  Bayesian method can predict argument pairs for individual annotators with competitive performance to rival methods on clean, combined data; \newline
  There are patterns of common agreement/disagreement among workers. &
  Bayesian Preference Components + linguistic features + embeddings (pair prediction). \\\hline\\
  5. UKPConvArgAllR & 
  No filtering, all pairs from original workers are provided; \newline 
  PageRank used to produce gold-standard ranking. & 
  Bayesian model can predict individual argument rankings;\newline
  Significant differences between individual rankings and gold-standard ranking. &
  Bayesian Preference Components + linguistic features + embeddings (ranking output). \\\hline\\ 
  %UKPConvArgAll (train), UKPConvArgStrict (test) &
  %Use the original pairs for training and try to predict the cleaned output &
  %(Optional -- more about preference learning from noisy data, so probably deserves a separate paper/student project?)
  %Show that the model can also be used to predict a gold standard -- end-to-end preference learning &
  %Bayesian Preference Components + linguistic features + embeddings + gold-standard pairs on the training data (model is not really designed for aggregation, but for sharing information between similar people and items; without gold standard training labels, we need to know the y-feature mixture of the gold-standard); 
  %HeatmapBCC with preference learning forward model (deals directly with noisy labels, rather than different opinions about convincingness; useful for learning from implicit feedback)
  \end{tabularx}
  \caption{\label{tab:expt_data} The datasets and hypotheses in each experiment.}
\end{table*}

\subsection{Alternative Application -- Notes}

The core contribution is to do preference learning with sparse observations with text data. 
There may be several other problems related to NLP and argumentation, more specifically, that
could benefit from this approach. 
Argument cloze task? 
The model can be adapted to classification problems, regression, or mixed observation types by applying a different likelihood. The core of the method is the abstraction of a latent function over items and people, dependent on latent features of items and people, with the ability to include side information.
The paper could therefore apply the model to multiple NLP tasks, and be a methodology paper. 
This needs us to discuss and distinguish the class of problems and novelty of the method. 
What are the alternative methods, e.g. if we were to use this for classification? One could supply all 
item and person data to a neural network and train it in a semi-supervised manner?
Sticking to the idea of preference learning or ranking, what other tasks could be handled in this way?

\section{Conclusions}\label{sec:conclusion}

We proposed a novel Bayesian preference learning approach 
for modelling both the preferences of individuals 
and the overall consensus of a crowd. 
Our model learns the latent utilities of items from pairwise comparisons 
using a combination of Gaussian processes and Bayesian matrix factorisation 
to capture differences in  opinion.
We introduce a stochastic variational inference (SVI) method, that, 
unlike previous work, can scale to arbitrarily large datasets,
since its time and memory complexity do not grow with the dataset size.
Our experiments confirm the method's scalability and
 show that jointly modelling the consensus and personal
preferences can improve predictions of both.
Our approach performs competitively
against less scalable alternatives
%despite the approximations required for SVI.
and improves on 
the previous state of the art
for predicting argument convincingness from crowdsourced data~\citep{simpson2018finding}.
%far better than previous
%Gaussian process preference learning methods 
%without harming  
%performance on individual utility prediction and significantly improving performance
%on consensus learning.

Future work will investigate learning inducing point locations and
optimising length-scale hyperparameters by maximising $\mathcal L$ as part of the variational inference method.
Another important direction will be to generalise the likelihood from pairwise comparisons
to comparisons involving $k$ items~\citep{pan2018stagewise}
or best--worst scaling~\citep{kiritchenko2017best}
to provide scalable Bayesian methods for other forms of comparative preference data.
%and investigate the possibility of integrating deep generative models 
%for learning feature representations from input data.
%the models can readily be adapted to regression or classification tasks by swapping out the preference likelihood, resulting in 
%different values for $\bs G$ and $\bs H$.
% Preference elicitation using information theoretic methods.


%%%%%%%%%%%%%%%%%%%%%%%%%%%%%%%%%%%%%%%%%%%%%%%%%%%%%%%%%%%%%%%%%%%%%%%%%%%%%%%%

% use section* for acknowledgment
\section*{Acknowledgments}

\bibliographystyle{spbasic}
\bibliography{simpson_scalable_bayesian_pref_learning_from_crowds}

\appendix

\section{Variational Lower Bound for GPPL}
\label{sec:vb_eqns}

Due to the non-Gaussian likelihood, Equation \ref{eq:plphi},
the posterior distribution over $\bs f$ contains intractable integrals:
\begin{flalign}
p(\bs f | \bs y, k_{\theta}, \alpha_0, \alpha_0) = 
\frac{\int \prod_{p=1}^P \Phi(z_p) \mathcal{N}(\bs f; \bs 0, \bs K_{\theta}/s) 
\mathcal{G}(s; \alpha_0, \beta_0) d s}{\int \int \prod_{p=1}^P \Phi(z_p) \mathcal{N}(\bs f'; \bs 0, \bs K_{\theta}/s) 
\mathcal{G}(s; \alpha_0, \beta_0) d s d f' }.
\label{eq:post_single}
\end{flalign}
We can derive a variational lower bound as follows, beginning with an approximation that does not use inducing points:
\begin{flalign}
\mathcal{L} = \sum_{p=1}^{P} \mathbb{E}_{q(\bs f)}\!\left[ \ln p\left( y_p| f(\bs x_{a_p}), f(\bs x_{b_p}) \right) \right]
\!+ \mathbb{E}_{q(\bs f),q(s)}\!\left[ \ln \frac{p\left( \bs f | \bs 0, \frac{\bs K}{s} \right)}
{q\left(\bs f\right)} \right] 
\!+ \mathbb{E}_{q(s)}\!\left[ \ln \frac{p\left( s | \alpha_0, \beta_0\right)}{q\left(s \right)} \right] &&
\label{eq:vblb}
\end{flalign}
Writing out the expectations in terms of the variational parameters, we get:
\begin{flalign}
\mathcal{L} = &\; \mathbb{E}_{q(\bs f)}\Bigg[ \sum_{p=1}^{P} y_p \ln\Phi(z_p) + (1-y_p) \left(1-\ln\Phi(z_p)\right) \Bigg] 
+ \mathbb{E}_{q(\bs f)}\left[\ln \mathcal{N}\left(\hat{\bs f}; \bs\mu, \bs K/\mathbb{E}[s] \right) \right]
\nonumber\\
& 
- \mathbb{E}_{q(\bs f}\left[\ln\mathcal{N}\left(\bs f; \hat{\bs f}, \bs C \right) \right]
 + \mathbb{E}_{q(s)}\left[ \ln\mathcal{G}\left( s; \alpha_0, \beta_0\right) - \ln\mathcal{G}\left(s; \alpha, \beta \right) \right]  
  \nonumber \\
 =&\;  \sum_{p=1}^{P} y_p \mathbb{E}_{q(\bs f)}[
\ln\Phi(z_p) ]+ (1-y_p) \left(1-\mathbb{E}_{q(\bs f)}[\ln\Phi(z_p)] \right) \Bigg]  \nonumber\\
 & -\frac{1}{2}\left\{
 %N \ln 2\pi + 
 \ln | \bs K | - \mathbb{E}[\ln s] + \mathrm{tr}\left( \left(\hat{\bs f}^T \hat{\bs f} + \bs C\right)\bs K^{-1} \right)
% - N \ln 2\pi 
- \ln |\bs C| - N
 \right\}  \nonumber \\
 & - \Gamma(\alpha_0) + \alpha_0(\ln \beta_0) + (\alpha_0-\alpha)\mathbb{E}[\ln s] + \Gamma(\alpha) + (\beta-\beta_0) \mathbb{E}[s] - \alpha \ln \beta. \\
\end{flalign}
The expectation over the likelihood can be computed using numerical integration. 
%I can no longer follow this... I think that the expectation containing Phi(z) looks dodgy as there should be a term
%relating to the variance of z. 
%We compute this by observing that the probit likelihood can be written as 
%a product of two terms:
%\begin{flalign}
%\mathcal{L} 
% =&\; \mathbb{E}_{q(\bs f)}\Bigg[ \sum_{p=1}^{P} y_p \ln\Phi(z_p) + y(b_p,a_p) \left(1-\ln\Phi(z_p)\right) \Bigg] \nonumber\\
% & -\frac{1}{2}\left\{
% \ln | \bs K | - \mathbb{E}[\ln s] + \mathrm{tr}\left( \left(\hat{\bs f}^T \hat{\bs f} + \bs C\right)\bs K^{-1} \right)
%- \ln |\bs C| - N
% \right\}  \nonumber \\
% & - \Gamma(\alpha_0) + \alpha_0(\ln \beta_0) + (\alpha_0-\alpha)\mathbb{E}[\ln s] + \Gamma(\alpha) + (\beta-\beta_0) \mathbb{E}[s] - \alpha \ln \beta. \\
%\end{flalign}
%
%We now replace the likelihood with a Gaussian approximation:
%\begin{flalign}
%\mathcal{L}_1 \approx \mathcal{L}_2 & = \mathbb{E}_{q(\bs f)}\left[ \mathcal{N}( \bs y | \Phi(\bs z), \bs Q) \right]
% + \ln \mathcal{N}\left(\hat{\bs f}; \bs\mu, \bs K/\mathbb{E}[s] \right) - \ln\mathcal{N}\left(\hat{\bs f}; \hat{\bs f}, \bs C \right) 
%&\nonumber\\
%& + \mathbb{E}_{q(s)}\left[ \ln\mathcal{G}\left( s; \alpha_0, \beta_0\right) - \ln\mathcal{G}\left(s; \alpha, \beta \right) \right] \nonumber&\\
%& =  - \frac{1}{2} \left\{ L \ln 2\pi + \ln |\bs Q| - \ln|\bs C| 
% + \ln|\bs K| - \mathbb{E}[\ln s] + (\hat{\bs f} - \bs\mu)\mathbb{E}[s]\bs K^{-1}
%(\hat{\bs f} - \bs\mu) \right. \nonumber &\\
%& \left. + \mathbb{E}_{q(\bs f)}\left[ (\bs y - \Phi(\bs z))^T \bs Q^{-1} (\bs y - \Phi(\bs z)) \right] \right\}
% - \Gamma(\alpha_0) + \alpha_0(\ln \beta_0) + (\alpha_0-\alpha)\mathbb{E}[\ln s] \nonumber&\\
%& + \Gamma(\alpha) + (\beta-\beta_0) \mathbb{E}[s] - \alpha \ln \beta,  &
%\end{flalign}
%where $\mathbb{E}[s] = \frac{\alpha}{\beta}$, $\mathbb{E}[\ln s] = \Psi(2\alpha) - \ln(2\beta)$,
%$\Psi$ is the digamma function and $\Gamma()$ is the gamma function, 
%Finally, we use a Taylor-series linearisation to make the remaining expectation tractable:
%\begin{flalign}
%\mathcal{L}_2 & \approx \mathcal{L}_3 = - \frac{1}{2} \left\{ L \ln 2\pi + \ln |\bs Q| - \ln|\bs C| \right.
% \left. + \ln|\bs K/\mathbb{E}[s]| + (\hat{\bs f} - \bs\mu)\mathbb{E}[s]\bs K^{-1}(\hat{\bs f} - \bs\mu) \right. \nonumber&&\\
% & \left. + (\bs y - \Phi(\hat{\bs z}))^T \bs Q^{-1} (\bs y - \Phi(\hat{\bs z}))\right\}
% - \Gamma(\alpha_0) + \alpha_0(\ln \beta_0) + (\alpha_0-\alpha)\mathbb{E}[\ln s] \nonumber&&\\
%& + \Gamma(\alpha) + (\beta-\beta_0) \mathbb{E}[s] - \alpha \ln \beta. &&
%\label{eq:vblb_terms} 
%\end{flalign}
Now we can introduce the sparse approximation to obtain the bound in Equation \ref{eq:lowerbound}:
\begin{flalign}
\mathcal{L} \approx \; & \mathbb{E}_{q(\bs f)}[\ln p(\bs y | \bs f)]
 + \mathbb{E}_{q(\bs f_m), q(s)}[\ln p(\bs f_m, s | \bs K, 
\alpha_0, \beta_0)] - \mathbb{E}_{q(\bs f_m)}[\ln q(\bs f_m)] 
- \mathbb{E}_{q(s)}[\ln q(s) ] & \nonumber \\ 
=\; & \sum_{p=1}^P \mathbb{E}_{q(\bs f)}[\ln p(y_p | f(\bs x_{a_p}), f(\bs x_{b_p}) )] - \frac{1}{2} \bigg\{ \ln|\bs K_{mm}| - \mathbb{E}[\ln s] - \ln|\bs S| - M
\nonumber &\\
& + \hat{\bs f}_m^T\mathbb{E}[s] \bs K_{mm}^{-1}\hat{\bs f}_m + 
\textrm{tr}(\mathbb{E}[s] \bs K_{mm}^{-1} \bs S) \bigg\}  + \ln\Gamma(\alpha) - \ln\Gamma(\alpha_0)  + \alpha_0(\ln \beta_0) \nonumber\\
& + (\alpha_0-\alpha)\mathbb{E}[\ln s]+ (\beta-\beta_0) \mathbb{E}[s] - \alpha \ln \beta, &
\label{eq:full_L_singleuser}
\end{flalign}
where the terms relating to $\mathbb{E}[p(\bs f | \bs f_m) - q(\bs f)]$ cancel.
% Without stochastic sampling, the variational factor $\ln q(\bs f_m)$ is given by:
% \begin{flalign}
% \ln q(\bs f_m) &= \ln \mathcal{N}\left(\bs y; \tilde{\Phi}(\bs z), \bs Q\right)]
% + \ln\mathcal{N}\left(\bs f_m; \bs 0, \bs K_{mm}/\mathbb{E}\left[s\right]\right)  + \textrm{const}, \nonumber \\
% %&= \ln \int \mathcal{N}(\bs y - 0.5; \bs G \bs f, \bs Q) 
% %\mathcal{N}(\bs f; \bs A \bs f_m, \bs K - \bs A \bs K_{nm}^T) & \nonumber\\
% %& \hspace{3.2cm} \mathcal{N}(\bs f_m; \bs 0, \bs K_{mm}\mathbb{E}[1/s]) \textrm{d} \bs f + \textrm{const} & \nonumber\\
%  & = \ln \mathcal{N}(\bs f_m; \hat{\bs f}_m, \bs S ), \\
% \bs S^{-1} &= \bs K^{-1}_{mm}/\mathbb{E}[s] + \bs A^T \bs G^T \bs Q^{-1} \bs G \bs A, \label{eq:S}\\
% \hat{\bs f}_m &= \bs S \bs A^T \bs G^T \bs Q^{-1} (\bs y - \Phi(\mathbb{E}[\bs z]) + \bs G \mathbb{E}[\bs f] ). %\label{eq:fhat_m}
% \end{flalign}
For crowdGPPL, our approximate variational lower bound is:
\begin{flalign}
\mathcal{L}_{cr} & = \label{eq:lowerbound_crowd_full}
\sum_{p=1}^P \ln p(y_p | \hat{\bs v}_{\!.,a_p}^T \! \hat{\bs w}_{\!.,j_p} \!+ \hat{t}_{a_p}\!,
 \hat{\bs v}_{\!.,b_p}^T\! \hat{\bs w}_{\!.,j_p} \!+ \hat{t}_{b_p})
- \frac{1}{2} 
\Bigg\{  \sum_{c=1}^C \bigg\{  
 \ln|\bs K_{mm}| 
\! - \! \mathbb{E}\left[\ln s^{(v)}_c\right]
\! - \! \ln|\bs S^{(v)}_{c}|  
& \nonumber \\
& 
\! - \! M_{\mathrm{items}} 
+ \hat{\bs v}_{m,c}^T \mathbb{E}\left[s^{(v)}_c\right] \bs K_{mm}^{-1}\hat{\bs v}_{m,c} 
+ \textrm{tr}\left(\mathbb{E}\left[s_c^{(v)}\right] \bs K_{mm}^{-1} \bs S_{v,c}\right) 
+ \ln|\bs L_{mm}|
- \mathbb{E}\left[\ln s^{(w)}_c \right]
& \nonumber \\
&  
- \ln|\bs \Sigma_{c}| 
\! - \! M_{\mathrm{users}}
  + \hat{\bs w}_{m,c}^T \mathbb{E}\left[ s_c^{(w)} \right] \bs L_{mm}^{-1}\hat{\bs w}_{m,c} 
+ \textrm{tr}\left( \mathbb{E}\left[ s_c^{(w)} \right] \bs L_{mm}^{-1} \bs \Sigma_{c} \right)
+ \ln|\bs K_{mm}|   
\bigg\}
& \nonumber \\
&
 - \mathbb{E}\left[\ln s^{(t)} \right]  
- \ln|\bs S^{(t)}| 
- M_{\mathrm{items}} 
+ \hat{\bs t}^T \mathbb{E}\left[s^{(t)}\right] \bs K_{mm}^{-1} \hat{\bs t} 
+ \textrm{tr}\left(\mathbb{E}\left[s^{(t)}\right] \bs K_{mm}^{-1} \bs S^{(t)} \right)
\Bigg\} 
& \nonumber \\
&
+ \sum_{c=1}^C \bigg\{ 
\ln\Gamma\left(\alpha_0^{(v)}\right)  + \alpha_0^{(v)}\left(\ln \beta^{(v)}_0\right)
+ \ln\Gamma\left(\alpha_c^{(v)}\right) + \left(\alpha_0^{(v)} - \alpha_c^{(v)}\right)\mathbb{E}\left[\ln s^{(v)}_c\right]
 & 
\nonumber \\ 
&
+ \left(\beta_c^{(v)} - \beta^{(v)}_0\right) \mathbb{E}[s^{(v)}_c] - \alpha_c^{(v)} \ln \beta_c^{(v)} 
+ \ln\Gamma\left(\alpha_0^{(w)}\right)  + \alpha_0^{(w)}\left(\ln \beta^{(w)}_0\right)
+ \ln\Gamma\left(\alpha_c^{(w)}\right) 
 & 
\nonumber \\ 
&
+ \left(\alpha_0^{(w)} - \alpha_c^{(w)}\right)\mathbb{E}\left[\ln s^{(w)}_c\right]
+ \left(\beta_c^{(w)} - \beta^{(w)}_0\right) \mathbb{E}[s^{(w)}_c] - \alpha_c^{(w)} \ln \beta_c^{(w)} \bigg\}
 + \ln\Gamma\left(\alpha_0^{(t)}\right)  
 & 
\nonumber \\ 
& 
 + \alpha_0^{(t)} \! \left(\ln \beta^{(t)}_0\right)
+  \ln\Gamma\left(\alpha^{(t)}\right) + \left( \! \alpha^{(t)}_0 \!-\! \alpha^{(t)} \! \right)\mathbb{E}\left[\ln s^{(t)}\right]
\! + \!  \left(\! \beta^{(t)} \!-\! \beta^{(t)}_0 \! \right) \mathbb{E}\left[s^{(t)}\right] \! - \!  \alpha^{(t)} \! \ln \beta^{(t)}
. &
\end{flalign}

\section{Posterior Parameters for Variational Factors in CrowdGPPL}
\label{sec:post_params}

For the latent item components, the posterior precision estimate for $\bs S^{-1}_{v,c}$ at iteration $i$ is given by:
\begin{flalign}
\left( \! \bs S^{(v)}_{c,i} \! \right)^{\!-1} \!\!\!\! = (1-\rho_i) \left( \! \bs S^{(v)}_{c,i-1} \! \right)^{\!-1} 
\!\!\!\! +\! \rho_i\left( \! \bs K^{-1}_{mm}\mathbb{E}\left[ \! s^{(v)}_c \! \right] \!
+ \! \pi_i \bs A_{v,i}^T \bs G_i^T \textrm{diag}\left(\hat{\bs w}_{c,\bs u}^2 \!\! + \bs\Sigma_{c,\bs u,\bs u}\right)
\bs Q_i^{-1} \bs G_i \bs A_{v,i}
\! \right) \!, &&
\label{eq:Sv}
\end{flalign}
where $\bs A_{i} = \bs K_{im} \bs K_{mm}^{-1}$, 
$\hat{\bs w}_{c}$ and $\bs\Sigma_{c}$ are the variational mean and covariance of 
the $c$th latent user component (defined below in Equations \ref{eq:what} and \ref{eq:Sigma}),
and ${\bs u} = \{ u_p \forall p \in P_i \}$ is the vector of user indexes in the sample of observations.
%The term $\textrm{diag}(\hat{\bs w}_{c,\bs j}^2 + \bs\Sigma_{c,\bs j})$ 
%scales the diagonal observation precision, $\bs Q^{-1}$, by the latent user factors.
We use $\bs S_{v,c}^{-1}$ to compute the means for each row of $\bs V_m$:
\begin{flalign}
\hat{\bs v}_{m,c,i} = \bs S^{(v)}_{c,i}\left( 
(1-\rho_i) \left( \bs S^{(v)}_{c,i-1} \right)^{-1} \hat{\bs v}_{m,c,i-1} + \rho_i \pi_i \bigg(
\bs S^{(v)}_{c,i} \bs A_{i}^T \bs G_i^T \textrm{diag}(\hat{\bs w}_{c,\bs u}) \bs Q_i^{-1} \right. && \nonumber \\
 \Big(\bs y_i - \Phi(\mathbb{E}[\bs z_i]) + \sum_{j=1}^U \bs H^{(i)}_{j}(\hat{\bs v}_c^T \hat{\bs w}_{c,j})\Big) \bigg) \bigg), &&
\label{eq:hatv}
\end{flalign}
where $\bs H^{(i)}_{j} \in |P_i| \times N$ contains partial derivatives of the pairwise likelihood
with respect to $F_{n,j} = \hat{v}_{c,n} \hat{w}_{c,j}$, 
with elements given by:
\begin{flalign}
H^{(i)}_{j,p,n} & = \Phi(\mathbb{E}[z_p])(1 - \Phi(\mathbb{E}[z_p])) (2y_p - 1)( [n = a_p] - [n = b_p]) [j = u_p]. &
\end{flalign}

For the consensus, the precision and mean are updated according to the following:
%This is needed to replace $\bs G$ in the single-user model, since the vector of utilities,
%$\bs f$, has been replaced by the matrix $\bs F$, where each column of $\bs F$ corresponds to a single user.
\begin{flalign}
\left( \bs S^{(t)}_i \right)^{-1} = \;\;& (1-\rho_i) \left( \bs S^{(t)}_{i-1} \right) + \rho_i\bs K^{-1}_{mm}\mathbb{E}\left[s^{(t)}\right] 
+ \rho_i \pi_i \bs A_{i}^T \bs G_i^T \bs Q_i^{-1} \bs G_i \bs A_{i} & \label{eq:St}\\
\hat{\bs t}_{m,i} = \;\;& \bs S^{(t)}_{i}\left(
(1 - \rho_i) \left( \bs S^{(t)}_{i-1} \right)^{-1}\hat{\bs t}_{m,i-1}  
 + \rho_i \pi_i \bs A_{i}^T \bs G_i^T \bs Q_i^{-1}
\left(\bs y_i - \Phi(\mathbb{E}[\bs z_i]) + \bs G_i \bs A_{i} \hat{\bs t}_{i} \right) \right). & \label{eq:hatt}
\end{flalign}

For the latent user components, the SVI updates for the parameters are:
\begin{flalign}
\bs \Sigma^{-1}_{c,i} = \; & (1-\rho_i)\bs \Sigma^{-1}_{c,i-1}
+ \rho_i\bs L^{-1}_{mm} \mathbb{E} \left[ s_c^{(w)} \right]
+ \rho_i \pi_i \bs A_{w,i}^T \bigg( \sum_{p \in P_i} \bs H^{(i)T}_{.,p} \textrm{diag}\left(\hat{\bs v}_{c,\bs a}^2 \right. 
&\nonumber \\
& \left. + \bs S^{(v)}_{c,\bs a, \bs a} + 
\hat{\bs v}_{c,\bs b}^2 + \bs S^{(v)}_{c,\bs b, \bs b}  
- 2\hat{\bs v}_{c,\bs a}\hat{\bs v}_{c,\bs b} - 2\bs S^{(v)}_{c,\bs a, \bs b} \right) \bs Q_i^{-1} \sum_{p \in P_i} \bs H^{(i)}_{.,p} \bigg) \bs A_{w,i} & \label{eq:Sigma} \\
%%%%
\hat{\bs w}_{m,c,i} = & \; \bs \Sigma_{c,i} \bigg( (1 - \rho_i)\bs \Sigma_{c,i-1}\hat{\bs w}_{m,c,i-1} + 
 \rho_i \pi_i \bs A_{w,i}^T \sum_{p \in P_i} \bs H^{(i)}_{.,p}
\left( \textrm{diag}(\hat{\bs v}_{c,\bs a}) \right. & \nonumber  \\
& \left. - \textrm{diag}(\hat{\bs v}_{c,\bs b}) \right) \bs Q_i^{-1} 
\bigg(\bs y_i - \Phi(\mathbb{E}[\bs z_i]) + \sum_{j=1}^U \bs H^{(i)}_u (\hat{\bs v}_c^T \hat{\bs w}_{c,j})\bigg) \bigg), & \label{eq:what}
\end{flalign}
where the subscripts $\bs a = \{ a_p \forall p \in P_i \}$
and  $\bs b = \{b_p \forall p \in P_i \}$ are lists of indices to the first and 
second items in the pairs, respectively, and $\bs A_{w,i} = \bs L_{im} \bs L_{mm}^{-1}$.


\begin{algorithm}[h]
 \KwIn{ Pairwise labels, $\bs y$, training item features, $\bs x$, training user features $\bs u$, 
 test item features $\bs x^*$, test user features $\bs u^*$}
 \nl Compute kernel matrices $\bs K$, $\bs K_{mm}$ and $\bs K_{nm}$ given $\bs x$\;
 \nl Compute kernel matrices $\bs L$, $\bs L_{mm}$ and $\bs L_{nm}$ given $\bs u$\;
 \nl Initialise $\mathbb{E} \!\left[s^{(t)}\!\right]$, $\mathbb{E}\!\left[s^{(v)}_c\!\right]\forall c$, 
 $\mathbb{E}\!\left[s^{(w)}_c\!\right]\forall c$, $\mathbb{E}[\bs V]$, $\hat{\bs V}_m$,
 $\mathbb{E}[\bs W]$, $\hat{\bs W}_m$,
  $\mathbb{E}[\bs t]$, $\hat{\bs t}_m$ 
  to prior means\;
 \nl Initialise $\bs S_{v,c}\forall c$ and $\bs S_t$ to prior covariance $\bs K_{mm}$\;
\nl Initialise $\bs S_{w,c}\forall c$ to prior covariance $\bs L_{mm}$\;
 \While{$\mathcal{L}$ not converged}
 {
 \nl Select random sample, $\bs P_i$, of $P$ observations\;
 \While{$\bs G_i$ not converged}
  {
  \nl Compute $\bs G_i$ given $\mathbb{E}[\bs F_i]$ \;
  \nl Compute $\hat{\bs t}_{m,i}$ and $\bs S_{i}^{(t)}$ \;
  \For{c in 1,...,C}
  {
    \nl Update $\mathbb{E}[\bs F_i]$ \;
    \nl Compute $\hat{\bs v}_{m,c,i}$ and $\bs S_{i,c}^{(v)}$ \;
    \nl Update $q\left(s^{(v)}_c\right)$, compute $\mathbb{E}\left[s^{(v)}_c\right]$ and 
    $\mathbb{E}\left[\ln s^{(v)}_c\right]$\; 
    \nl Update $\mathbb{E}[\bs F_i]$ \;
    \nl Compute $\hat{\bs W}_{m,c,i}$ and $\bs \Sigma_{i,c}$ \;    
    \nl Update $q\left(s^{(w)}_c\right)$, compute $\mathbb{E}\left[s^{(w)}_c\right]$ 
    and $\mathbb{E}\left[\ln s^{(w)}_c\right]$\;
  }
  \nl Update $\mathbb{E}[\bs F_i]$ \;
 }
 \nl Update $q\left(s^{(t)}\right)$, compute $\mathbb{E}\left[s^{(t)}\right]$ and
 $\mathbb{E}\left[\ln s^{(t)}\right]$ \;
 }
\nl Compute kernel matrices for test items, $\bs K_{**}$ and $\bs K_{*m}$, given $\bs x^*$ \;
\nl Compute kernel matrices for test users, $\bs L_{**}$ and $\bs L_{*m}$, given $\bs u^*$ \;
\nl Use converged values of $\mathbb{E}[\bs F]$ and $\hat{\bs F}_m$ to estimate
posterior over $\bs F^*$ at test points \;
\KwOut{ Posterior mean of the test values, $\mathbb{E}[\bs F^*]$ and covariance, $\bs C^*$ }
\vspace{0.5cm}
\caption{The SVI algorithm for crowdGPPL.}
\label{al:crowdgppl}
\end{algorithm}

\section{Predictions with CrowdGPPL}
\label{sec:predictions}

The means, item covariances and user variance required for predictions with crowdGPPL (Equation \ref{eq:predict_crowd})
 are defined as follows:
\begin{flalign}
\hat{\bs t}^* = \bs K_{*m} \bs K^{-1}_{mm} \hat{\bs t}_{m}, \hspace{1cm} 
& \bs C^{(t)*} \!= \frac{\bs K_{**}}{\mathbb{E}\left[s^{(t)}\right]} + \bs A_{*m}\left(\bs S^{(t)} \!-\! \bs K_{mm}\right) 
\bs A_{*m}^T, 
\label{eq:tstar} & \\
\hat{\bs v}_{c}^* = \bs K_{*m} \bs K^{-1}_{mm} \hat{\bs v}_{m,c}, \hspace{1cm} 
& \bs C^{(v)*}_{c} \!= \frac{\bs K_{**}}{\mathbb{E}\left[s^{(v)}_c\right ]} + \bs A_{*m} \left(\bs S^{(v)}_{c} 
\!\!-\! \bs K_{mm} \right) \bs A_{*m}^T  & \\
\hat{\bs w}_{c}^* = \bs L_{*m} \bs L^{-1}_{mm} \hat{\bs w}_{m,c}, \hspace{1cm}
& \omega_{c,u}^* = 1/ \mathbb{E}\left[s^{(w)}_c \right] + \bs A^{(w)}_{um}(\bs \Sigma_{w,c} - \bs L_{mm}) \bs A^{(w)T}_{um} & \label{eq:omegastar}
\end{flalign}
where  $\bs A_{*m}=\bs K_{*m}\bs K_{mm}^{-1}$,
$\bs A^{(w)}_{um}=\bs L_{um}\bs L_{mm}^{-1}$ and $\bs L_{um}$ is the covariance between user $u$ and the inducing 
users.

% \section{Converged Lower Bound Derivatives}
% \label{sec:gradients}
% % The gradient of $\mathcal{L}_3$ with respect to the lengthscale, $l_d$, is as follows:
% % \begin{flalign}
% % \nabla_{l_d} \mathcal{L}_3 & =  - \frac{1}{2} \left\lbrace 
% % \frac{\partial \ln|\bs K/\mathbb{E}[s]|}{\partial l_d} - \frac{\partial \ln|\bs C|}{\partial l_d} 
% % \nonumber \right.
% %  \left.  - (\hat{\bs f}-\bs\mu)\mathbb{E}[s] \frac{\partial K^{-1}}{\partial l_d} (\hat{\bs f}-\bs\mu)
% % \right\rbrace \nonumber & \\
% % %& = \frac{1}{2} \mathbb{E}[s] \left\lbrace \frac{\partial \ln |\bs C \bs K^{-1}|}{\partial l_d}
% % %\right. \\
% % %& \left.  - (\hat{\bs f}-\bs\mu) \bs K^{-1} \frac{\partial \bs K}{\partial l_d} \bs K^{-1} (\hat{\bs f}-\bs\mu)
% % %\right\rbrace  \nonumber \\
% % & =  -\frac{1}{2} \left\lbrace  \frac{\partial \ln | \frac{1}{\mathbb{E}[s]}\bs K \bs C^{-1} |}{\partial l_d} \right. 
% % \left.  + \mathbb{E}[s] (\hat{\bs f}-\bs\mu) \bs K^{-1} \frac{\partial \bs K}{\partial l_d} \bs K^{-1} (\hat{\bs f}-\bs\mu)
% % \right\rbrace   &
% % %& =  - \frac{1}{2} \left\lbrace \frac{\partial \ln|\bs K/s| }{\partial l_d} + \frac{\partial \ln |\bs K^{-1}s + \bs G\bs Q^{-1}\bs G^T|}{\partial l_d}
% % %\right. \\
% % %& \left.  - \mathbb{E}[s] (\hat{\bs f}-\bs\mu) \bs K^{-1} \frac{\partial \bs K}{\partial l_d} \bs K^{-1} (\hat{\bs f}-\bs\mu)
% % %\right\rbrace  \nonumber\\
% % %& =  -\frac{1}{2} \left\lbrace \frac{\partial \ln |\bs I + \bs K/s\bs G\bs Q^{-1}\bs G^T|}{\partial l_d}
% % %\right. \\
% % %& \left.  - \mathbb{E}[s] (\hat{\bs f}-\bs\mu) \bs K^{-1} \frac{\partial \bs K}{\partial l_d} \bs K^{-1} (\hat{\bs f}-\bs\mu)
% % %\right\rbrace  \nonumber
% % \end{flalign}
% % Using the fact that $\ln | A | = \mathrm{tr}(\ln A)$, $\bs C = \left[\bs K^{-1} - \bs G \bs Q^{-1} \bs G^T \right]^{-1}$, and $\bs C = \bs C^{T}$, we obtain:
% % \begin{flalign}
% % \nabla_{l_d} \mathcal{L}_3 & =  -\frac{1}{2} \mathrm{tr}\left(\left(\mathbb{E}[s]\bs K^{-1}\bs C\right) \bs G\bs Q^{-1}\bs G^T \frac{\partial \bs K}{\partial l_d}
% % \right)
% %  + \frac{1}{2}\mathbb{E}[s] (\hat{\bs f}-\bs\mu) \bs K^{-1} \frac{\partial \bs K}{\partial l_d} \bs K^{-1} (\hat{\bs f}-\bs\mu)  \nonumber\\ 
% % & =  -\frac{1}{2} \mathrm{tr}\left(\left(\mathbb{E}[s]\bs K^{-1}\bs C\right)
% % \left(\bs C^{-1} - \bs K^{-1}/\mathbb{E}[s]\right) \frac{\partial \bs K}{\partial l_d}
% % \right) 
% % + \frac{1}{2}\mathbb{E}[s] (\hat{\bs f}-\bs\mu) \bs K^{-1} \frac{\partial \bs K}{\partial l_d} \bs K^{-1} (\hat{\bs f}-\bs\mu).  \label{eq:gradient_ls}
% % \end{flalign}
% % Assuming a product over kernels for each feature, $\bs K=\prod_{d=1}^{D} \bs K_d$, we can compute the kernel gradient 
% % as follows for the Mat\'ern $\frac{3}{2}$ kernel function:
% % \begin{flalign}
% % \frac{\partial \bs K}{\partial l_d} & = \prod_{d'=1,d'\neq d}^D K_{d} \frac{\partial K_{l_d}}{\partial l_d} \\
% % \frac{\partial K_{l_d}}{\partial l_d} & = \frac{3\bs |\bs x_d - \bs x_d'|^2}{l_d^3} \exp\left( - \frac{\sqrt{3} \bs |\bs x_d - \bs x_d'|}{l_d} \right)
% % \label{eq:kernel_der}
% % \end{flalign}
% % where $|\bs x_d - \bs x_d'|$ is the distance between input points.
%
% When $\mathcal{L}$ has converged to a maximum, 
% $\nabla_{l_{\! d}} \mathcal{L}$ simplifies to:
% \begin{flalign}
%  &\nabla_{\!l_{\! d}} \mathcal{L} \longrightarrow 
% \frac{1}{2} \mathrm{tr}\!\left(\! \left(
% \mathbb{E}[s](\hat{\bs f}_{\! m} \hat{\bs f}_{\! m}^T + \bs S^T)\bs K_{\! mm}^{-1} \! -  \bs I \! \right)
%  \!\frac{\partial \bs K_{\! mm}}{\partial l_d} \bs K_{\! mm}^{-1} \right) \!. &
% \label{eq:gradient_single}
% \end{flalign}
% For crowdGPPL, assuming that $\bs V$ and $\bs t$ have the same kernel function,
% the gradient
% %The gradients with respect to the length-scale, $l_{w,d}$,
% for the $d$th item feature is given by:
% \begin{flalign}
%  &\nabla_{l_{ v,d}} \mathcal{L}_{cr} \longrightarrow
% %  \sum_{c=1}^{C} \bigg\{ \mathbb{E}[s_c] 
% %  \hat{\bs v}_{ m,c}^T \bs K_{ mm,v}^{-1} 
% % \frac{\partial \bs K_{ mm,v}}{\partial l_{w,d}} \bs K_{ mm,v}^{-1} \hat{\bs v}_{ m,c} 
% %  + & \nonumber \\
% %  & \mathrm{tr}\left( \left(
% % \mathbb{E}[s_c]\bs S_{v,c}^T\bs K_{ mm,v}^{-1}  - \frac{1}{2} \bs I  \right)
% %  \frac{\partial \bs K_{ mm,v}}{\partial l_{w,d}} \bs K_{ mm,v}^{-1} \right) \bigg\}
% %  + & \nonumber \\
% %  & \mathbb{E}[s_t] 
% %  \hat{\bs t}_{ m}^T \bs K_{ mm,t}^{-1} 
% % \frac{\partial \bs K_{ mm,t}}{\partial l_{w,d}} \bs K_{ mm,t}^{-1} \hat{\bs t}_{ m} 
% %  + \mathrm{tr}\left( \left(
% % \mathbb{E}[s_t]\bs S_{t}^T\bs K_{ mm,t}^{-1}  - \frac{1}{2} \bs I  \right)
% %  \frac{\partial \bs K_{ mm,t}}{\partial l_{w,d}} \bs K_{ mm,t}^{-1} \right)
% %  & \nonumber \\
% % & = 
% \frac{1}{2} \mathrm{tr}\left( \left( \sum_{c=1}^{C} \mathbb{E}[s_c] \left\{ \hat{\bs v}_{m,c} 
%  \hat{\bs v}_{m,c}^T + \bs S_{v,c}^T \right\}
%  \bs K_{ mm,v}^{-1}  - C\bs I  \right)
%  \frac{\partial \bs K_{ mm,v}}{\partial l_{w,d}} \right.
%  & \nonumber \\
%  & \left.  \bs K_{ mm,v}^{-1} \right) + \frac{1}{2}\mathrm{tr}\left( \left(
% \mathbb{E}[s_t](\hat{\bs t}_{ m}\hat{\bs t}_{ m}^T + \bs S_{t}^T) \bs K_{ mm,t}^{-1}  
% - \bs I  \right)
%  \frac{\partial \bs K_{ mm,t}}{\partial l_{w,d}} \bs K_{ mm,t}^{-1} \right)
% .&
% \label{eq:gradient_crowd_items}
% \end{flalign}
% % If different kernels are used for different components, then the equation above can be modified to
% % simply sum over terms relating to the components with a shared kernel function. 
% The gradients for the $d$th user feature length-scale, $l_{w,d}$, follows the same form:
% \begin{flalign}
%  &\nabla_{l_{w,d}} \mathcal{L}_{cr} \!\!\!\longrightarrow \frac{1}{2} 
%  \!\mathrm{tr}\left( \!\left( \sum_{c=1}^{C} \left\{ \hat{\bs w}_{m,c} \hat{\bs w}_{m,c}^T \!+
% \bs \Sigma_c^T\right\} \!\bs K_{mm,w}^{-1} \! - C\bs I  \right)
%  \frac{\partial \bs K_{mm,w}}{\partial l_{w,d}} \bs K_{mm,w}^{-1} \!\right) \!. &
% \label{eq:gradient_crowd_users}
% \end{flalign}
%
% % When combining kernel functions for each features using a product,
% % as in Equation \ref{eq:kernel}, the partial derivative of the covariance matrix $\bs K_{mm}$ with respect to 
% % $l_d$ is given by:
% % \begin{flalign}
% % \frac{\partial \bs K_{mm}}{\partial l_d} 
% % & = \frac{\bs K_{mm}}{\bs K_{d}}
% % \frac{ \bs K_{d}(|\bs x_{mm,d}, \bs x'_{mm,d})}{\partial l_d} \nonumber ,\\
% % \end{flalign}
% The partial derivative of the covariance matrix $\bs K_{mm}$ with respect to 
% $l_d$ depends on the choice of kernel function. 
% The Mat\`ern $\frac{3}{2}$ function is a widely-applicable, differentiable kernel function 
% that has been shown empirically to outperform other well-established kernels 
% such as the squared exponential, and makes weaker assumptions of smoothness of 
% the latent function~\citep{rasmussen_gaussian_2006}. 
% It is defined as:
% \begin{flalign}
% k_d\left(\frac{|x_d - x_d'|}{l_d} \right) = \left(1 + \frac{\sqrt{3} | x_d - x_d'|}{l_d}\right) 
% \exp \left(- \frac{\sqrt {3} | x_d - x_d'|}{l_d}\right).
% \end{flalign}
% %For the Mat\`ern $\frac{3}{2}$ kernel,  
% Assuming that the kernel functions for each feature, $k_d$, are combined using
% a product, as in Equation \ref{eq:kernel}, 
% the partial derivative $\frac{\partial \bs K_{mm}}{\partial l_d}$ is a matrix, where each 
% entry, $i,j$,  is defined by:
% \begin{flalign}
% & \frac{\partial K_{mm,ij}}{\partial l_d} = 
% \prod_{d'=1, d' \neq d}^D k_{d'}\left(\frac{|x_{d'} - x_{d'}'|}{l_{d'}}\right)
% \frac{3 (\bs x_{i,d} - \bs x_{j,d})^2}{l_d^3} \exp\left( - \frac{\sqrt{3} \bs |\bs x_{i,d} - \bs x_{j,d}|}{l_d} \right), &
% \label{eq:kernel_der}
% \end{flalign}
% where we assume the use of Equation\ref{eq:kernel} to combine kernel 
% functions over features using a product.
%
%
% To make use of Equations \ref{eq:gradient_single} to \ref{eq:kernel_der},
% we nest the variational algorithm defined in Section \ref{sec:inf} inside
% an iterative gradient-based optimization method.
% Optimization then begins with an initial guess for all length-scales, $l_d$,
% such as the median heuristic.
% Given the current values of $l_d$, the optimizer (e.g. L-BFGS-B)
% runs the VB algorithm to convergence, 
% computes $\nabla_{l_{\! d}} \mathcal{L}$,
% then proposes a new candidate value of $l_d$.
% The process repeats until the optimizer converges or reaches a maximum number 
% of iterations, and returns the value of $l_d$ that maximized $\mathcal{L}$.

\section{Mathematical Notation}
\label{sec:not}

\begin{table}[h!]
 \begin{tabularx}{\columnwidth}{p{1.7cm} X }
 \toprule 
 Symbol & Meaning \\
 \midrule 
 \multicolumn{2}{l}{\textbf{General symbols used with multiple variables}} \\
 $\hat{}$ & an expectation over a variable \\
 $\tilde{}$ & an approximation to the variable \\
 upper case, bold letter & a matrix \\
 lower case, bold letter & a vector \\
 lower case, normal letter & a function or scalar \\
 * & indicates that the variable refers to the test set, rather than the training set \\
  \multicolumn{2}{l}{\textbf{Pairwise preference labels}} \\
 $y(a,b)$ & a binary label indicating whether item $a$ is preferred to item $b$ \\
 $y_p$ & the $p$th pairwise label in a set of observations \\
 $\bs y$ & the set of observed values of pairwise labels \\
 $\Phi$ & cumulative density function of the standard Gaussian (normal) distribution \\
 $\bs x_a$ & the features of item a (a numerical vector) \\
 $\bs X$ & the features of all items in the training set \\
 $N$ & number of items in the training set \\
 $P$ & number of pairwise labels in the training set \\
 $\bs x^*$ & the features of all items in the test set \\
 $\delta_a$ & observation noise in the utility of item $a$ \\
 $\sigma^2$ & variance of the observation noise in the utilities \\
 $z_p$ & the difference in utilities of items in pair $p$, normalised by its total variance \\
 $\bs z$ & set of $z_p$ values for training pairs \\ 
  \bottomrule
 \end{tabularx}
 \caption{Table of symbols used to represent variables in this paper (continued on next page
 in Table \ref{tab:sym2}).}
 \label{tab:sym1}
\end{table}
\begin{table}
 \begin{tabularx}{\columnwidth}{p{1.7cm} X }
 \toprule 
 Symbol & Meaning \\
 \midrule 
\multicolumn{2}{l}{\textbf{GPPL (some terms also appear in crowdGPPL)}} \\
 $f$ & latent utility function over items in single-user GPPL \\
 $\bs f$ & utilities, i.e., values of the latent utility function for a given set of items \\
 $\bs C$ & posterior covariance in $\bs f$; in crowdGPPL, superscripts indicate 
 whether this is the covariance of consensus values or latent item components \\
 $s$ & an inverse function scale; in crowdGPPL, superscripts indicate which function this variable scales \\
 $k$ & kernel function \\
 $\theta$ & kernel hyperparameters for the items \\
 $\bs K$ & prior covariance matrix over items \\
 $\alpha_0$ & shape hyperparameter of the inverse function scale prior \\
 $\beta_0$ & scale hyperparameters of the inverse function scale prior \\
 \multicolumn{2}{l}{\textbf{CrowdGPPL}} \\
 $\bs F$ & matrix of utilities, where rows correspond to items and columns to users \\
 $\bs t$ & consensus utilities \\
 $C$ & number of latent components \\
 $c$ & index of a component \\
 $\bs V$ & matrix of latent item components, where rows correspond to components \\
 $\bs v_c$ & a row of $\bs V$ for the $c$th component \\
 $\bs W$ & matrix of latent user components, where rows correspond to components \\
 $\bs w_c$ & a row of $\bs W$ for the $c$th component \\ 
 $\bs \omega_c$ & posterior variance for the $c$th user component \\
 $\eta$ & kernel hyperparameters for the users \\
 $\bs L$ & prior covariance matrix over users \\
 $\bs u_j$ & user features for user $j$ \\
 $U$ & number of users in the training set \\
 $\bs U$ & matrix of features for all users in the training set \\ 
 \multicolumn{2}{l}{\textbf{Probability distributions}} \\
 $\mathcal{N}$ & (multivariate) Gaussian or normal distribution \\
 $\mathcal{G}$ & Gamma distribution \\
% \bottomrule
% \end{tabularx}
% \caption{Table of symbols used to represent variables in this paper (continued on next page
% in Table \ref{tab:sym2}).}
% \label{tab:sym1}
%\end{table}
%\begin{table}
% \begin{tabularx}{\columnwidth}{p{1.7cm} X }
% \toprule 
% Symbol & Meaning \\
% \midrule 
 \multicolumn{2}{l}{\textbf{Stochastic Variational Inference (SVI) }} \\
 $M$ & number of inducing items \\
 $\bs Q$ & estimated observation noise variance for the approximate posterior \\
 $\gamma, \lambda$ & estimated hyperparameters of a Beta prior distribution over $\Phi(z_p)$ \\
 $i$ & iteration counter for stochastic variational inference \\
 $\bs f_m$ & utilities of inducing items \\
 $\bs K_{mm}$ & prior covariance of the inducing items \\
 $\bs K_{nm}$ & prior covariance between training and inducing items \\
 $\bs S$ & posterior covariance of the inducing items; in crowdGPPL, a superscript and subscript 
 indicate which variable this is the posterior covariance for \\
 $\bs \Sigma$ & posterior covariance over the latent user components \\
 $\bs A$ & $\bs K_{nm} \bs K_{mm}^{-1}$ \\
 $\bs G$ & linearisation term used to approximate the likelihood \\
 $a$ & posterior shape parameter for the Gamma distribution over $s$ \\
 $b$ & posterior scale parameter for the Gamma distribution over $s$ \\
 $\rho_i$ & a mixing coefficient, i.e., a weight given to the $i$th update when combining with current values of variational
 parameters \\
 $\epsilon$ & delay \\
 $r$ & forgetting rate \\
 $\pi_i$ & weight given to the update at the $i$th iteration using a subsample of the data \\
 $|P_i|$ & number of pairwise labels in the $i$th iteration subsample \\
 \bottomrule
 \end{tabularx}
 \caption{Table of symbols used to represent variables in this paper (continued from Table \ref{tab:sym1} on previous page).}
 \label{tab:sym2}
\end{table}

%%%%%%%%%%%%%%%%%%%%%%%%%%%%%%%%%%%%%%%%%%%%%%%%%%%%%%%%%%%%%%%%%%%%%%%%%%%%%%%%%

\end{document}
