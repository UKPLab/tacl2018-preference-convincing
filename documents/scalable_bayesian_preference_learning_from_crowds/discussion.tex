\section{Conclusions}

We proposed a Bayesian preference learning approach 
for modeling both the individual utilities of members of a crowd 
and forming a consensus from 
unreliable annotations, such as those provided by a crowd or through implicit user feedback.
The model learns the latent utilities of items from pairwise comparisons using a combination of Gaussian processes and Bayesian matrix factorization.
To enable inference at the scale of real-world datasets in fields such as NLP,
we derived a stochastic variational inference scheme.
Our empirical results confirm that our approach scales well with the amount of training data,
and that jointly modeling the consensus and personal preferences of users can improve the predictions
of both.
%far better than previous
%Gaussian process preference learning methods 
%without harming  
%performance on individual utility prediction and significantly improving performance
%on consensus learning.
Future work will evaluate the benefit of learning inducing point locations from data,
optimizing length-scale hyperparameters by maximizing $\mathcal L$,
and investigate the possibility of integrating deep generative models 
for learning feature representations from input data.
%the models can readily be adapted to regression or classification tasks by swapping out the preference likelihood, resulting in 
%different values for $\bs G$ and $\bs H$.
% Preference elicitation using information theoretic methods.
