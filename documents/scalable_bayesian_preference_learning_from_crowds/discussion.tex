\section{Conclusions}\label{sec:conclusion}

We proposed a novel Bayesian preference learning approach 
for modelling both the preferences of individuals 
and the overall consensus of a crowd. 
Our model learns the latent utilities of items from pairwise comparisons 
using a combination of Gaussian processes and Bayesian matrix factorisation 
to capture differences in  opinion.
We introduce a stochastic variational inference (SVI) method, that, 
unlike previous work, can scale to arbitrarily large datasets,
since its time and memory complexity do not grow with the dataset size.
Our experiments confirm the method's scalability and
 show that jointly modelling the consensus and personal
preferences can improve predictions of both.
Our approach performs competitively
against less scalable alternatives
%despite the approximations required for SVI.
and improves on 
the previous state of the art
for predicting argument convincingness from crowdsourced data~\citep{simpson2018finding}.
%far better than previous
%Gaussian process preference learning methods 
%without harming  
%performance on individual utility prediction and significantly improving performance
%on consensus learning.

Future work will investigate learning inducing point locations and
optimising length-scale hyperparameters by maximising 
the variational lower bound, $\mathcal L$,
 as part of the variational inference method.
Another important direction will be to generalise the likelihood from pairwise comparisons
to comparisons involving more than two items~\citep{pan2018stagewise}
or best--worst scaling~\citep{kiritchenko2017best}
to provide scalable Bayesian methods for other forms of comparative preference data.
%and investigate the possibility of integrating deep generative models 
%for learning feature representations from input data.
%the models can readily be adapted to regression or classification tasks by swapping out the preference likelihood, resulting in 
%different values for $\bs G$ and $\bs H$.
% Preference elicitation using information theoretic methods.
