\section{Variational Inference}
\label{sec:vb_eqns}

Due to the non-Gaussian likelihood, Equation \ref{eq:plphi},
the posterior distribution over $\bs f$ contains intractable integrals:
\begin{flalign}
p(\bs f | \bs y, k_{\theta}, \alpha_0, \alpha_0) = 
\frac{\int \prod_{p=1}^P \Phi(z_p) \mathcal{N}(\bs f; \bs 0, \bs K_{\theta}/s) 
\mathcal{G}(s; \alpha_0, \beta_0) d s}{\int \int \prod_{p=1}^P \Phi(z_p) \mathcal{N}(\bs f'; \bs 0, \bs K_{\theta}/s) 
\mathcal{G}(s; \alpha_0, \beta_0) d s d f' }.
\label{eq:post_single}
\end{flalign}

We can derive a variational lower bound as follows, beginning with an approximation that does not use inducing points:
\begin{flalign}
\mathcal{L}_1 = \sum_{i=1}^{L} \mathbb{E}_{q}\left[ \log p\left( v_i \succ u_i | f(v_i), f(u_i) \right) \right]
+ \mathbb{E}_{q}\left[ \log \frac{p\left( \bs f | \bs\mu, \bs K/s \right)}{q\left(\bs f\right)} \right]
+ \mathbb{E}_{q}\left[ \log \frac{p\left( s | \alpha_0, \beta_0\right)}{q\left(s \right)} \right] &&
\label{eq:vblb}
\end{flalign}
Substituting the forms of the distributions with their variational parameters, we get:
\begin{flalign}
\mathcal{L}_1 & = \mathbb{E}_{q}\Bigg[ \sum_{i=1}^{L} [v_i \succ u_i]\log\Phi(z_i) + [v_i \prec u_i]\left(1-\log\Phi(z_i)\right) \Bigg] \nonumber&&\\
& + \log \mathcal{N}\left(\hat{\bs f}; \bs\mu, \bs K/\hat{s} \right) 
- \log\mathcal{N}\left(\hat{\bs f}; \hat{\bs f}, \bs C \right)
 + \mathbb{E}_{q}\left[ \log\mathcal{G}\left( s; \alpha_0, \beta_0\right) - \log\mathcal{G}\left(s; \alpha, \beta \right) \right]  &&
\end{flalign}
We now replace the likelihood with a Gaussian approximation:
\begin{flalign}
\mathcal{L}_1 \approx \mathcal{L}_2 & = \mathbb{E}_{q}\left[ \mathcal{N}( \bs y | \Phi(\bs z), \bs Q) \right]
 + \log \mathcal{N}\left(\bs f; \bs\mu, \bs K/\hat{s} \right) - \log\mathcal{N}\left(\bs f; \hat{\bs f}, \bs C \right) 
&\nonumber\\
& + \mathbb{E}_q\left[ \log\mathcal{G}\left( s; \alpha_0, \beta_0\right) - \log\mathcal{G}\left(s; \alpha, \beta \right) \right] \nonumber&\\
& =  - \frac{1}{2} \left\{ L \log 2\pi + \log |\bs Q| - \log|\bs C| 
 + \log|\bs K/s| + (\hat{\bs f} - \bs\mu)\hat{s}\bs K^{-1}
(\hat{\bs f} - \bs\mu) \right. \nonumber &\\
& \left. + \mathbb{E}_q\left[ (\bs y - \Phi(\bs z))^T \bs Q^{-1} (\bs y - \Phi(\bs z)) \right] \right\}
 - \Gamma(\alpha_0) + \alpha_0(\log \beta_0) + (\alpha_0-\alpha)\mathbb{E}[\log s] \nonumber&\\
& + \Gamma(\alpha) + (\beta-\beta_0) \hat{s} - \alpha \log \beta,  &
\end{flalign}
where $\mathbb{E}[s] = \frac{\alpha}{\beta}$, $\mathbb{E}[\log s] = \Psi(2\alpha) - \log(2\beta)$,
$\Psi$ is the digamma function and $\Gamma()$ is the gamma function, 
Finally, we use a Taylor-series linearisation to make the remaining expectation tractable:
\begin{flalign}
\mathcal{L}_2 & \approx \mathcal{L}_3 = - \frac{1}{2} \left\{ L \log 2\pi + \log |\bs Q| - \log|\bs C| \right.
 \left. + \log|\bs K/\hat{s}| + (\hat{\bs f} - \bs\mu)\hat{s}\bs K^{-1}(\hat{\bs f} - \bs\mu) \right. \nonumber&&\\
 & \left. + (\bs y - \Phi(\hat{\bs z}))^T \bs Q^{-1} (\bs y - \Phi(\hat{\bs z}))\right\}
 - \Gamma(\alpha_0) + \alpha_0(\log \beta_0) + (\alpha_0-\alpha)\mathbb{E}[\log s] \nonumber&&\\
& + \Gamma(\alpha) + (\beta-\beta_0) \hat{s} - \alpha \log \beta. &&
\label{eq:vblb_terms} 
\end{flalign}
Now, we can introduce the sparse approximation to obtain the bound in Equation \ref{eq:lowerbound}:
\begin{flalign}
\mathcal{L} =\; & \mathbb{E}_{q(\bs f, \bs f_m, s)}[\log p(\bs y | \bs f) + \log p(\bs f_m, s | \bs K, 
\alpha_0, \beta_0) -\log q(\bs f_m) - \log q(s) ] & \nonumber \\ 
=\; & \sum_{p=1}^P \mathbb{E}_{q(\bs f)}[\log p(y_p | f_{a_p}, f_{b_p})] - \frac{1}{2} \bigg\{ \log|\bs K_{mm}| - \mathbb{E}[\log s] - \log|\bs S| - M
\nonumber &\\
& + \hat{\bs f}_m^T\mathbb{E}[s] \bs K_{mm}^{-1}\hat{\bs f}_m + 
\textrm{tr}(\mathbb{E}[s] \bs K_{mm}^{-1} \bs S) \bigg\}  + \log\Gamma(\alpha) - \log\Gamma(\alpha_0)  + \alpha_0(\log \beta_0) \nonumber\\
& + (\alpha_0-\alpha)\mathbb{E}[\log s]+ (\beta-\beta_0) \mathbb{E}[s] - \alpha \log \beta, &
\label{eq:full_L_singleuser}
\end{flalign}
where the terms relating to $\mathbb{E}[p(\bs f | \bs f_m) - q(\bs f)]$ cancel.
Without stochastic sampling, the variational factor $\log q(\bs f_m)$ is given by:
\begin{flalign}
\log q(\bs f_m) &= \log \mathcal{N}\left(\bs y; \tilde{\Phi}(\bs z), \bs Q\right)]
+ \log\mathcal{N}\left(\bs f_m; \bs 0, \bs K_{mm}/\mathbb{E}\left[s\right]\right)  + \textrm{const}, \nonumber \\
%&= \log \int \mathcal{N}(\bs y - 0.5; \bs G \bs f, \bs Q) 
%\mathcal{N}(\bs f; \bs A \bs f_m, \bs K - \bs A \bs K_{nm}^T) & \nonumber\\
%& \hspace{3.2cm} \mathcal{N}(\bs f_m; \bs 0, \bs K_{mm}\mathbb{E}[1/s]) \textrm{d} \bs f + \textrm{const} & \nonumber\\
 & = \log \mathcal{N}(\bs f_m; \hat{\bs f}_m, \bs S ), \\
\bs S^{-1} &= \bs K^{-1}_{mm}/\mathbb{E}[s] + \bs A^T \bs G^T \bs Q^{-1} \bs G \bs A, \label{eq:S}\\
\hat{\bs f}_m &= \bs S \bs A^T \bs G^T \bs Q^{-1} (\bs y - \Phi(\mathbb{E}[\bs z]) + \bs G \mathbb{E}[\bs f] ). \label{eq:fhat_m}
\end{flalign}

For the crowdGPPL model, the variational lower bound is:
\begin{flalign}
\mathcal{L}_{cr} & = \label{eq:lowerbound_crowd_full}
\sum_{p=1}^P \mathbb{E}_{q(\bs f)}[\log p(y_p | \bs v_{a_p}^T \! \bs w_{a_p} \!+ t_{a_p}\!, \bs v_{b_p}^T\! \bs w_{b_p} \!+ t_{b_p})] 
- \frac{1}{2} 
\Bigg\{  \sum_{c=1}^C \bigg\{  \! - \! M_n \! - \! M_u & \nonumber \\
&  
 + \log|\bs K_{v,mm}| + \log|\bs K_{w,mm}|
- \log|\bs S_{v,c}|  - \mathbb{E}[\log s_c] 
+ \hat{\bs v}_{m,c}^T \mathbb{E}[s_c]\bs K_{v,mm}^{-1}\hat{\bs v}_{m,c} & \nonumber \\
& 
+ \textrm{tr}(\mathbb{E}[s_c] \bs K_{v,mm}^{-1} \bs S_{v,c}) 
- \log|\bs \Sigma_{c}|  + \hat{\bs w}_{m,c}^T \bs K_{w,mm}^{-1}\hat{\bs w}_{m,c} 
+ \textrm{tr}(\bs K_{w,mm}^{-1} \bs \Sigma_{c})
\bigg\}
& \nonumber \\
&  
- M_n + \log|\bs K_{t,mm}|
- \log|\bs S_{t}|  - \mathbb{E}[\log \sigma] 
+ \hat{\bs t}^T \mathbb{E}[\sigma] \bs K_{t,mm}^{-1} \hat{\bs t}  &
\nonumber \\
&
+ \textrm{tr}(\mathbb{E}[\sigma] \bs K_{t,mm}^{-1} \bs S_{t})
\Bigg\} 
- (C+1)(\log\Gamma(\alpha_0)  + \alpha_0(\log \beta_0))
& \nonumber \\
& + \sum_{c=1}^C \bigg\{ 
\log\Gamma(\alpha_c) + (\alpha_0 - \alpha_c)\mathbb{E}[\log s_c]
+ (\beta_c - \beta_0) \mathbb{E}[s_c] - \alpha_c \log \beta_c \bigg\}
 & 
\nonumber \\ 
& + \log\Gamma(\alpha_{\sigma}) + (\alpha_0 - \alpha_{\sigma})\mathbb{E}[\log \sigma]
+ (\beta_{\sigma} - \beta_0) \mathbb{E}[s_c] - \alpha_{\sigma} \log \beta_{\sigma}
, &
\end{flalign}

The gradient of $\mathcal{L}_3$ with respect to the lengthscale, $l_d$, is as follows:
\begin{flalign}
\nabla_{l_d} \mathcal{L}_3 & =  - \frac{1}{2} \left\lbrace 
\frac{\partial \log|\bs K/\hat{s}|}{\partial l_d} - \frac{\partial \log|\bs C|}{\partial l_d} 
\nonumber \right.
 \left.  - (\hat{\bs f}-\bs\mu)\hat{s} \frac{\partial K^{-1}}{\partial l_d} (\hat{\bs f}-\bs\mu)
\right\rbrace \nonumber & \\
%& = \frac{1}{2} \hat{s} \left\lbrace \frac{\partial \log |\bs C \bs K^{-1}|}{\partial l_d}
%\right. \\
%& \left.  - (\hat{\bs f}-\bs\mu) \bs K^{-1} \frac{\partial \bs K}{\partial l_d} \bs K^{-1} (\hat{\bs f}-\bs\mu)
%\right\rbrace  \nonumber \\
& =  -\frac{1}{2} \left\lbrace  \frac{\partial \log | \frac{1}{\hat{s}}\bs K \bs C^{-1} |}{\partial l_d} \right. 
\left.  + \hat{s} (\hat{\bs f}-\bs\mu) \bs K^{-1} \frac{\partial \bs K}{\partial l_d} \bs K^{-1} (\hat{\bs f}-\bs\mu)
\right\rbrace   &
%& =  - \frac{1}{2} \left\lbrace \frac{\partial \log|\bs K/s| }{\partial l_d} + \frac{\partial \log |\bs K^{-1}s + \bs G\bs Q^{-1}\bs G^T|}{\partial l_d}
%\right. \\
%& \left.  - \hat{s} (\hat{\bs f}-\bs\mu) \bs K^{-1} \frac{\partial \bs K}{\partial l_d} \bs K^{-1} (\hat{\bs f}-\bs\mu)
%\right\rbrace  \nonumber\\
%& =  -\frac{1}{2} \left\lbrace \frac{\partial \log |\bs I + \bs K/s\bs G\bs Q^{-1}\bs G^T|}{\partial l_d}
%\right. \\
%& \left.  - \hat{s} (\hat{\bs f}-\bs\mu) \bs K^{-1} \frac{\partial \bs K}{\partial l_d} \bs K^{-1} (\hat{\bs f}-\bs\mu)
%\right\rbrace  \nonumber
\end{flalign}
Using the fact that $\log | A | = \mathrm{tr}(\log A)$, $\bs C = \left[\bs K^{-1} - \bs G \bs Q^{-1} \bs G^T \right]^{-1}$, and $\bs C = \bs C^{T}$, we obtain:
\begin{flalign}
\nabla_{l_d} \mathcal{L}_3 & =  -\frac{1}{2} \mathrm{tr}\left(\left(\hat{s}\bs K^{-1}\bs C\right) \bs G\bs Q^{-1}\bs G^T \frac{\partial \bs K}{\partial l_d}
\right)
 + \frac{1}{2}\hat{s} (\hat{\bs f}-\bs\mu) \bs K^{-1} \frac{\partial \bs K}{\partial l_d} \bs K^{-1} (\hat{\bs f}-\bs\mu)  \nonumber\\ 
& =  -\frac{1}{2} \mathrm{tr}\left(\left(\hat{s}\bs K^{-1}\bs C\right)
\left(\bs C^{-1} - \bs K^{-1}/\hat{s}\right) \frac{\partial \bs K}{\partial l_d}
\right) 
+ \frac{1}{2}\hat{s} (\hat{\bs f}-\bs\mu) \bs K^{-1} \frac{\partial \bs K}{\partial l_d} \bs K^{-1} (\hat{\bs f}-\bs\mu).  \label{eq:gradient_ls}
\end{flalign}
Assuming a product over kernels for each feature, $\bs K=\prod_{d=1}^{D} \bs K_d$, we can compute the kernel gradient 
as follows for the Mat\'ern $\frac{3}{2}$ kernel function:
\begin{flalign}
\frac{\partial \bs K}{\partial l_d} & = \prod_{d'=1,d'\neq d}^D K_{d} \frac{\partial K_{l_d}}{\partial l_d} \\
\frac{\partial K_{l_d}}{\partial l_d} & = \frac{3\bs |\bs x_d - \bs x_d'|^2}{l_d^3} \exp\left( - \frac{\sqrt{3} \bs |\bs x_d - \bs x_d'|}{l_d} \right)
\label{eq:kernel_der}
\end{flalign}
where $|\bs x_d - \bs x_d'|$ is the distance between input points.

\section{Converged Lower Bound Derivatives}
\label{sec:gradients}

When $\mathcal{L}$ has converged to a maximum, 
$\nabla_{l_{\! d}} \mathcal{L}$ simplifies to:
\begin{flalign}
 &\nabla_{\!l_{\! d}} \mathcal{L} \longrightarrow 
\frac{1}{2} \mathrm{tr}\!\left(\! \left(
\mathbb{E}[s](\hat{\bs f}_{\! m} \hat{\bs f}_{\! m}^T + \bs S^T)\bs K_{\! mm}^{-1} \! -  \bs I \! \right)
 \!\frac{\partial \bs K_{\! mm}}{\partial l_d} \bs K_{\! mm}^{-1} \right) \!. &
\label{eq:gradient_single}
\end{flalign}
For crowdGPPL, assuming that $\bs V$ and $\bs t$ have the same kernel function,
the gradient
%The gradients with respect to the length-scale, $l_{w,d}$,
for the $d$th item feature is given by:
\begin{flalign}
 &\nabla_{l_{ v,d}} \mathcal{L}_{cr} \longrightarrow
%  \sum_{c=1}^{C} \bigg\{ \mathbb{E}[s_c] 
%  \hat{\bs v}_{ m,c}^T \bs K_{ mm,v}^{-1} 
% \frac{\partial \bs K_{ mm,v}}{\partial l_{w,d}} \bs K_{ mm,v}^{-1} \hat{\bs v}_{ m,c} 
%  + & \nonumber \\
%  & \mathrm{tr}\left( \left(
% \mathbb{E}[s_c]\bs S_{v,c}^T\bs K_{ mm,v}^{-1}  - \frac{1}{2} \bs I  \right)
%  \frac{\partial \bs K_{ mm,v}}{\partial l_{w,d}} \bs K_{ mm,v}^{-1} \right) \bigg\}
%  + & \nonumber \\
%  & \mathbb{E}[\sigma] 
%  \hat{\bs t}_{ m}^T \bs K_{ mm,t}^{-1} 
% \frac{\partial \bs K_{ mm,t}}{\partial l_{w,d}} \bs K_{ mm,t}^{-1} \hat{\bs t}_{ m} 
%  + \mathrm{tr}\left( \left(
% \mathbb{E}[\sigma]\bs S_{t}^T\bs K_{ mm,t}^{-1}  - \frac{1}{2} \bs I  \right)
%  \frac{\partial \bs K_{ mm,t}}{\partial l_{w,d}} \bs K_{ mm,t}^{-1} \right)
%  & \nonumber \\
% & = 
\frac{1}{2} \mathrm{tr}\left( \left( \sum_{c=1}^{C} \mathbb{E}[s_c] \left\{ \hat{\bs v}_{m,c} 
 \hat{\bs v}_{m,c}^T + \bs S_{v,c}^T \right\}
 \bs K_{ mm,v}^{-1}  - C\bs I  \right)
 \frac{\partial \bs K_{ mm,v}}{\partial l_{w,d}} \right.
 & \nonumber \\
 & \left.  \bs K_{ mm,v}^{-1} \right) + \frac{1}{2}\mathrm{tr}\left( \left(
\mathbb{E}[\sigma](\hat{\bs t}_{ m}\hat{\bs t}_{ m}^T + \bs S_{t}^T) \bs K_{ mm,t}^{-1}  
- \bs I  \right)
 \frac{\partial \bs K_{ mm,t}}{\partial l_{w,d}} \bs K_{ mm,t}^{-1} \right)
.&
\label{eq:gradient_crowd_items}
\end{flalign}
% If different kernels are used for different components, then the equation above can be modified to
% simply sum over terms relating to the components with a shared kernel function. 
The gradients for the $d$th user feature length-scale, $l_{w,d}$, follows the same form:
\begin{flalign}
 &\nabla_{l_{w,d}} \mathcal{L}_{cr} \!\!\!\longrightarrow \frac{1}{2} 
 \!\mathrm{tr}\left( \!\left( \sum_{c=1}^{C} \left\{ \hat{\bs w}_{m,c} \hat{\bs w}_{m,c}^T \!+
\bs \Sigma_c^T\right\} \!\bs K_{mm,w}^{-1} \! - C\bs I  \right)
 \frac{\partial \bs K_{mm,w}}{\partial l_{w,d}} \bs K_{mm,w}^{-1} \!\right) \!. &
\label{eq:gradient_crowd_users}
\end{flalign}

% When combining kernel functions for each features using a product,
% as in Equation \ref{eq:kernel}, the partial derivative of the covariance matrix $\bs K_{mm}$ with respect to 
% $l_d$ is given by:
% \begin{flalign}
% \frac{\partial \bs K_{mm}}{\partial l_d} 
% & = \frac{\bs K_{mm}}{\bs K_{d}}
% \frac{ \bs K_{d}(|\bs x_{mm,d}, \bs x'_{mm,d})}{\partial l_d} \nonumber ,\\
% \end{flalign}
The partial derivative of the covariance matrix $\bs K_{mm}$ with respect to 
$l_d$ depends on the choice of kernel function. 
The Mat\`ern $\frac{3}{2}$ function is a widely-applicable, differentiable kernel function 
that has been shown empirically to outperform other well-established kernels 
such as the squared exponential, and makes weaker assumptions of smoothness of 
the latent function~\citep{rasmussen_gaussian_2006}. 
It is defined as:
\begin{flalign}
k_d\left(\frac{|x_d - x_d'|}{l_d} \right) = \left(1 + \frac{\sqrt{3} | x_d - x_d'|}{l_d}\right) 
\exp \left(- \frac{\sqrt {3} | x_d - x_d'|}{l_d}\right).
\end{flalign}
%For the Mat\`ern $\frac{3}{2}$ kernel,  
Assuming that the kernel functions for each feature, $k_d$, are combined using
a product, as in Equation \ref{eq:kernel}, 
the partial derivative $\frac{\partial \bs K_{mm}}{\partial l_d}$ is a matrix, where each 
entry, $i,j$,  is defined by:
\begin{flalign}
& \frac{\partial K_{mm,ij}}{\partial l_d} = 
\prod_{d'=1, d' \neq d}^D k_{d'}\left(\frac{|x_{d'} - x_{d'}'|}{l_{d'}}\right)
\frac{3 (\bs x_{i,d} - \bs x_{j,d})^2}{l_d^3} \exp\left( - \frac{\sqrt{3} \bs |\bs x_{i,d} - \bs x_{j,d}|}{l_d} \right), &
\label{eq:kernel_der}
\end{flalign}
where we assume the use of Equation\ref{eq:kernel} to combine kernel 
functions over features using a product.


To make use of Equations \ref{eq:gradient_single} to \ref{eq:kernel_der},
we nest the variational algorithm defined in Section \ref{sec:inf} inside
an iterative gradient-based optimization method.
Optimization then begins with an initial guess for all length-scales, $l_d$,
such as the median heuristic.
Given the current values of $l_d$, the optimizer (e.g. L-BFGS-B)
runs the VB algorithm to convergence, 
computes $\nabla_{l_{\! d}} \mathcal{L}$,
then proposes a new candidate value of $l_d$.
The process repeats until the optimizer converges or reaches a maximum number 
of iterations, and returns the value of $l_d$ that maximized $\mathcal{L}$.