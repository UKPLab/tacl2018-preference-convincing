\todo{ add in table of notations (maybe in the appendix)}

\section{Bayesian Preference Learning for Crowds}\label{sec:model}

%%%% Notes

% Title or name of the model: 
% -- cannot decide this until we get most of the paper complete: will emphasis be on crowds? distilling ground truth
% from noisy sources (Bayesian preference learning for fusing unreliable sources)? user preferences/collaborative filtering?
% -- need a new name to differentiate from Houlsby et al. and Khan et al.?
% -- what are the differences in the model? Let's get the model written up.
% -- should also be some buzzword or word to generate interest: 
%    -- 'variational' is on the up, could be used in paper title
%    -- 'stochastic variational' is also on the up
%    -- 'crowdsourcing' on the way down as at 2010 level
%    -- 'gaussian process' on the way up
%    -- 'matrix factorization' kind of on the way up
%    -- 'scalable' on the way up
%    -- 'interactive learning' on the way down
%    -- 'preference learning' flattish, may be on way up slowly
% -- aimed at crowdsourcing problems (uses a common mean function as consensus?)
% -- other parameters for importance of features?
% -- combined preference learning? Preference aggregation? Collaborative crowdsourced preference learning? Bayesian preference learning for crowds? another word for 'multi-user' or 'many users and many items' vs. crowds?

% TO ADD: Why does the variance in f cancel out when predicting the probability of a pairwise label?
% TO ADD: Why does \sigma disappear if we learn the output scale.

% We also estimate the output scale of the GPs, the latent components, and item bias as part of the 
% variational approximation allowing us to estimate these parameters in a Bayesian manner without 
% resorting to maximum likelihood approaches.

% mention how the noise model deals with inconsistencies in preferences

% \begin{enumerate}
% \item What are the benefits of Bayesian methods and Gaussian processes in particular?
% \item The proposal by \citep{chu2005preference} shows how the advantages of a Bayesian
% approach can be exploited for preference learning by modifying the observation model

% Extensions:
% -- how do we replace the GP with a NN?
% -- would this move us from a Bayesian to an ML solution?
% -- maybe save this for future work? Or add a few lines if we can make it fit with the theme of the paper.
% -- another is to replace the fixed number of clusters with a CRP, then the whole thing can be nonparameteric preference learning with crowds.

%\subsection{Modeling Pairwise Preferences}

% Include the case for one user -- preferences may depend on a number of observed features.

%TODO items or instances?
%TODO prefer the use of 'utility' to 'value' since there is some uncertainty (the evaluations by the users can flip at random)

% this should be introduced in section 1. \citep{handleycomparative} -- compares BT with TM models
For a pair of items, $a$ and $b$, 
$a \succ b$ indicates that $a$ is preferred to $b$.  
% %\begin{equation}
% $y(a, b) = \begin{cases}
% 1 & a \succ b \\
% 0 & b \succ a
% \end{cases}$.
% %\end{equation} 
We assume that items have latent \emph{utilities},
$f(\bs x_a)$ and $f(\bs x_b)$, that represent their value to a user,
and that the 
utility $f(\bs x_a) : \mathbb{R}^D \mapsto \mathbb{R}$ 
is a function of the item's features, where $\bs x_a$ is a vector of length $D$
containing the features of item $a$.
Hence if $f(\bs x_a) > f(\bs x_b)$, then $a \succ b$.
We record the outcome of a comparison between $a$ and $b$ as 
a pairwise label, $y(a, b)$.
Assuming that pairwise labels never contain errors,
then $y(a, b)=1$ if $a \succ b$ and $0$ otherwise.

\citet{thurstone1927law} proposed the \emph{random utility model},
which relaxes the assumption that pairwise labels, $y(a, b)$,
are always consistent with the ordering of $f(\bs x_a)$ and $f(\bs x_b)$.
Under the random utility model, the likelihood $p(y(a,b)=1)$ 
increases as $f_a - f_b$ increases, i.e.,
as the utility of item $a$ increases
relative to the utility of item $b$.
However, since $0 < p(y(a,b)=1) < 1$, the model 
is uncertain about the value of $y(a,b)$,
which accommodates labelling errors or
inconsistency in a user's choices at different points in time.
The uncertainty is lower if the values $f_a$ and $f_b$ are further apart, 
which reflects greater consistency in a user's choices
when their preferences are stronger.
 
The random utility model is defined by a likelihood function that
maps the utilities to $p(y(a,b))$.
Two important random utility models for pairwise labels are the Bradley-Terry model ~\citep{bradley1952rank,plackett1975analysis,luce1959possible},
which assumes a logistic likelihood,
and the Thurstone-Mosteller model, also known as \emph{Thurstone case V}~\citep{thurstone1927law,mosteller2006remarks}, which assumes a probit likelihood.
%or the Thurstone-Mosteller model.
%\begin{align}
%p(y(a, b) | f) & = \frac{1}{1 + \exp( f(\bs x_a) - f(\bs x_b) ) }
%\end{align}
In the Thurstone case V model, 
noise in the observations is explained by a Gaussian-distributed noise term, $\delta \sim \mathcal{N}(0, \sigma^2)$:
\begin{align}
 p(y(a, b) | f(\bs x_a) + \delta_{a}, f(\bs x_b) + \delta_{b} )  
 \hspace{0.9cm} & = \begin{cases}
 1 & \text{if }f(\bs x_a) + \delta_{a} \geq f(b) + \delta_{b} \\
 0 & \text{otherwise,}
 \end{cases} &
 \label{eq:thurstone}
\end{align}
Integrating out the unknown values of $\delta_a$ and $\delta_b$ gives:
\begin{align}
& p( y(a, b) | f(\bs x_a), f(\bs x_b) )  \nonumber\\
& = \int\!\!\!\! \int p( y(a, b) | f(\bs x_a) + \delta_{a}, f(\bs x_b) + \delta_{b} ) \mathcal{N}(\delta_{a}; 0, \sigma^2)\mathcal{N}(\delta_{b}; 0, \sigma^2) d\delta_{a} d\delta_{b} \nonumber\\
& = \Phi\left(\frac{f(\bs x_a) - f(\bs x_b)}{\sqrt{2\sigma^2}}\right) = \Phi\left( z \right), 
\label{eq:plphi}
\end{align}
where $z = \frac{f(\bs x_a) - f(\bs x_b)}{\sqrt{2\sigma^2}}$,
and $\Phi$ is the cumulative distribution function of the standard normal distribution,
meaning that $\Phi(z)$ is a probit likelihood. 

In practice, $f(\bs x_a)$ and $f(\bs x_b)$ must be inferred from
pairwise training labels, $\bs y$,
so that we obtain a posterior distribution over their values.
If we assume that this posterior is a multivariate Gaussian distribution,
then the probit likelihood allows us to analytically marginalise 
$f(\bs x_a)$ and $f(\bs x_b)$
to obtain the probability of a pairwise label:
\begin{align}
p(y(a,b)| \bs y) 
= \Phi(\hat{z}),& & \hat{z} = \frac{\hat{f}_a - \hat{f}_b}{\sqrt{2\sigma^2 + C_{a,a} + C_{b,b} 
- 2C_{a,b}} }, \label{eq:predict_z}
\end{align}
where $\hat{f}_a$ and $\hat{f}_b$ are the means and
$\bs C$ is the covariance matrix of the multivariate Gaussian over
$f(\bs x_a)$ and $f(\bs x_b)$.
This likelihood model is the basis of Gaussian process preference learning (GPPL)~\citep{chu2005preference}. 
Due to this mathematical convenience,
we also assume this pairwise label likelihood for the crowd
preference learning model proposed in this paper.
Here, in contrast to \citet{chu2005preference},
we simplify the formulation by assuming that $\sigma^2 = 0.5$,
which leads to $z$ having a denominator of $\sqrt{2 \times 0.5}=1$.
Instead, we model varying degrees of noise in the pairwise labels
by scaling $f$ itself, as we describe in the next section.
%Obtaining the posterior over $f$ is itself challenging, however, 
%and therefore in Section $\ref{sec:inf}$ we propose 
%an approximate inference method to address this problem.


\subsection{Single User Preference Learning}
%TODO check use of 'user' versus 'annotator' terminology throughout, especially in the intro. Is it explicit that the term 'user' is abstract and refers to any of these things?
We can model the preferences of a single user by assuming
a Gaussian process prior over the user's utility function, 
%is a function of item features and 
$f \sim \mathcal{GP}(0, k_{\theta}/s)$, where $k_{\theta}$ is a kernel function with hyperparameters $\theta$
and $s$ is an inverse scale parameter.
The kernel function takes numerical item features as inputs and determines the covariance between values of $f$ for different items. 
The choice of kernel function and its hyperparameters controls the shape and smoothness of the function 
across the feature space and is often treated as a model selection problem.
Typical kernel functions include the \emph{squared exponential} 
and the \emph{Mat\'ern}~\citep{rasmussen_gaussian_2006},
which both make minimal assumptions and are effective in a wide range of tasks. 
These functions assign higher covariance to items with similar feature values.
We use the kernel function $k_{\theta}$ to compute a covariance matrix $K_{\theta}$,
between a set of $N$ observed items with features $\{ \bs x_1, ..., \bs x_N \}$.

The model extends the original definition of
\emph{Gaussian process preference learning (GPPL)}~\citep{chu2005preference},
by introducing the inverse scale, $s$. 
We also assume that $s$ is drawn from a gamma prior, 
$s \sim \mathcal{G}(\alpha_0, \beta_0)$, with shape $\alpha_0$ and scale $\beta_0$.
The value of $1/s$ determines the variance of $f$,
and therefore 
the magnitude of differences between $f(\bs x_a)$ and $f(\bs x_b)$ for
items $a$ and $b$. This in turn affects the level of certainty
in the pairwise label likelihood as per Equation \ref{eq:plphi}.

Given a set of $P$ pairwise labels, %for a single user, 
$\bs y=\{y_1,...,y_P\}$,
where %the $p$th label, 
$y_p=y(a_p, b_p)$ is the preference label for items $a_p$ and $b_p$, % refers to items $\{ a_p, b_p \}$.
we can write the joint distribution over all variables as follows:
\begin{flalign}
p\left( \bs{y}, \bs f, s | k_{\theta}, \alpha_0, \beta_0 \right) 
=  \prod_{p=1}^P p( y_p | \bs f ) 
\mathcal{N}(\bs f; \bs 0, \bs K_{\theta}/s) \mathcal{G}(s; \alpha_0, \beta_0) %\nonumber \\
%=  \prod_{p=1}^P \Phi\left( z_p \right) 
%\mathcal{N}(\bs f; \bs 0, \bs K_{\theta}/s) \mathcal{G}(s; \alpha_0, \beta_0), &
\label{eq:joint_single}
\end{flalign}
where 
$\bs f = \{f(\bs {x}_1),...,f(\bs {x}_N)\}$
is a vector containing the utilities of the $N$ items referred to by $\bs y$,
and $p( y_p | \bs f ) = \Phi\left( z_p \right)$ is the pairwise likelihood (Equation \ref{eq:plphi}). 
We henceforth refer to this model simply as \emph{GPPL}.

\subsection{Crowd Preference Learning} \label{sec:crowd_model}

% 0. Consider multiple users, each with a set of observed features.
% 1. imagine augmenting the observed features with a number of latent features (this is kind of what Khan et al. do)
% 2. imagine that the latent features relate preference function values between items and users using the smallest number of features
% 3. hence the observed features are not needed given the latent features, but we can create a hierarchical model where the latent features depend on the observed ones if available.

To predict the individual preferences of users in a crowd,
we could assume an independent GPPL model for each user.
However, by modelling all users' preferences jointly, we can
exploit the correlations between different users' interests
to improve predictions when preference data is sparse,
and reduce the cost of storing separate models for all users.
Groups of users may share common interests over certain subsets of items,
for example, in a book recommendation task,
some users may share an interest in one particular genre 
but otherwise prefer different categories of books.
Identifying such correlations between users' interests helps to predict 
the preferences of users for whom we have only observed a small number of preferences.
This is the core idea of recommendation techniques such as collaborative filtering~\citep{resnick1997recommender} and matrix factorisation~\citep{koren2009matrix}.

As well as predicting individual preferences, 
we wish to predict the consensus by aggregating
preference labels from multiple users. 
If we do not account for the individual biases of different users
when predicting the consensus,
these biases may affect the consensus predictions,
particularly when data for certain items is only available from a small subset of users.
We address this problem by proposing \emph{crowdGPPL}, which jointly models 
the preferences of individual users as well as the underlying consensus of the crowd.
Unlike previous methods for inferring the consensus, 
such as \emph{CrowdBT}~\citep{chen2013pairwise}, we do not treat differences between individuals as simply the result of labelling errors, 
but instead assume that differences result from the users' subjective biases
towards particular items. 
 
% is there a better word than 'label sources' for the different sources of implicit feedback or other types of labeling?
%In a scenario with multiple users or label sources, 
For crowdGPPL, 
we represent utilities in a matrix, $\bs{F} \in \mathbb{R}^{N \times U}$,
with $N$ rows corresponding to items and $U$ columns corresponding to users. %, and entries are preference values.
We assume that $\bs F$ is the product of two low-dimensional matrices
plus a vector of consensus utilities, $\bs{t}$, of the $N$ items, as follows:
\begin{equation}
\bs{F} = \bs{V}^T \bs{W} + \bs{t},
\end{equation}
where $\bs{W} \in \mathbb{R}^{C \times U}$ is a latent representation
of the users,
$\bs{V} \in \mathbb{R}^{C \times N}$ is a latent representation of the items,
and $C$ is the number of latent \emph{components}, i.e., the dimension
of the latent representations.
The column $\bs v_{.,a}$ of $\bs V$, and the column $\bs w_{.,j}$ of $\bs W$,
 are latent vector representations of item $a$ and user $j$,
 respectively.
%Users with similar values for a certain feature will have similar preferences for 
%the subset of items with corresponding feature values. 
Each latent component corresponds to a utility function 
for certain items, which is shared by a subset of users.
For example, in the case of book recommendation,
one component could represent the membership of certain books to a particular
genre and the interests of certain users in that genre.
%Besides latent features, we may also observe a number of item features, $\bs x$,
%and . 
%There are two ways that observed features can be incorporated into the 
%model: (1) as additional dimensions of V or W (each feature cannot be a member of %both); (2) as input features on which V and W depend. The advantage of the latter is that 
%In the single user model, we assumed a single latent utility function, $f$, of the observed item features. 

For crowdGPPL, we assume that each row of $\bs V$, $\bs v_c=\{ 
v_c(\bs{x}_1),...,v_c(\bs{x}_N)\}$,  
contains evaluations of a latent function, 
$v_c\sim \mathcal{GP}(\bs 0, k_{\theta} /s^{(v)}_c)$,
of item features, $\bs x_a$,
where $k$ is a kernel function, $s^{(v)}_c$ is an inverse function scale,
and $\theta$ are kernel hyperparameters.
%Since our goal is to infer a consensus from a crowd as well as to model individual 
%users' preferences, 
We also assume that $\bs t = \{t(\bs {x}_1),...,t(\bs {x}_N)\}$
contains evaluations of a consensus utility function over item features,
$t\sim \mathcal{GP}(\bs 0, k_{\theta} /s^{(t)})$, which is shared across all users,
with inverse scale $s^{(t)}$.
Similarly, each row of $\bs W$, 
$\bs w_c=\{w_c(\bs u_1),...,w_c(\bs u_U)\}$,
 contains evaluations of a latent function,
$w_c \sim \mathcal{GP}(\bs 0, k_{\eta}/s_c^{(w)})$,
of user features, $\bs u_j$, 
with inverse scale $s_c^{(w)}$
and kernel hyperparameters $\eta$.
Therefore, the utilities in $\bs F$ are evaluations of a joint
preference function, $f$, which is a weighted sum over the latent functions:
\begin{flalign}
  f(\bs x_a, \bs u_j) = \sum_{c=1}^C  v_c(\bs x_a) w_c(\bs u_j) + t(\bs x_a),
\end{flalign}
where $\bs u_j$ are the features of user $j$ and $\bs x_a$ are the features of item $a$.

CrowdGPPL therefore combines latent features of items and
users -- represented by the latent components -- with the
utilities of the items according to an underlying consensus across users.
Given the consensus utility for item $a$, $t(\bs x_a)$,
the individual preferences of user $j$ then deviate from the consensus according
to $\sum_{c=1}^C  v_c(\bs x_a) w_c(\bs u_j)$. 
Hence, when inferring the consensus utilities, we can subtract the effect of individual
biases. The consensus can also help 
when inferring personal preferences for user-item combinations that are
not similar to the combinations in the training data, by
taking into account any objective quality or widespread appeal that an item may have.
%We provide a Bayesian treatment to matrix factorization by placing Gaussian process priors over the latent functions.
%differently for each user and item. For example, the observed user feature 'age' may correlate with some latent interests of users, but certain users will deviate from their peer group. 
% what happens if two users have identical features (say, the feature representation
% has only simple values, such as age in years)? They have 1-1 covariance, but there 
% is variance in the GP at one location, so both can be drawn separately from the prior.
%It is not necessary to learn a separate scale for $w_c$, since $v_c$ and $w_c$ are multiplied with each other, making a single $s^{(v)}_c$ equivalent to the product of two separate scales. 
%The choice of $C$ can be treated as a hyperparameter, or modeled using a non-parametric prior, such as 
%the Indian Buffet Process, which assumes an infinite number of latent components ~\citep{ding2010nonparametric}.
%This section described a Bayesian matrix factorization model, 
%which we will subsequently extend to a preference learning model for crowds of users and label sources. 
% joint distribution
% notes about problems with inference.

This form of equation is also used in the 
\emph{cross-task crowdsourcing} model of ~\citet{mo2013cross}, which 
uses matrix factorisation to model annotator performance in different tasks.
In their model, $\bs t$ corresponds to the difficulty of a task to all users.
However, unlike crowdGPPL, they do not use Gaussian processes to model the factors, 
nor apply
the approach to preference learning.
 
We combine the matrix factorization method with the preference likelihood of Equation \ref{eq:plphi}
to obtain a joint preference model for multiple users, \emph{crowdGPPL}:
%represent a consensus between users,
%if present, while allowing individual users' preferences to deviate from this value through $\bs V^T \bs W$. 
%Hence, $\bs t$ can model the underlying ground truth or consensus in crowdsourcing scenarios, or when using
%multiple label sources to learn preferences for one individual.
\begin{flalign}
p\left( \bs{y}, \bs V, \bs W, \bs t, s^{(v)}_1, .., s^{(v)}_C, s^{(w)}_1, .., s^{(w)}_C,
s^{(t)} | k_{\theta}, k_{\eta}, \alpha_0^{(t)}, \beta_0^{(t)},
\alpha_0^{(v)}, \beta_0^{(v)}, \alpha_0^{(w)}, \beta_0^{(w)} \right) 
 && \nonumber \\ 
= \prod_{p=1}^P \Phi\left( z_p \right) 
\mathcal{N}(\bs t; \bs 0, \bs K_{\theta} /s^{(t)})
\mathcal{G}({s^{(t)}}; \alpha_0^{(t)}, \beta_0^{(t)})
\prod_{c=1}^C \left\{
\mathcal{N}(\bs v_c; \bs 0, \bs K_{\theta} /s^{(v)}_c)
\right.
 && \nonumber \\  
\left.
\mathcal{N}(\bs w_c; \bs 0, \bs L_{\eta}/s^{(w)}_c) \mathcal{G}(s^{(v)}_c; \alpha_0^{(v)}, \beta_0^{(v)})\mathcal{G}(s^{(w)}_c; \alpha_0^{(w)}, \beta_0^{(w)}) \right\}, 
\label{eq:joint_crowd}
\end{flalign}
where 
$z_p = \bs v_{.,a_p}^T \bs{w}_{.,u_p} + t_{a_p} - \bs v_{.,b_p}^T \bs{w}_{.,u_p} - t_{b_p}$,
 $s^{(t)}$ is the inverse scale of $t$,
index $p$ refers to a tuple, $\{u_p, a_p, b_p \}$, which identifies the user and a pair of items,
and $L_{\eta}$ is the prior covariance between user feature vectors computed
using the kernel function $k_{\eta}$.

Although the model assumes a fixed number of components, $C$,
the GP priors over $\bs w_c$ and $\bs v_c$ have a zero mean, which act as \emph{shrinkage}
or \emph{ARD priors} that favour values close to zero~\citep{mackay1995probable,psorakis2011overlapping}. 
Those components that are not required to explain the data will have posterior
expectations and expected scales $1/s^{(v)}$ and $1/s^{(w)}$ approaching zero.
Therefore, due to our choice of prior, it is not necessary to optimise the value
of $C$, providing a sufficiently large number is chosen. 
