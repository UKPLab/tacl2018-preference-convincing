\section{Bayesian Preference Learning for Crowds}\label{sec:model}

%%%% Notes

% Title or name of the model: 
% -- cannot decide this until we get most of the paper complete: will emphasis be on crowds? distilling ground truth
% from noisy sources (Bayesian preference learning for fusing unreliable sources)? user preferences/collaborative filtering?
% -- need a new name to differentiate from Houlsby et al. and Khan et al.?
% -- what are the differences in the model? Let's get the model written up.
% -- should also be some buzzword or word to generate interest: 
%    -- 'variational' is on the up, could be used in paper title
%    -- 'stochastic variational' is also on the up
%    -- 'crowdsourcing' on the way down as at 2010 level
%    -- 'gaussian process' on the way up
%    -- 'matrix factorization' kind of on the way up
%    -- 'scalable' on the way up
%    -- 'interactive learning' on the way down
%    -- 'preference learning' flattish, may be on way up slowly
% -- aimed at crowdsourcing problems (uses a common mean function as consensus?)
% -- other parameters for importance of features?
% -- combined preference learning? Preference aggregation? Collaborative crowdsourced preference learning? Bayesian preference learning for crowds? another word for 'multi-user' or 'many users and many items' vs. crowds?

% TO ADD: Why does the variance in f cancel out when predicting the probability of a pairwise label?
% TO ADD: Why does \sigma disappear if we learn the output scale.

% We also estimate the output scale of the GPs, the latent components, and item bias as part of the 
% variational approximation allowing us to estimate these parameters in a Bayesian manner without 
% resorting to maximum likelihood approaches.

% mention how the noise model deals with inconsistencies in preferences

% \begin{enumerate}
% \item What are the benefits of Bayesian methods and Gaussian processes in particular?
% \item The proposal by \citep{chu2005preference} shows how the advantages of a Bayesian
% approach can be exploited for preference learning by modifying the observation model

% Extensions:
% -- how do we replace the GP with a NN?
% -- would this move us from a Bayesian to an ML solution?
% -- maybe save this for future work? Or add a few lines if we can make it fit with the theme of the paper.
% -- another is to replace the fixed number of clusters with a CRP, then the whole thing can be nonparameteric preference learning with crowds.

\subsection{Modeling Pairwise Preferences}

% Include the case for one user -- preferences may depend on a number of observed features.

% this should be introduced in section 1. \citep{handleycomparative} -- compares BT with TM models
A pairwise comparison, $y(a \succ b)$, between items $a$ and $b$ 
has a binary label that is one if $a$ is preferred to $b$, 
or zero if $b$ is preferred to $a$ (also written $a \prec b$).
We assume that the likelihood of pairwise label $y(a,b)$ depends on the underlying
value of the items to the user, represented through a latent function of the items' features,
$f(\bs x_a)$, where $\bs x_b$ is a vector representation of the features of item $a$.
The relationship between the value function, $f$, and the pairwise labels
can be modeled by any of several different likelihood functions,
including the Bradley-Terry model~\citep{bradley1952rank,plackett1975analysis,luce1959possible}
and the Thurstone-Mosteller model~\citep{thurstone1927law,mosteller2006remarks}.
The Bradley-Terry model takes the following form:
\begin{align}
p(y(a \succ b) | f) & = \frac{1}{1 + \exp( f(\bs x_a) - f(\bs x_b) ) }
\end{align}
This is a logistic likelihood, which allows pairwise labels that do not reflect 
the true relative values of the items, due to labeling errors, variability in the user's judgements, or if
the preferences are derived from noisy implicit data such as clicks streams.
The error rate is determined by the relative difference in $f$ values of the items.

A different view is to treat the errors as the result of random noise in the value function:
\begin{align}
 p(y(a \succ b) | f, \delta_{a}, \delta_{b} )  
 \hspace{0.9cm} & = \begin{cases}
 1 & \text{if }f(\bs x_a) + \delta_{a} \geq f(b) + \delta_{b} \\
 0 & \text{otherwise,}
 \end{cases} &
 \label{eq:thurstone}
\end{align}
where $\delta \sim \mathcal{N}(0, 0.5)$ is Gaussian-distributed noise.
Integrating out the unknown values of $\delta_a$ and $\delta_b$,
we get a probit likelihood:
\begin{align}
p( y(a \succ b) | f ) 
& = \int\int p( y(a \succ b) | f, \delta_{a}, \delta_{b} ) \mathcal{N}(\delta_{a}; 0, 0.5)\mathcal{N}(\delta_{b}; 0, 0.5) d\delta_{a} d\delta_{b} &\nonumber\\
& = \Phi\left( z \right), 
\label{eq:plphi}
\end{align}
where $z = f(\bs x_a) - f(\bs x_b)$,
and $\Phi$ is the cumulative distribution function of the standard normal distribution. 
This is a Thurstone-Mosteller model, sometimes referred to as \emph{Thurstone case V}, and was used for Gaussian process preference learning (GPPL) 
by \citet{chu2005preference}, with the difference that they learned the variance of the random noise, $\delta$
rather than assuming it is $0.5$. However, this is unnecessary in practice, since we 
scale instead the value function, $f$, to reduce or increase the certainty in the pairwise labels.
Both the logistic and probit approaches can be used here, but we proceed with
the probit likelihood (as in \citep{herbrich2007trueskill,chu2005preference}
because it allows us to handle uncertainty in $f$ in a simple manner %as Gaussian noise
by modifying $z$:
% we haven't mentioned that f is Gaussian yet. 
\begin{align}
\hat{z} = \frac{\mu_a - \mu_b}{\sqrt{1 + \sigma_a + \sigma_b - \sigma_{a,b}} }
\end{align}
where $\mu_a$ and $\mu_b$ are the expected values of $f(\bs x_a)$ and $f(\bs x_b)$ respectively, 
$\sigma_a$ and $\sigma_b$ are the corresponding variances,
and $\sigma_{a,b}$ is the covariance between $f(\bs x_a)$ and $f(\bs x_b)$.

\subsection{Single User Preference Learning}

First consider modeling the preferences of a single user. In this case, we assume that the value function, 
$f$, is a function of item features and has a Gaussian process prior: 
$f \sim \mathcal{GP}(0, k_{\theta}/s)$, where $k_{\theta}$ is a kernel function with hyper-parameters $\theta$, 
and $1/s$ is the scale of the function
drawn from a gamma prior, $s \sim \mathcal{G}(\alpha_0, \beta_0)$, with shape $\alpha_0$ and scale $\beta_0$.
The value of $s$ determines the variance of $f$ and therefore its magnitude, which affects the level of certainty
in the pairwise label likelihood, Equation \ref{eq:plphi}.
The kernel function takes item features as inputs and determines the covariance between values of $f$ for different items. Typically, we choose a kernel function that produces higher covariance between items with similar feature values,
such as the \emph{squared exponential} or \emph{Mat\'ern} functions.
The choice of kernel function is a model selection problem as it controls the shape and smoothness of the function 
across the feature space. However, the Mat\'ern and squared exponential make minimal assumptions and so are effective in a wide range of tasks (see ~\citet{rasmussen_gaussian_2006} for more). 

% Covariance of different items with same feature: worth mentioning here? We need to multiply the kernel by a small
% amount < 1 to ensure covariance is not 1 and the values can differ.

We observe a set of $P$ pairwise preference labels for a single user, $\bs y=\{y_1,...,y_P\}$,
where the $p$th label, $y_p=y(a_p \succ b_p)$.% refers to items $\{ a_p, b_p \}$.
The joint distribution over all variables is as follows:
\begin{flalign}
p\left( \bs{y}, \bs f, s | k_{\theta}, \alpha_0, \beta_0 \right) 
=  \prod_{p=1}^P p( y_p | \bs f ) 
\mathcal{N}(\bs f; \bs 0, \bs K_{\theta}/s) \mathcal{G}(s; \alpha_0, \beta_0) \nonumber \\
=  \prod_{p=1}^P \Phi\left( z_p \right) 
\mathcal{N}(\bs f; \bs 0, \bs K_{\theta}/s) \mathcal{G}(s; \alpha_0, \beta_0), &
\label{eq:joint_single}
\end{flalign}
where $\bs f = \{f(\bs {x}_1),...,f(\bs {x}_N)\}$
are the latent values for the $N$ items referred to by the pairwise labels 
$\bs y$, and $\theta$, $\alpha_0$ and $\beta_0$ are hyper-parameters.

\subsection{Latent components: Bayesian Matrix Factorization}

% 0. Consider multiple users, each with a set of observed features.
% 1. imagine augmenting the observed features with a number of latent features (this is kind of what Khan et al. do)
% 2. imagine that the latent features relate preference function values between items and users using the smallest number of features
% 3. hence the observed features are not needed given the latent features, but we can create a hierarchical model where the latent features depend on the observed ones if available.

We wish to exploit similarities between the value functions of different users or label sources to improve our preference model, particularly when faced with sparse data.
% is there a better word than 'label sources' for the different sources of implicit feedback or other types of labeling?
In a scenario with multiple users or label sources, 
we can represent preference values in a matrix, $\bs{F}$,
where rows correspond to items, columns to users, and entries are preference values.
If we factorize this matrix, we obtain two lower-dimensional matrices,
one for users, $\bs{W} \in \mathcal{R}^{C \times U}$, 
and one for the items, $\bs{V} \in \mathcal{R}^{N \times C}$,
where $C$ is the number of latent components, 
$U$ is the number of users, and $N$ is
the number of items: $\bs{F} = \bs{V}^T \bs{W}$.
Each row of $V$ matrices is a vector representation of an item, 
while each row of $W$ is a vector representation of a user, 
both containing the values of latent features.
Users with similar values for a certain feature will have similar preferences for 
the subset of items with corresponding feature values. 
The features could represent, for example, in the case of book recommendation, interests in a particular genre of book. 
Using vector representations for users and items reflects that users may have overlapping sets of interests,
and that items may have multiple features that make them attractive.

Besides latent features, we may also observe a number of item features, $\bs x$,
and user features, $\bs u$. 
%There are two ways that observed features can be incorporated into the 
%model: (1) as additional dimensions of V or W (each feature cannot be a member of %both); (2) as input features on which V and W depend. The advantage of the latter is that 
In the single user model, we assumed a single latent value function, $f$, of the observed item features. 
For the multi-user case, we assume that there are $C$ latent functions, $v_c$ over 
item features and $C$ latent functions, $w_c$, over user features,
and thereby model the relationship between each observed feature and each of the latent features. 
The matrices $V$ and $W$ are evaluations of these functions at the points corresponding to 
the observed users and items.
Therefore, the latent preference function, $f$, for a user with features $\bs u$ is 
a weighted sum over latent functions:
\begin{align}
  f(\bs x_a, \bs u_j) = \sum_{c=1}^C w_c(\bs u_j) v_c(\bs x_a)
\end{align}
%differently for each user and item. For example, the observed user feature 'age' may correlate with some latent interests of users, but certain users will deviate from their peer group. 
% what happens if two users have identical features (say, the feature representation
% has only simple values, such as age in years)? They have 1-1 covariance, but there 
% is variance in the GP at one location, so both can be drawn separately from the prior.
To provide a Bayesian treatment to matrix factorization, we place Gaussian process priors over the latent functions:
\begin{align}
v_c \sim \mathcal{GP}(\bs 0, k_{\theta} /s_c) & & w_c \sim \mathcal{GP}(\bs 0, k_{\theta}).
\end{align}
It is not necessary to learn a separate scale for $w_c$, since $v_c$ and $w_c$ are multiplied with each other, making a single $s_c$ equivalent to the product of two separate scales. 
The choice of $C$ can be treated as a hyperparameter, or modeled using a non-parametric prior, such as 
the Indian Buffet Process, which assumes an infinite number of latent components ~\citep{ding2010nonparametric}.
For simplicity, we assume fixed values of $C$ in this paper, and allow the scale parameter $s_c \approx 0$ 
to effectively remove any dimensions that are not required to model the data.
This section described a Bayesian matrix factorization model, 
which we will subsequently extend to a preference learning model for crowds of users and label sources. 

%TODO: include a mention of nonparametric IBP priors over infinite C. 

\subsection{Crowd Preference Learning} \label{sec:crowd_model}

% joint distribution
% notes about problems with inference.

We combine the matrix factorization method with the preference likelihood of Equation \ref{eq:plphi}
to obtain a joint preference model for multiple users or label sources.
In addition to the latent components, we introduce a common value function over item features, 
$t\sim \mathcal{GP}(\bs 0, k_{\theta} /\sigma_t)$, 
that is shared across all users. 
Its values $\bs t = \{t(\bs {x}_1),...,t(\bs {x}_N)\}$ represent a consensus between users,
if present, while allowing individual users' preferences to deviate from this value through $\bs V^T \bs W$. 
Hence, $\bs t$ can model the underlying ground truth in crowdsourcing scenario, or when using
multiple label sources to learn preferences for one individual.
The joint distribution of this crowd model is:
\begin{flalign}
p\left( \bs{y}, \bs V, \bs W, \bs t, s_1, ..., s_C, \sigma_t | k_{\theta}, \alpha_0, \beta_0 \right) 
= & \prod_{p=1}^P \Phi\left( z_p \right) 
\mathcal{N}(\bs t; \bs 0, \bs K_{t,\theta} /\sigma_t)
\mathcal{G}({\sigma_t}; \alpha_0, \beta_0) \nonumber \\
& \hspace{-2.6cm} \prod_{c=1}^C \left\{
\mathcal{N}(\bs v_c; \bs 0, \bs K_{v,\theta} /s_c) 
\mathcal{N}(\bs w_c; \bs 0, \bs K_{w,\theta}) \mathcal{G}(s_c; \alpha_0, \beta_0) \right\}, \\
\textrm{where } z_p = &  \bs v_{.,a_p}^T \bs{w}_{.,j_p} - \bs v_{.,b_p}^T \bs{w}_{.,j_p}, &
\label{eq:joint_crowd}
\end{flalign}
and $\sigma_t$ is the inverse scale of $t$.
The index $p$ now refers to a tuple, $\{j_p, a_p, b_p \}$ that identifies the user and a pair of items.
