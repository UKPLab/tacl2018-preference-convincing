% Note that Khan's method does not need factorization assumptions in the approximate posterior.
% Instead, they have no prior over v (item features).
% They have a diagonal covariance for the user features -- cheap. 
% They need a separate GP per user but it seems like this is not a problem in practice -- I guess
% the method scales linearly with no. users. In our case, we model covariance between users, so
% scaling is poor unless you can use inducing points or diagonal covariance.


\section{Related Work}
\label{sec:rw}

%\paragraph{Pairwise Preference Learning.}
To obtain a ranking from pairwise labels, 
many preference learning methods model
the user's choices as a random function of the latent 
\emph{utility} of the items.
Inferring the utilities of items allows us to rank them, estimate numerical ratings
and predict pairwise labels.
Many popular instances of this approach, known as a \emph{random utility model}~\citep{thurstone1927law},
are variants of  
the Bradley-Terry (BT) model~\citep{bradley1952rank,plackett1975analysis,luce1959possible},
which assumes a logistic likelihood,
or the Thurstone-Mosteller model
%, also known as \emph{Thurstone case V}
~\citep{thurstone1927law,mosteller2006remarks},
which assumes a probit likelihood.
%Examples include best-worst scaling~\citep{marley2005some}, which extends the BT model,
%and TrueSkill~\citep{herbrich2007trueskill}, a Thurstone-Mosteller-based model that learns 
%the skill of game players by treating match outcomes as noisy pairwise labels.
Recent work on the BT model has 
% analysed bounds on error rates~\citep{chen2015spectral}, 
%sample complexity~\citep{shah2015estimation} and 
developed computationally efficient active learning, but does not consider input 
features~\citep{li2018hybrid}.
%However, these examples do not consider input features.
Another commonly-used ranking method, SVM-rank~\citep{joachims2002optimizing},
predicts pairwise labels from input features 
%but optimises pairwise label prediction directly 
without a random utility model, so cannot predict utilities.
\emph{Gaussian process preference learning (GPPL)}
provides a Bayesian treatment of the random utility model,
using input features to predict the utilities of test items and share information
between similar items~\citep{chu2005preference}.
As GPPL can only predict the preferences of a single user,  
we introduce a new, scalable approach to model individuals in a crowd. 
%the preferences of crowds of users.

%\paragraph{Annotator Disagreements when Estimating a Consensus.}
Previous work on preference learning from crowdsourced data 
treats disagreements as annotation errors and infers only the consensus,
rather than modelling personal preferences.
For instance, 
\citet{chen2013pairwise} and \citet{wang2016blind} 
tackle annotator disagreement %between annotators in a crowd 
using Bayesian approaches that learn the labelling accuracy of each worker.
Recently, \citet{pan2018stagewise} and \citet{han2018robust} 
introduced scalable methods that extend this idea from pairwise labels
to noisy \textit{k}-ary preferences,
i.e., totally-ordered subsets of $k$ items.
\citet{fu2016robust} improved SVM-rank by identifying outliers in crowdsourced data
that correspond to probable errors,
while \citet{uchida2017entity} extend SVM-rank to account for different levels of confidence in each pairwise annotation expressed by the annotators.
%TODO decide if this should be added.
%For crowdsourced classification tasks, \citet{simpson2017bayesian} show that modelling 
%individual annotator noise levels in combination with a Gaussian process can improve performance, but this has not yet been adapted for preference learning.
However, while these approaches differentiate the level of \emph{noise}
for each annotator,
they ignore labelling \emph{bias} 
as the differences between users are not random but depend on personal preferences toward particular items.
With small numbers of labels per item, these biases may reduce the accuracy of the estimated
consensus.
Furthermore, previous aggregation methods for crowdsourced preferences
do not consider item features,
so cannot predict the utility of test items~\citep{chen2013pairwise,wang2016blind,han2018robust,pan2018stagewise,li2018hybrid}.
Our approach goes beyond these methods
by predicting personal preferences
and incorporating input features.

%\paragraph{Inferring Personal Preferences for Members of a Crowd. }
A number of methods use \emph{matrix factorisation} to predict personal preferences 
from pairwise labels, including 
\citet{yi_inferring_2013}, who focus on small numbers of pairs per user,
and \citet{salimans2012collaborative}, who apply Bayesian matrix factorisation to 
handle sparse data.
Matrix factorisation represents observed ratings in a user-item matrix,
which it decomposes into two low-rank matrices, one corresponding to users and one to items.
Users with similar ratings have similar rows in the user
matrix, where each entry is a weight over a latent rating.
By multiplying the low-dimensional representations, we can predict ratings for unseen
user-item pairs. 
\citet{kim2014latent} use a simplification that assumes that
each user's preferences depend on only one latent rankings.
However, these techniques do not account for input features.

%crowdranking \citet{yi_inferring_2013} uses the crowd to make up for the fact that a target user has small data -- it's a form of collaborative filtering with pairwise labels using a non-Bayesian inference algorithm.
% Several other works learn multiple rankings from crowdsourced pairwise labels
% rather than a single gold-standard ranking, 
% but do not consider the item or user features so cannot extrapolate to new users or 
% items~\citep{yi_inferring_2013,kim2014latent}. 
% Both \citet{yi_inferring_2013} and learn a small number of
% latent ranking functions that can be combined to construct personalized preferences, 
% although neither provide a Bayesian treatment to handle data sparsity.
%include the work on collaborative GPPL
%Several extensions of BMF use Gaussian process priors over latent factors 
%to model correlations between 
%items given side information or observed item features~\citep{adams2010incorporating,zhou2012kernelized,houlsby2012collaborative,bolgar2016bayesian}. 
%However, these techniques are not directly applicable to 
%learning from pairwise comparisons 
%as they assume that the observations are Gaussian-distributed numerical ratings~\citep{shi2017survey}. 

GPs were previously used 
for personal preference prediction
by \citet{guo2010gaussian}, who propose a GP over the joint feature 
space of users and items. Since this scales cubically
in the number of users, \citet{abbasnejad2013learning} 
propose to cluster users into behavioural groups,
but distinct clusters do not
allow for collaborative learning between users whose preferences only partially overlap, 
e.g. when two users both like one genre of music, 
but have different preferences over other genres. 
\citet{khan2014scalable} instead learn a GP for each user,
then add a matrix factorisation term that performs collaborative filtering.
However, this approach does not model the relationship between
 input features and the low-rank matrices,
 unlike \citet{lawrence2009non} who place GP priors over latent ratings.
 Neither of these last two methods
 are fully Bayesian as the users' weights
 are optimised rather than marginalised.
An alternative is the \emph{collaborative GP (collabGP)}~\citep{houlsby2012collaborative},
which places GP priors over user weights and latent factors,
thereby exploiting input features for both users and items. 
However, unlike our approach, collabGP predicts only pairwise labels, not 
%over pairs also means that it 
%does not model %output posteriors for 
the utilities of items, which are useful for rating and ranking,
and can only be trained using pairwise labels, even if observations of the utilities
are available.
Furthermore, existing GP-based approaches
suffer from scalability issues and 
none of the previous methods jointly model the consensus as well as personal preferences
in a fully-Bayesian manner.

%which is needed to handle small amounts of data for both users and items
%and account for personal biases when inferring the consensus.
 % PCA: Gaussian noise. "The classical PCA converts a set of samples with possibly correlated variables into another   
 % set of samples with linearly uncorrelated variables via an orthogonal transformation [1]. Based on this, PCA
 % is an effective technique widely used in performing dimensionality reduction and extracting features." -- Shi et al 2017. shi2017survey
 % SVD: like PCA with the mean vector set to zeroes.
 % variations of PCA: for handling outliers or large sparse errors
 % most matrix factorizations are special cases of PCA and in practice do not consider the mean vector.
 % probabilistic PCA: latent variables are unit isotropic Gaussians --> all have 0 covariance and 1 variance.
 % Bayesian PCA: places priors on all latent variables.
 % Probabilistic factor analysis: assumes different variances on each of the latent factors.
 % Probabilistic matrix factorisation: ignores the mean. --> I.e. can be done with SVD
 % I think this means our method is a form of PFA? But extended to consider correlations in the weights.
 % NMF: as matrix factorisation but the low-rank matrices are non-negative.


%\paragraph{Scalable Approximate Bayesian Inference.}

%Scalability of BMF
%Many of the approaches for modelling individual user preferences use matrix factorization 
%to share information between users and items,
%which 
%Matrix factorisation can benefit from a Bayesian treatment to reduce overfitting or 
%handle noisy data. 
%Models that combine Gaussian processes with non-Gaussian likelihoods 
%require approximate inference methods that often scale poorly with 
%the amount of training data. 
Established methods for GP inference with non-Gaussian likelihoods,
such as the Laplace approximation 
and expectation propagation~\citep{rasmussen_gaussian_2006}, have
time complexity $\mathcal{O}(N^3)$ with $N$ data points
 and memory complexity $\mathcal{O}(N^2)$. 
For collabGP, \citet{houlsby2012collaborative}
%propose a  kernel for pairwise 
%preference learning and 
use a sparse \emph{generalized fully independent training conditional} (GFITC) 
approximation~\citep{snelson2006sparse} to reduce time complexity to $\mathcal{O}(PM^2 + UM^2)$ and 
memory complexity to $\mathcal{O}(PM + UM)$,
where $P$ is the number of pairwise labels, 
$M \ll P$ is a fixed number of inducing points, 
and $U$ is the number of users.
However, this is not sufficiently scalable
for very large numbers of users or pairs, 
due to increasing memory consumption 
and optimisation steps that cannot be distributed. %run in parallel on subsets of the data, 
%since the objective function is a not a sum over data points.
%The marginal likelihood does not factor between data points -- updating for one data point may not 
% improve the marginal likelihood. If the objective function is a sum of terms, we can do gradient 
% descent steps on each term independently.  
%Updates  to the individual factors are not gradient descent steps for the overall objective 
%-- local objective only.
% This means they have some sequential dependencies on one another, so I don't know what
% happens if you do them out of order.
Recent work on distributing and parallelising Bayesian matrix factorisation 
is not easily applicable to models that incorporate GPs
 ~\citep{ahn2015large,saha2015scalable,vander2017distributed,chen2018large}. 
 
To handle large numbers of pairwise labels, \citet{khan2014scalable}
%develop a variational EM algorithm and 
subsample the data rather than learning from the complete training set.
An alternative is \emph{stochastic variational inference (SVI)}~\citep{hoffman2013stochastic}, 
which optimises a posterior approximation using 
a different subsample of training data at each iteration, meaning it learns from
all training data over multiple iterations while limiting costs per iteration.
SVI has been applied to GP regression~\citep{hensman2013gaussian} and classification~\citep{hensman2015scalable},
further improving scalability over earlier sparse approximations.
~\citet{nguyen2014collaborative} introduce SVI for multi-output GPs,
where each output is a weighted combination of latent functions.
% plus a latent function specific to that output. 
They apply their method to capture dependencies between regression tasks,
treating the weights for the latent functions as hyperparameters. 
In this paper, we 
%also use shared latent functions to capture dependencies between different users' preferences,
%but 
introduce a Bayesian treatment of the weights %using GPs % over latent user features,
and apply SVI instead to preference learning.
% with a non-Gaussian likelihood into the SVI framework to enable learning from pairwise labels.
An SVI method for GPPL
%preference learning that places a GP over items, rather than pairs, 
was previously introduced by \citet{simpson2018finding},
which we detail in Section \ref{sec:inf}.
%However, this method, a variant of GPPL,
However, as GPPL does not consider the individual preferences of users in a crowd,
 we propose a new model, crowdGPPL, which
jointly models personal preferences and the crowd consensus
using a combination of Gaussian processes and Bayesian matrix factorisation.
